% Arquivo LaTeX de exemplo de dissertação/tese a ser apresentados à CPG do IME-USP
% 
% Versão 5: Sex Mar  9 18:05:40 BRT 2012
%
% Criação: Jesús P. Mena-Chalco
% Revisão: Fabio Kon e Paulo Feofiloff
%  
% Obs: Leia previamente o texto do arquivo README.txt

\documentclass[12pt,twoside,a4paper]{book}



% ---------------------------------------------------------------------------- %
% Pacotes 
\usepackage[T1]{fontenc}
\usepackage[brazil]{babel}
\usepackage[utf8]{inputenc}
\usepackage[pdftex]{graphicx}           % usamos arquivos pdf/png como figuras
\usepackage{setspace}                   % espaçamento flexível
\usepackage{indentfirst}                % indentação do primeiro parágrafo
\usepackage{makeidx}                    % índice remissivo
\usepackage[nottoc]{tocbibind}          % acrescentamos a bibliografia/indice/conteudo no Table of Contents
\usepackage{courier}                    % usa o Adobe Courier no lugar de Computer Modern Typewriter
\usepackage{type1cm}                    % fontes realmente escaláveis
\usepackage{listings}                   % para formatar código-fonte (ex. em Java)
\usepackage{titletoc}
\usepackage{enumerate}
\usepackage{float}
%\usepackage[bf,small,compact]{titlesec} % cabeçalhos dos títulos: menores e compactos
\usepackage[fixlanguage]{babelbib}
\usepackage[font=small,format=plain,labelfont=bf,up,textfont=it,up]{caption}
\usepackage[font=footnotesize]{subcaption}
%\usepackage{subfigure}
\usepackage[usenames,svgnames,dvipsnames]{xcolor}
\usepackage[a4paper,top=2.54cm,bottom=2.0cm,left=2.0cm,right=2.54cm]{geometry} % margens
%\usepackage[pdftex,plainpages=false,pdfpagelabels,pagebackref,colorlinks=true,citecolor=black,linkcolor=black,urlcolor=black,filecolor=black,bookmarksopen=true]{hyperref} % links em preto
%\usepackage[pdftex,plainpages=false,pdfpagelabels,pagebackref,colorlinks=true,citecolor=DarkGreen,linkcolor=NavyBlue,urlcolor=DarkRed,filecolor=green,bookmarksopen=true]{hyperref}
%hyperref
\usepackage[pdftex,plainpages=false,pdfpagelabels,pagebackref,colorlinks=true,citecolor=DarkGreen,linkcolor=NavyBlue,urlcolor=DarkRed,filecolor=Green,bookmarksopen=true]{hyperref}
% links coloridos

\captionsetup[subcaption]{font=footnotesize}

% ---------------------------------------------------------------------------- %
% fontes e símbolos
\usepackage{amsfonts}
\usepackage{textcomp}

% multicols wrapfig
\usepackage{multicol}
\usepackage{wrapfig}

% ---------------------------------------------------------------------------- %
% Teoremas
\usepackage{amsthm}
\newtheorem{theorem}{Teorema}[section]
\newtheorem{lemma}[theorem]{\bf{Lema}}

% ---------------------------------------------------------------------------- %
% Equações Numeradas
\usepackage{amsmath}
\numberwithin{equation}{section}
%\numberwithin{equation}{subsection}


% % links coloridos
\usepackage[all]{hypcap}                    % soluciona o problema com o hyperref e capitulos
\usepackage[round,sort,nonamebreak]{natbib} % citação bibliográfica textual(plainnat-ime.bst)
\bibpunct{(}{)}{;}{a}{\hspace{-0.7ex},}{,} % estilo de citação. Veja alguns exemplos em http://merkel.zoneo.net/Latex/natbib.php

\fontsize{60}{62}\usefont{OT1}{cmr}{m}{n}{\selectfont}

% ---------------------------------------------------------------------------- %
\DeclareMathOperator*{\argmin}{arg\,min}
\DeclareMathOperator*{\argmax}{arg\,max}
\DeclareMathOperator*{\erf}{erf}
% ---------------------------------------------------------------------------- %

% ---------------------------------------------------------------------------- %
% Glossarios, Acronimos e Simbolos, 
%\usepackage{hyperref}                    % 
\usepackage[sanitize=none,acronym,toc]{glossaries}
%\usepackage[acronym,toc]{glossaries}
% Define a new glossary type
\newglossary[slg,toc]{symbols}{sym}{sbl}{Lista de S\'imbolos}
\newglossary[elg]{equations}{eqn}{eql}{Equations}
\makeglossaries


%\usepackage{imakeidx}
\usepackage{imakeidx} % para o índice remissivo
\makeindex[intoc]    

% ---------------------------------------------------------------------------- %
% Dicionario de Termos:
%-----------------------------------------
% terms
%-----------------------------------------
\newglossaryentry{equake}
{
	name={terremoto},
	description={ruptura de alguma estrutura geológica},
	plural={terremotos}
}

\newglossaryentry{seismicity}
{
	name={sismicidade},
	description={ocorrência dos tremores},
}

\newglossaryentry{hypocenter}
{
	name={hipocentro},
	description={representação geométrica do ponto no espaço, onde 
		se iniciou o \gls{rupture_process} da \gls{crust}},
	plural={hipocentros}
}

\newglossaryentry{epicenter}
{
	name={epicentro},
	description={projeção ortogonal, sobre a superfície, do \gls{hypocenter}},
	plural={epicentros}
}


\newglossaryentry{seismotectonic}
{
	name={sismotectônica},
	description={o estudo das relações entre os \glspl{equake} e a \gls{tectonic} recente de uma região.
				 Procura compreender exatamente quais são os mecanismos que levam à uma ruptura geológica 
				 e são responsáveis pela
				 \gls{seismic_activity} em uma certa área. Isso é feito analisando-se de forma combinada 
				 registros recentes de tectonismo global e regional, 
				 considerando também evidências históricas e geomorfológicas},
}


\newglossaryentry{rupture_process}
{
	name={processo de ruptura},
	description={processo que envolve o rompimento de uma região da crosta,
			o deslocamento relativo entre essas regiões, e consequantemente,
			a liberação de uma grande quantidade de energia, de forma praticamente
			instantânea, tomando-se como referência o \gls{geologic_time}},
	plural={processos de ruptura}
}


\newglossaryentry{geologic_time}
{
	name={tempo geológico},
	description={escala de tempo que vai desde a formação do universo até os tempos atuais,
				englobando a formação do planeta e as transformações ocorridas desde então},
}


\newglossaryentry{tectonic}
{
	name={tect\^onica},
	description={disciplina científica focada nos processos respons\'aveis 
				 pela cria\c{c}\~ao e transforma\c{c}\~ao das estruturas geológicas da Terra e de outros planetas.},
	plural={tect\^onicas}
}


%\newglossaryentry{oq}
%{
%	name={OpenQuake},
%	description={programa de código aberto para o calculo de risco sísmico mantido pela Fundação \gls{gem}},
%	plural={OpenQuake}
%}


\newglossaryentry{crust}
{
	name={crosta terrestre},
	description={parte superficial, rígida e mais externa do planeta Terra},
}

\newglossaryentry{mantle}
{
	name={manto terrestre},
	description={material da por{ç}{ã}o intermediária do planeta, 
		fluido em tempo geológico},
}

\newglossaryentry{core}
{
	name={n{ú}cleo terrestre},
	description={por{ç}{ã}o mais central do planeta, com predomin{â}ncia de compostos metálicos},
}

\newglossaryentry{tectonic_plate_theory}
{
	name={teoria tect{ô}nica das placas},
	description={foi uma teoria revolucionária para a \gls{tectonic},
				propondo que a \gls{crust} terrestre estivesse dividida 
				em placas {à} deriva sobre o \gls{mantle}},
}


\newglossaryentry{litho_plate}
{
	name={placa litosf{é}rica},
	plural={placas litosf{é}ricas},
	description={placa de material da \gls{lithosphere}},
}


\newglossaryentry{lithosphere}
{
	name={litosfera},
	description={região rúptil, mais externa do planeta, formada pela \gls{crust} 
		(continental e ocêanica) e parte do \gls{mantle} superior, com aproximadamente 
		60\gls*{sym:km} de profundidade},
}


\newglossaryentry{astenosphere}
{
	name={astenosfera},
	description={região dúctil entre a \gls{lithosphere} e o \gls{mantle},
				com profundidades que variam de 60 a 700km},
}

\newglossaryentry{smoothing}
{
	name={técnicas de suavização},
	description={consiste em capturar importantes feições do conjunto de dados,
				 eliminando ruídos e outras estruturas de curto comprimento de onda
				 presentes nos dados},
}

\newglossaryentry{kernel_function}
{
	name={função de núcleo},
	description={funções n-dimensionais, cuja integral em todo o domínio resulta em 1,
				 podendo ser usadas como estimativas para 
				 funções de densidade de probabilidade},
	plural={funções de kernel},
}

\newglossaryentry{seismic_rate}
{
	name={taxa de sismicidade},
	description={taxa com que terremotos são produzidos por determinada \gls{seismic_source}},
	plural={taxas de sismicidade},
}

\newglossaryentry{seismic_activity}
{
	name={atividade sísmica},
	description={frequ{ê}cia de ocorr{ê}ncia de \glspl{equake}},
}

\newglossaryentry{poisson_process}
{
	name={processo de Poisson},
	description={uma sequencia de intervalos discretos com um experimento de Bernoulli em cada},
}

\newglossaryentry{seismic_source}
{
	name={fonte sísmica},
	description={estrutura geológica capaz de produzir tremores de terra},
	plural={fontes sísmicas}
}

\newglossaryentry{point_source}
{
	name={fonte sísmica pontual},
	description={representação geométrica por um ponto, de uma fonte sísmica},
	plural={fontes sísmicas pontuais},
}

\newglossaryentry{gmpe}
{
	name={GMPE},
	description={Equação de predição do movimento do chão},
	plural={GMPEs},
}


\newglossaryentry{area_source}
{
	name={fonte sísmica poligonal},
	description={representação geométrica por um polígono em superfície, 
				 de uma fonte sísmica},
	plural={fontes sísmicas poligonais},
}

\newglossaryentry{titulo_da_dissertacao}
{
	name={titulo_da_dissertacao},
	description={Técnicas de suavização aplicadas
					à caracterização de fontes sísmicas e 
					à análise probabilística de ameaça sísmica},
}

\newglossaryentry{isocista}
{
	name={isocista},
	description={curva que une valores de mesma intensidade sísmica},
}

       
%-----------------------------------------
% symbols
%-----------------------------------------

\newglossaryentry{sym:t}
{
	name={\ensuremath{t}},
	description={time},
	symbol={\ensuremath{t}},
	type=symbols
}

\newglossaryentry{sym:P}
{
	name={\ensuremath{P}},
	description={probability},
	symbol={\ensuremath{P}},
	type=symbols
}

\newglossaryentry{sym:E}
{
	name={\ensuremath{E}},
	description={expected value},
	symbol={\ensuremath{E}},
	type=symbols
}

\newglossaryentry{sym:Var}
{
	name={\ensuremath{Var}},
	description={variance},
	symbol={\ensuremath{Var}},
	type=symbols
}

\newglossaryentry{sym:epsilon}
{
	name={\ensuremath{\epsilon}},
	description={error},
	symbol={\ensuremath{\epsilon}},
	type=symbols
}

\newglossaryentry{sym:sigma}
{
	name={\ensuremath{\sigma}},
	description={standard deviation},
	symbol={\ensuremath{\sigma}},
	type=symbols
}


\newglossaryentry{sym:r}
{
	name={\ensuremath{\boldsymbol{r}}},
	description={space locallity},
	symbol={\ensuremath{\boldsymbol{r}}},
	type=symbols
}


\newglossaryentry{sym:m}
{
	name={\ensuremath{m}},
	description={magnitude},
	symbol={\ensuremath{m}},
	type=symbols
}


\newglossaryentry{sym:lambda}
{
	name={\ensuremath{\lambda}},
	description={seismic rate regressor},
	symbol={\ensuremath{\lambda}},
	type=symbols
}

\newglossaryentry{sym:M_0}
{
	name={\ensuremath{M_0}},
	description={seismic moment},
	symbol={\ensuremath{M_0}},
	type=symbols
}


\newglossaryentry{sym:mu}
{
	name={\ensuremath{\mu_{stf}}},
	description={stiffness coefficient},
	symbol={\ensuremath{\mu_{stf}}},
	type=symbols
}


\newglossaryentry{sym:A}
{
	name={\ensuremath{A}},
	description={felt area},
	symbol={\ensuremath{A}},
	type=symbols
}


\newglossaryentry{sym:D}
{
	name={\ensuremath{\tilde{D}}},
	description={mean displacement},
	symbol={\ensuremath{\tilde{D}}},
	type=symbols
}


\newglossaryentry{sym:MW}
{
	name={\ensuremath{M_W}},
	description={moment magnitude},
	symbol={\ensuremath{M_W}},
	type=symbols
}

\newglossaryentry{sym:A_richter}
{
	name={\ensuremath{\hat{A}}},
	description={amplitude from an Wood-Anderson seismometer},
	symbol={\ensuremath{\hat{A}}},
	type=symbols
}

\newglossaryentry{sym:d_richter}
{
	name={\ensuremath{\hat{d}}},
	description={distance far 100km from earthquake},
	symbol={\ensuremath{\hat{d}}},
	type=symbols
}


\newglossaryentry{sym:b}
{
	name={\ensuremath{b}},
	description={b-value}, 
	symbol={\ensuremath{b}},
	type=symbols
}


\newglossaryentry{sym:a}
{
	name={\ensuremath{a}},
	description={a-value},
	symbol={\ensuremath{a}},
	type=symbols
}


\newglossaryentry{sym:N_m}
{
	name={\ensuremath{N(m,m+\mathrm{d}m)}},
	description={number of earthquakes with magnitude values between $m$ and $m + \mathrm{d}m$ },
	symbol={\ensuremath{N(m)}},
	type=symbols
}


\newglossaryentry{sym:m_min}
{
	name={\ensuremath{m_{min}}},
	description={minimum magnitude},
	symbol={\ensuremath{m_{min}}},
	type=symbols
}

\newglossaryentry{sym:m_max}
{
	name={\ensuremath{m_{max}}},
	description={maximum magnitude},
	symbol={\ensuremath{m_{max}}},
	type=symbols
}

% \newglossaryentry{sym:m_c}
% {
% 	name={\ensuremath{m_c}},
% 	description={completeness magnitude},
% 	symbol={\ensuremath{m_c}},
% 	type=symbols
% }



\newglossaryentry{sym:m_corner}
{
	name={\ensuremath{m_{corner}}},
	description={magnitude value which controls the Kagan-MFD behavior},
	symbol={\ensuremath{m_{corner}}},
	type=symbols
}

\newglossaryentry{sym:M}
{
	name={\ensuremath{M}},
	description={magnitude random variable},
	symbol={\ensuremath{M}},
	type=symbols
}

\newglossaryentry{sym:beta}
{
	name={\ensuremath{\beta}},
	description={\beta = \gls{sym:b}\ln{10}},
	symbol={\ensuremath{\beta}},
	type=symbols
}

\newglossaryentry{sym:beta_p}
{
	name={\ensuremath{\beta_p}},
	description={$\beta_p = \frac{2}{3}\gls{sym:b}$ is the Pareto's beta},
	symbol={\ensuremath{\beta_p}},
	type=symbols
}

\newglossaryentry{sym:alpha}
{
	name={\ensuremath{\alpha}},
	description={total number of earthquakes},
	symbol={\ensuremath{\alpha}},
	type=symbols
}


\newglossaryentry{sym:ri}
{
	name={\ensuremath{\boldsymbol{r}_i}},
	description={spatial location of earthquake $i$},
	symbol={\ensuremath{\boldsymbol{r}_i}},
	type=symbols
}


\newglossaryentry{sym:ti}
{
	name={\ensuremath{t_i}},
	description={time location of earthquake $i$},
	symbol={\ensuremath{t_i}},
	type=symbols
}


\newglossaryentry{sym:hi}
{
	name={\ensuremath{h_i}},
	description={temporal bandwidth for earthquake $i$},
	symbol={\ensuremath{h_i}},
	type=symbols
}


\newglossaryentry{sym:di}
{
	name={\ensuremath{d_i}},
	description={spatial bandwidth for earthquake $i$},
	symbol={\ensuremath{d_i}},
	type=symbols
}


\newglossaryentry{sym:wi}
{
	name={\ensuremath{ w }},
	description={weight},
	symbol={\ensuremath{ w }},
	type=symbols
}


\newglossaryentry{sym:Mc_rt}
{
	name={\ensuremath{ M_c\left( \gls{sym:r}, \gls{sym:t} \right)  }},
	description={completenes magnitude on location \gls{sym:r} and \gls{sym:t}},
	symbol={\ensuremath{ M_c\left( \gls{sym:r}, \gls{sym:t} \right) }},
	type=symbols
}


\newglossaryentry{sym:Mc}
{
	name={\ensuremath{M_c}},
	description={completeness magnitude},
	symbol={\ensuremath{M_c}},
	type=symbols
}

\newglossaryentry{sym:Md}
{
	name={\ensuremath{M_d}},
	description={minimum magnitude value on the catalogue},
	symbol={\ensuremath{M_d}},
	type=symbols
}


\newglossaryentry{sym:Rmin}
{
	name={\ensuremath{R_{min}}},
	description={minimum seismic rate},
	symbol={\ensuremath{R_{min}}},
	type=symbols
}


\newglossaryentry{sym:R}
{
	name={\ensuremath{R(\gls{sym:r},\gls{sym:t})}},
	description={seismic rate located \gls{sym:r} distant from the earthquake occurred on the instant \gls{sym:t}},
	symbol={\ensuremath{R(\gls{sym:r},\gls{sym:t})}},
	type=symbols
}


\newglossaryentry{sym:Rrm}
{
	name={\ensuremath{R(\gls{sym:r},\gls{sym:m})}},
	description={seismic rate located \gls{sym:r} distant from the earthquake occurred on the instant \gls{sym:t}},
	symbol={\ensuremath{R(\gls{sym:r},\gls{sym:t})}},
	type=symbols
}


\newglossaryentry{sym:Kt}
{
	name={\ensuremath{K_t \left( \frac{ t - \gls{sym:ti} }{ \gls{sym:hi} } \right) }},
	description={time domain kernel function, where
					\gls{sym:ti} is the \glsdesc{sym:ti} and
					\gls{sym:hi} is the \glsdesc{sym:hi}
				},
	symbol={\ensuremath{K_t \left( \frac{ t - \gls{sym:ti} }{ \gls{sym:hi} } \right)}},
	type=symbols
}

\newglossaryentry{sym:Kr}
{
	name={\ensuremath{K_r \left( \frac{ \| \gls{sym:r} - \gls{sym:ri} \| }{d_i} \right) }},
	description={space domain kernel function, where
					\gls{sym:ri} is the \glsdesc{sym:ri} and
					\gls{sym:di} is the \glsdesc{sym:di}
	},
	symbol={\ensuremath{K_r \left( \frac{ \| \gls{sym:r} - \gls{sym:ri} \| }{d_i} \right)}},
	type=symbols
}


\newglossaryentry{sym:Krm}
{
	name={\ensuremath{K(\gls{sym:r},\gls{sym:m})}},
	description={kernel function for some distance \gls{sym:r} from earthquakes with magnitudes  \gls{sym:m}},
	symbol={\ensuremath{K_1 \left( \frac{ t - \gls{sym:ti} }{ \gls{sym:hi} } \right)}},
	type=symbols
}

\newglossaryentry{sym:a_cnn}
{
	name={\ensuremath{a_{cnn}}},
	description={space-time coupling factor},
	symbol={\ensuremath{a_{cnn}}},
	type=symbols
}

\newglossaryentry{sym:k_cnn}
{
	name={\ensuremath{k_{cnn}}},
	description={$k^{th}$ nearest neighbor},
	symbol={\ensuremath{k_{cnn}}},
	type=symbols
}


\newglossaryentry{sym:dk}
{
	name={\ensuremath{d_k}},
	description={$\max{\left\{ d_j \right\}}, j=1,\ldots,k_{cnn}$},
	symbol={\ensuremath{d_k}},
	type=symbols
}

\newglossaryentry{sym:hk}
{
	name={\ensuremath{h_k}},
	description={$\max{\left\{ h_j \right\} }, j=1,\ldots,k_{cnn}$},
	symbol={\ensuremath{h_k}},
	type=symbols
}

\newglossaryentry{sym:ixiy}
{
	name={\ensuremath{\left(i_x, i_y\right)}},
	description={each grid cell},
	symbol={\ensuremath{\left(i_x, i_y\right)}},
	type=symbols
}


\newglossaryentry{sym:N}
{
	name={\ensuremath{N}},
	description={number of earthquakes on the catalog},
	symbol={\ensuremath{N}},
	type=symbols
}



\newglossaryentry{sym:Np}
{
	name={\ensuremath{N_p\left(i_x, i_y\right)}},
	description={predicted seismic rate on cell \gls{sym:ixiy}},
	symbol={\ensuremath{N_p\left(i_x, i_y\right)}},
	type=symbols
}


\newglossaryentry{sym:Nu}
{
	name={\ensuremath{N_u}},
	description={\gls{sym:Nt}/\gls{sym:Nc}},
	symbol={\ensuremath{N_u}},
	type=symbols
}


\newglossaryentry{sym:Nc}
{
	name={\ensuremath{N_c}},
	description={number of grid cells},
	symbol={\ensuremath{N_c}},
	type=symbols
}


\newglossaryentry{sym:Npi}
{
	name={\ensuremath{N_p(i)}},
	description={predicted seismicity rate on spatial bin where earthquake
	$i$ occurred}, symbol={\ensuremath{N_p(i)}},
	type=symbols
}


\newglossaryentry{sym:NAi}
{
	name={\ensuremath{N_A(i)}},
	description={seismic rate on $i$ predicted by the $A$ model}, 
	symbol={\ensuremath{N_A(i)}},
	type=symbols
}


\newglossaryentry{sym:NBi}
{
	name={\ensuremath{N_B(i)}},
	description={seismic rate on $i$ predicted by the $B$ model},
	symbol={\ensuremath{N_B(i)}},
	type=symbols
}


\newglossaryentry{sym:Ts}
{
	name={\ensuremath{T_s}},
	description={Student distribution $T$-value},
	symbol={\ensuremath{T_s}},
	type=symbols
}

\newglossaryentry{sym:nxy}
{
	name={\ensuremath{n\left(i_x, i_y\right)}},
	description={number of target earthquakes observed on cell \gls{sym:ixiy}},
	symbol={\ensuremath{n\left(i_x, i_y\right)}},
	type=symbols
}


\newglossaryentry{sym:Nt}
{
	name={\ensuremath{N_t}},
	description={number of events on the target catalog},
	symbol={\ensuremath{N_t}},
	type=symbols
}



\newglossaryentry{sym:L}
{
	name={\ensuremath{L}},
	description={log-likelihood},
	symbol={\ensuremath{L}},
	type=symbols
}


\newglossaryentry{sym:Lu}
{
	name={\ensuremath{L_u}},
	description={uniform model likelihood},
	symbol={\ensuremath{L}},
	type=symbols
}


\newglossaryentry{sym:pNn}
{
	name={\ensuremath{p(N_p, n)}},
	description={probability of observe exactly $n$ events with probability \gls{sym:np} each one},
	symbol={\ensuremath{p(N_p, n)}},
	type=symbols
}


\newglossaryentry{sym:G}
{
	name={\ensuremath{G}},
	description={probability gain from each earthquake on target-catalog over a spatialy uniform Poisson model.}, 
	symbol={\ensuremath{G}}, 
	type=symbols
}


\newglossaryentry{sym:I}
{
	name={\ensuremath{ I_{inf}(A,B)}},
	description={information gain from the model $A$ over the model $B$}, 
	symbol={\ensuremath{I_{inf}(A,B)}}, 
	type=symbols
}


\newglossaryentry{sym:aW}
{
	name={\ensuremath{a_W}},
	description={fractal dimension factor, generally about 1.5 and 2}, 
	symbol={\ensuremath{a_W}}, 
	type=symbols
}



\newglossaryentry{sym:DW}
{
	name={\ensuremath{D_W}},
	description={fractal dimenstion of spatial earthquake occurrence $D_W = 2-\gls{sym:aW}$}, 
	symbol={\ensuremath{D_W}}, 
	type=symbols
}



\newglossaryentry{sym:hm}
{
	name={\ensuremath{h(m)}},
	description={fixed kernel bandwidth for earthquakes with magnitude $m$}, 
	symbol={\ensuremath{h(m)}}, 
	type=symbols
}



\newglossaryentry{sym:a0}
{
	name={\ensuremath{a_0}},
	description={\gls{sym:hm} linear parameter}, 
	symbol={\ensuremath{a_0}}, 
	type=symbols
}



\newglossaryentry{sym:a1}
{
	name={\ensuremath{a_1}},
	description={\gls{sym:hm} exponential parameter}, 
	symbol={\ensuremath{a_1}}, 
	type=symbols
}




\newglossaryentry{sym:dF}
{
	name={\ensuremath{d_F}},
	description={correlation distance or \emph{fixed bandwidth}}, 
	symbol={\ensuremath{d_F}}, 
	type=symbols
}


\newglossaryentry{sym:dij}
{
	name={\ensuremath{d_{ij}}},
	description={distance between grid cells $i$ and $j$}, 
	symbol={\ensuremath{d_{ij}}}, 
	type=symbols
}


\newglossaryentry{sym:I0}
{
	name={\ensuremath{I_0}},
	description={maximum intensity}, 
	symbol={\ensuremath{I_0}}, 
	type=symbols
}


\newglossaryentry{sym:Af}
{
	name={\ensuremath{A_{f}}},
	description={felt area in $\text{km}^2$}, 
	symbol={\ensuremath{A_{f}}}, 
	type=symbols
}



       

%-----------------------------------------
% acronyms
%-----------------------------------------

\newacronym{psha}{PSHA}{Probabilistic Seismic Hazard Analysis}
\newacronym{dsha}{DSHA}{Deterministic Seismic Hazard Analysis}
\newacronym{GMPE}{\gls{gmpe}}{\glsdesc{gmpe}}
\newacronym{mfd}{MFD}{Magnitude Frequency Distribution}
\newacronym{gr}{GR}{Gutenberg-Richter}
\newacronym{cnn}{CNN}{Coupled-Nearest-Neighbour}
\newacronym{va}{r.v.}{random variable}
\newacronym{iid}{i.i.d.}{independent and identicaly distributed}
\newacronym{pdf}{density}{probability density function}
\newacronym{pmf}{pmf}{probability mass function}
\newacronym{kde}{KDE}{Kernel Density Estimation}

\newacronym{pga}{PGA}{peak ground-motion acceleration}

\newacronym{isc}{ISC}{International Seismological Centre}
\newacronym{gem}{GEM}{Global Earthquake Model}
\newacronym{opensha}{openSHA}{Open Source Seismic Hazard Analysis}
\newacronym{oq}{OQ}{Openquake}
\newacronym{eeri}{EERI}{Earthquake Engineering Research Institute}
\newacronym{cgmw}{CGMW}{Comission for Geological Map of the World}



\newacronym{hl}{HazardLib}{hazard library}
\newacronym{rl}{RiskLib}{risk library}
\newacronym{nrml}{NRML}{Natural-hazard and Risk Markup Language}
\newacronym{hmtk}{HMTK}{Hazard Modeller's Toolkit}
\newacronym{oqp}{oq-platform}{Plataforma web de Interação com o OQ}
\newacronym{oqe}{oq-engine}{OQ hazard and risk calculator}

\newacronym{obsis}{ObSis}{Seismological Observatory}
\newacronym{unb}{UnB}{Brasilia University}
\newacronym{iag}{IAG}{Institute of Astronomy, Geophysics and Atmosferical Sciences}
\newacronym{usp}{USP}{São Paulo University}
\newacronym{ipt}{IPT}{Institute of Technological Research}
\newacronym{unesp}{UNESP}{São Paulo State University}
\newacronym{ufrn}{UFRN}{Federal University of Rio Grande do Norte}

\newacronym{bsb}{BSB}{Brazilian Seismic Bulletin}
\newacronym{csv}{CSV}{Common separated values}

\newacronym{bsb2013}{{BSB-2013.08}}{Brazilian Seismic Bulletin version 2013.08}
\newacronym{bsb2014}{{BSB-2014.11}}{Brazilian Seismic Bulletin version 2011.11}
\newacronym{iscgem}{{\gls*{isc}-\gls*{gem}}}{ISC-GEM catalog for South America}


       
%-----------------------------------------
% equations
%-----------------------------------------

\newglossaryentry{eqn:M_0}
{
	name={\ensuremath{\gls{sym:M_0} = \gls{sym:mu}\gls{sym:A}\gls{sym:D}}},
	description={onde \gls{sym:mu} é \glsdesc{sym:mu}, 
				 \gls{sym:A} é \glsdesc{sym:A} e 
				 \gls{sym:D} é \glsdesc{sym:D}. 
				 Tem unidades de energia [N.m]
		 },
	type=equations
}

\newglossaryentry{eqn:M_W}
{
	name={\ensuremath{\gls{sym:MW} = \frac{2}{3} \log_{10}{\gls{sym:M_0}} - 10.7 }},
	description={onde \gls{sym:M_0} é \glsdesc{sym:M_0} em [N.m]
		 },
	type=equations
}

\newglossaryentry{eqn:richter}
{
	name={\ensuremath{\log{\gls{sym:A_richter}} = 3.37 - 3\log{\gls{sym:d_richter}}}},
	description={onde \gls{sym:A_richter} é \glsdesc{sym:A_richter} e 
					  \gls{sym:d_richter} é \glsdesc{sym:d_richter}
		 },
	type=equations
}

\newglossaryentry{eqn:gr_mfd}
{
	name={\ensuremath{\log{\gls{sym:N_m}} = \gls{sym:a} - \gls{sym:b}\gls{sym:m} }},
	description={onde \gls{sym:N_m} é o \glsdesc{sym:N_m}, 
					  \gls{sym:a} é o \glsdesc{sym:a},
					  \gls{sym:b} é o \glsdesc{sym:b}
    },
	type=equations
}


\newglossaryentry{eqn:Fm_richter}
{
	name={\ensuremath{\log{\gls{sym:N_m}} = \gls{sym:a} - \gls{sym:b}\gls{sym:m} }},
	description={onde \gls{sym:N_m} é \glsdesc{sym:N_m}, 
					  \gls{sym:a} é o \glsdesc{sym:a},
					  \gls{sym:b} é o \glsdesc{sym:b}
    },
	type=equations
}

       
% ---------------------------------------------------------------------------- %


% ---------------------------------------------------------------------------- %
% Cabeçalhos similares ao TAOCP de Donald E. Knuth
\usepackage{fancyhdr}
\pagestyle{fancy}
\fancyhf{}
\renewcommand{\chaptermark}[1]{\markboth{\MakeUppercase{#1}}{}}
\renewcommand{\sectionmark}[1]{\markright{\MakeUppercase{#1}}{}}
\renewcommand{\headrulewidth}{0pt}

% ---------------------------------------------------------------------------- %
\graphicspath{{./images/}}             % caminho das figuras (recomendável)
\frenchspacing                          % arruma o espaço: id est (i.e.) e exempli gratia (e.g.) 
\urlstyle{same}                         % URL com o mesmo estilo do texto e não mono-spaced
\raggedbottom                           % para não permitir espaços extra no texto
\fontsize{60}{62}\usefont{OT1}{cmr}{m}{n}{\selectfont}
\cleardoublepage
\normalsize

% ---------------------------------------------------------------------------- %
% Opções de listing usados para o código fonte
% Ref: http://en.wikibooks.org/wiki/LaTeX/Packages/Listings
\lstset{ %
language=Python,                % choose the language of the code
basicstyle=\footnotesize,       % the size of the fonts that are used for the code
numbers=left,                   % where to put the line-numbers
numberstyle=\footnotesize,      % the size of the fonts that are used for the line-numbers
stepnumber=1,                   % the step between two line-numbers. If it's 1 each line will be numbered
numbersep=5pt,                  % how far the line-numbers are from the code
showspaces=false,               % show spaces adding particular underscores
showstringspaces=false,         % underline spaces within strings
showtabs=false,                 % show tabs within strings adding particular underscores
frame=single,	                % adds a frame around the code
framerule=0.6pt,
tabsize=2,	                    % sets default tabsize to 2 spaces
captionpos=b,                   % sets the caption-position to bottom
breaklines=true,                % sets automatic line breaking
breakatwhitespace=false,        % sets if automatic breaks should only happen at whitespace
escapeinside={\%*}{*)},         % if you want to add a comment within your code
backgroundcolor=\color[rgb]{1.0,1.0,1.0}, % choose the background color.
rulecolor=\color[rgb]{0.8,0.8,0.8},
extendedchars=true,
xleftmargin=10pt,
xrightmargin=10pt,
framexleftmargin=10pt,
framexrightmargin=10pt
}

% ---------------------------------------------------------------------------- %
% Corpo do texto
\begin{document}
\frontmatter 
% cabeçalho para as páginas das seções anteriores ao capítulo 1 (frontmatter)
\fancyhead[RO]{{\footnotesize\rightmark}\hspace{2em}\thepage}
\setcounter{tocdepth}{2}
\fancyhead[LE]{\thepage\hspace{2em}\footnotesize{\leftmark}}
\fancyhead[RE,LO]{}
\fancyhead[RO]{{\footnotesize\rightmark}\hspace{2em}\thepage}

\onehalfspacing  % espaçamento

% ---------------------------------------------------------------------------- %
% CAPA
% Nota: O título para as dissertações/teses do IME-USP devem caber em um 
% orifício de 10,7cm de largura x 6,0cm de altura que há na capa fornecida pela SPG.
\thispagestyle{empty}
\begin{center}
    \vspace*{2.3cm}
    \textbf{\Large{\glsdesc*{titulo_da_dissertacao}}}\\
    
    \vspace*{1.2cm}
    \Large{Marlon Pirchiner}
    
    \vskip 2cm
    \textsc{
    Dissertação apresentada à\\[-0.25cm] 
    Escola de Matemática Aplicada da\\[-0.25cm]
    Fundação Getúlio Vargas-RJ\\[-0.25cm]
    para obtenção do título de \\[-0.25cm]
    Mestre em Ciências}
    
    \vskip 1.5cm
    Programa: Modelagem Matemática de Informação\\
    Orientador: Prof. Dr. Vincent Guigues\\
    %Coorientador: Prof. Dr. Stephane Drouet

   	\vskip 1cm
    \normalsize{}
    
    \vskip 0.5cm
    \normalsize{Rio de Janeiro, agosto de 2014}
\end{center}

% ---------------------------------------------------------------------------- %
% Página de rosto (SÓ PARA A VERSÃO DEPOSITADA - ANTES DA DEFESA)
% Resolução CoPGr 5890 (20/12/2010)
%
% IMPORTANTE:
%   Coloque um '%' em todas as linhas
%   desta página antes de compilar a versão
%   final, corrigida, do trabalho
%
%\newpage
%\thispagestyle{empty}
%    \begin{center}
%        \vspace*{2.3 cm}
%	    \textbf{\Large{\glsdesc*{titulo_da_dissertacao}}}\\
%        \vspace*{2 cm}
%    \end{center}

%    \vskip 2cm

%    \begin{flushright}
%	Esta é a versão original da dissertação elaborada pelo\\
%	candidato Marlon Pirchiner, tal como \\
%	submetida à Comissão Julgadora.
%   \end{flushright}

%\pagebreak


% ---------------------------------------------------------------------------- %
% Página de rosto (SÓ PARA A VERSÃO CORRIGIDA - APÓS DEFESA)
% Resolução CoPGr 5890 (20/12/2010)
%
% Nota: O título para as dissertações/teses do IME-USP devem caber em um 
% orifício de 10,7cm de largura x 6,0cm de altura que há na capa fornecida pela SPG.
%
% IMPORTANTE:
%   Coloque um '%' em todas as linhas desta
%   página antes de compilar a versão do trabalho que será entregue
%   à Comissão Julgadora antes da defesa
%
%
\newpage
\thispagestyle{empty}
     \begin{center}
         \vspace*{2.3 cm}
         \textbf{\Large{\glsdesc*{titulo_da_dissertacao}}}\\
         \vspace*{2 cm}
     \end{center}
% 
     \vskip 2cm
% 
     \begin{flushright}
 	Esta versão da dissertação contém as correções e alterações sugeridas\\
 	pela Comissão Julgadora durante a defesa da versão original do trabalho,\\
 	realizada em 15/07/2014. 
     \vskip 2cm
% 
     \end{flushright}
     \vskip 4.2cm
% 
     \begin{quote}
     \noindent Comissão Julgadora:
%     
     \begin{itemize}
 		\item Vincent Guigues (orientador) - EMAp/FGV-RJ 
 		\item Flávio Codeço Coelho - EMAp/FGV-RJ 
 		\item Marcelo Assumpção - IAG-USP 
 		\item Moacyr Alvim Horta Barbosa da Silva - EMAp/FGV-RJ 
 		\item Stéphane Drouet - ON 
     \end{itemize}
%       
     \end{quote}
 \pagebreak

% CATALOGRAPHIC
\newpage
\thispagestyle{empty}

	\begin{figure}[H]
	  \centering
	  \includegraphics[width=\textwidth]{_catalographic} 
	\end{figure}

\pagebreak


% SIGNATURES
\newpage
\thispagestyle{empty}

	\begin{figure}[H]
	  \centering
	  \includegraphics[width=\textwidth]{_signatures} 
	\end{figure}

\pagebreak



\pagenumbering{roman}     % começamos a numerar 

% ---------------------------------------------------------------------------- %
% Agradecimentos:
% Se o candidato não quer fazer agradecimentos, deve simplesmente eliminar esta página 
\chapter*{Agradecimentos}
Aos meus familiares e amigos pela benevolência de sempre.

À Didi.

A todos do Grupo de Sismologia (e também a todo pessoal) do Instituto de
Astronomia Geofísica e Ciências Atmosféricas (IAG) da Universidade de São Paulo (USP) 
por todo apoio e suporte de sempre e durante o tempo em que
estive entre o curso de mestrado e o trabalho.

Aos companheiros e professores pelas conversas e discussões ao longo do curso.

% ---------------------------------------------------------------------------- %
% Resumo
\chapter*{Resumo}

A avaliação de risco sísmico, fundamental para as decisões sobre as estruturas de obras de engenharia
e mitigação de perdas, envolve fundamentalmente a análise de ameaça sísmica. 
Calcular a ameaça sísmica é o mesmo que calcular a probabilidade de que certo nível de 
determinada medida de intensidade em certo local durante um certo tempo seja excedido.

Dependendo da complexidade da atividade geológica essas estimativas podem ser bastante sofisticadas.
Em locais com baixa sismicidade, como é o caso do Brasil, o pouco tempo (geológico) de observação
e a pouca quantidade de informação são fontes de muitas incertezas e dificuldade de análise pelos métodos mais clássicos
e conhecidos que geralmente consideram, através de opiniões de especialistas, determinadas zonas sísmicas.

Serão discutidas algumas técnicas de suavização e seus fundamentos como métodos alternativos ao zoneamento,
em seguida se exemplifica suas aplicações no caso brasileiro.

\noindent \textbf{Palavras-chave:} terremotos, ameaça sísmica, suavização.

% ---------------------------------------------------------------------------- %
% Abstract
\chapter*{Abstract}
\noindent PIRCHINER, M. \textbf{Smoothing seismicity techniques applied to probabilistic seismic
hazard analysis in Brazil}.
2014.
Master thesis - Applied Math School,
Getúlio Vargas Foundation, Rio de Janeiro, 2014.
\\

Seismic risk assesment is crucial to make better decisions about engineering structures and loss mitigation.
It involves, mainly, the evaluation of seismic hazard.
Seismic hazard assesment is the computation of probability which the level of some ground
motion intensity measure, in a given site, which, in some time window, will be exceeded.

Depending on geological and tectonic complexity, the seismic hazard evaluation becomes more sofisticated.
At sites with low seismicity, which is the brazilian case, the relative (geologically) low observation time
and the lack of earthquake and tectonic information, increases the uncertainties and makes more difficult
the standard analysis, which in general, consider expert opinions, to characterize the
seismicity into disjoint zones. 

This text discusses some smoothing seismicity techniques and their theoretical foundations as alternative
methods for seismicity characterization. Next, the methods are exemplified in the brazilian context.

\noindent \textbf{Keywords:} earthquakes, seismic hazard, smoothing.

% ---------------------------------------------------------------------------- %
% Sumário
\tableofcontents    % imprime o sumário

% Lista de Abreviaturas
\printglossary[type=\acronymtype,title=Lista de Abreviaturas]

% Lista de Simbolos
\printglossary[type=symbols,title=Lista de S\'imbolos]

%\printglossaries
%\printglossaries


% ---------------------------------------------------------------------------- %
%\chapter{Lista de Abreviaturas}
%\begin{tabular}{ll}
%         CFT         & Transformada contínua de Fourier (\emph{Continuous Fourier Transform})\\
%         DFT         & Transformada discreta de Fourier (\emph{Discrete Fourier Transform})\\
%        EIIP         & Potencial de interação elétron-íon (\emph{Electron-Ion Interaction Potentials})\\
%        STFT         & Tranformada de Fourier de tempo reduzido (\emph{Short-Time Fourier Transform})\\
%\end{tabular}

% ---------------------------------------------------------------------------- %
%\chapter{Lista de Símbolos}
%\begin{tabular}{ll}
%        $\lambda$   & Seismic Rate\\
%        $\psi$      & Função de análise \emph{wavelet}\\
%        $\Psi$      & Transformada de Fourier de $\psi$\\
%\end{tabular}




% ---------------------------------------------------------------------------- %
% Listas de figuras e tabelas criadas automaticamente
\listoffigures            
\listoftables            


% ---------------------------------------------------------------------------- %
% Capítulos do trabalho
\mainmatter

% cabeçalho para as páginas de todos os capítulos
\fancyhead[RE,LO]{\thesection}

\singlespacing              % espaçamento simples
%\onehalfspacing            % espaçamento um e meio


% ============================================================================ %

%% ------------------------------------------------------------------------- %%
\chapter{Introdução}
\label{cap:introducao}

Um elemento primordial na análise
de \emph{risco} sísmico é a análise da \emph{ameaça} sísmica,
onde a identificação e caracterização das fontes sismogênicas (causadoras de
movimento do chão, fundamentalmente tremores de terra) é a primeira das etapas. 

Considera-se nessa fase, principalmente as falhas
geológicas, o acúmulo de tensão medido através do movimento relativo da crosta
terrestre, a neotecnônica da crosta, o possível acoplamento entre placas, os tremores
(rupturas e falhamentos) já registrados anteriormente, enfim, todo conhecimento geológico
disponível, para caracterizar (a) a geometria espacial da feição geológica e provável fonte
sísmica e (b) o número de ocorrência - taxa - dos tremores conforme a
proporção de energia liberada por cada um - magnitude.

No Brasil, onde a ocorrência de tremores não é desprezível mas menor do que a de
outras partes do planeta, o processo de identificação das fontes sísmicas é
executado geralmente através da opinão de especialistas que fazem o zoneamento
sísmico segundo as informações, técnicas e a experiência que dispõem.

Para cada uma dessas zonas sísmicas, que serão consideradas como tendo atividade
sísmica uniforme, é determinada a distribuição da ocorrência de tremores em função
da magnitude de cada tremor.

Existem também outras propostas metodológicas envolvendo técnicas de suavização
que permitem estimativas da taxa de sismicidade, por exemplo por funções de núcleo.
As propostas de \citet{frankel_1995}, a de \citet{woo_1996} e a de
\citet{helmstetter_2012} serão discutidas com maior detalhe.

O que todas elas possuem em comum é o objetivo de caracterizar a taxa de
sismicidade (ocorrência de tremores) em uma malha sobre a região de interesse
através da soma da contribuição de funções de núcleo (gaussianas, leis de
potência, etc) em cada nó dessa malha.

O pressuposto central dessa
idéia é que os grandes sismos (com menos evidências, pois
aconteceram poucos fenômenos observáveis desse tipo) 
tendem a ocorrer no entorno de
onde já ocorreram antes outros tremores (menores e mais frequentes).

Fundamentalmente o que os diferencia é a forma de escolher a largura para essas
funções de núcleo associadas a cada tremor do catálogo.

O que se pretende é observar um pouco mais detalhadamente o comportamento
desses diferentes métodos num ambiente com baixa e esparsa sismicidade.

Perifericamente, aproveitou-se a oportunidade para avaliar um recente
conjunto de programas de computador, com código livre, implementado para esse tipo de análise.
 

%% ------------------------------------------------------------------------- %%
\section{Objetivos}
\label{sec:objetivo}

O principal objetivo pretendido é a avaliação da
aplicabilidade de algumas técnicas de suavização para a
caracterização da ocorrência de sismos no Brasil.

Secundariamente, aproveita-se a oportunidade para testar o uso de um conjunto
recente de programas de computador disponível livremente pela e para a comunidade
científica, o Openquake\footnote{\url{http://www.globalquakemodel.org/openquake}}. 

Perifericamente, será possível fazer alguma contribuição ao Openquake
ou a alguma outra biblioteca associada destinada a modelagem dos parâmetros de entrada
para o cálculo de ameaça.

%% ------------------------------------------------------------------------- %%
\section{Contribuições}
\label{sec:contribucoes}

As principais contribuições deste trabalho são:

\begin{itemize}
  \item Discorrer sobre métodos alternativos ao zoneamento para a caracterização de fontes
  sismogênicas, a primeira das etapas da análise probabilística de ameaça 
  sísmica.

  \item Compreender e assimilar como usar Openquake, um conjunto de programas de
  computador desenvolvido recentemente e oferecido com código livre, contribuindo
  para oferecer uma ferramenta a mais para o cálculo da ameaça sísmica à comunidade
  de sismologia e engenharia sísmica brasileira.
  
\end{itemize}

%% ------------------------------------------------------------------------- %%
\section{Organização do Trabalho}
\label{sec:organizacao_trabalho}

No capítulo \ref{cap:conceitos}, são apresentados os conceitos mais elementares
de sismologia e de estatística relevantes para uma melhor compreensão do tema.
Em seguida, no capítulo \ref{cap:regiao_de_estudo} é apresentado com maior detalhe
a região de estudo sob o ponto de vista geológico e tectônico. No capítulo \ref{cap:teoria}
é apresentada e formalizada a teoria e os fundamentos dos métodos discutidos. O capítulo \ref{cap:processamento}
discorre sobre as etapas de processamento propriamente ditas, enquanto o capítulo \ref{cap:resultados}
foca apenas nos resultados de cada método e em resultados anteriores. As discussões relevantes a partir da análise
dos resultados é apresentada no capítulo \ref{cap:conclusoes} finalmente.
        % associado ao arquivo: 'cap-introducao.tex'
%% ------------------------------------------------------------------------- %%
\chapter{Conceitos}
\label{cap:conceitos}

Este capítulo apresenta, um a um, os conceitos mais elementares
e tenta harmonizar a terminologia empregada no decorrer do texto.


%% ------------------------------------------------------------------------- %%
\section{\Gls{tectonic}}
\index{\gls{tectonic}}
\label{sec:02_tectonica}

A \gls{tectonic} é \glsdesc*{tectonic}.

Uma das principais evidências das transformações geológicas do planeta
são os \glspl{equake}. A figura \ref{f:global_epicenters} \citep{lowman_jr_1998}
é um mapa global com a ocorrência geográfica dos tremores. Nele é possivel notar que
os sismos não são distribuídos uniformemente pelo globo.

\begin{figure}[H]
   \centering
   \includegraphics[width=0.80\textwidth]{global_pde_mag_all}
   \caption{Mapa Mundial de Epicentros}
   \label{f:global_epicenters}
\end{figure}

O padrão apresentado pela \gls{seismic_activity} global foi essencial
para o desenvolvimento posterior da \gls*{tectonic_plate_theory}.

%% ------------------------------------------------------------------------- %%
\subsection{\Gls{tectonic_plate_theory}}
\index{\Gls{tectonic}!\Gls{tectonic_plate_theory}}
\label{sec:02_placas}

A \gls*{tectonic_plate_theory}, desenvolvida na segunda metade do século XX,
cartografava na superfície do globo as \glspl{litho_plate}.


\begin{figure}[H]
   \centering
   \includegraphics[width=0.80\textwidth]{litho_plates_overview}
   \caption[Cartografia das placas litosféricas]
   		   {Cartografia das placas litosféricas}
   \label{f:plates_overview}
\end{figure}

As \glspl{litho_plate}, como pode ser visto na figura \ref{f:plates_overview} \citep{usgs_plates_1996},
e o conceito de \gls{astenosphere} (\glsdesc{astenosphere})
surgem para conformar uma teoria capaz de explicar
uma série de fenômenos tectônicos observados e ainda não bem explicados naquela época.


%% ------------------------------------------------------------------------- %%
\subsubsection{Bordas}
\index{\Gls{tectonic_plate_theory}!bordas}
\label{sec:02_bordas}

Nas bordas das \glspl{litho_plate} a tectônica é mais intensa
provocando uma enorme diversidade de fenômenos geológicos de acordo
com o tipo de interação. Algumas delas estão esquematizadas na
figura \ref{f:plate_boundaries} \citet{vigil_1997}.

\begin{figure}[H]
   \centering
   \includegraphics[width=0.95\textwidth]{plate_boundaries}
   \caption[Diferentes tipos de interações entre \glspl{litho_plate} em suas bordas]
   		   {Diferentes tipos de interações entre \glspl{litho_plate} em suas bordas}
   \label{f:plate_boundaries}
\end{figure}

Na figura \ref{f:plate_boundaries} estão ilustrados os diferentes tipos de interação
entre as \glspl{litho_plate} nas suas bordas, que causam, como já se sabe, a maior
parte dos \glspl{equake} e vulcanismo.

Só na borda das placas é liberada cerca de 95\% da quantidade total da energia
disseminada na forma de \glspl{equake} no globo.

%% ------------------------------------------------------------------------- %%
\subsubsection{Interior}
\index{\Gls{tectonic_plate_theory}!interior}
\label{sec:02_interior}

A dificuldade é explicar, com maior detalhe, porque e como são liberados os outros
5\% do total de energia em \glspl{equake}, mais raros, no interior das \glspl{litho_plate}.

Não há pleno consenso nem um modelo geral para a explicação do mecanismo de ocorrência dos
sismos no interior das placas \citep{talwani_2014} embora sejam conhecidas diversas zonas sísmicas em regiões no
interior de placas que apresentam sismicidade importante com sismos cujas
magnitudes em alguns casos foram superior a 7, como em Nova Madrid, nos Estados Unidos
e em locais da China e da Austrália para citar alguns outros.


%% ------------------------------------------------------------------------- %%
\subsection{Sismotectônica}
\index{\gls*{seismotectonic}}
\label{sec:sismotectonica}

A \gls{seismotectonic} é \glsdesc*{seismotectonic}.

Na prática consiste por um lado, num esforço de compreensão dos
processos geológicos através da observação dos tremores e analogamente, compreender os tremores através da observação
de processos geológicos mensuráveis.

É fácil notar a contribuição dessa disciplina para a análise de sismicidade e vice-versa.







\section{Probabilidade}
\index{probabilidade}
\label{sec:probabilidade}



\subsection{\Glsdesc{pdf}}
\index{\glsdesc{pdf}}
\label{sec:pdf}

A \gls{pdf} de uma \gls{va}
descreve a probabilidade de que essa \gls{va}
assuma, entre todas as realizações possíveis, uma em especial.

Seja $X$, uma \gls{va} unidimensional. A \gls{pdf}
$f_X(x)$ de $X$ é definida por

\begin{equation}
	P \left\{ X \in [x_0,x_1[\, \right\} = \int_{x_0}^{x_1}\!f_X(x)\,\mathrm{d}x, \; \forall x_0 \leq x_1 \in \mathbb{R}.
	\label{eq:pdf}
\end{equation}

Para que uma função possa assumir o papel de \gls{pdf} é necessário que ela
possua as seguites propriedades:
\begin{enumerate}[(i)]
	\item $f_X(x) \ge 0\;\forall x$  (a função $f_X$ deve ser sempre positiva), e
	\item $\int_{-\infty}^{+\infty} f_X(x) \mathrm{d}x = 1$ (e deve somar, sobre todos os valores possíveis, a unidade).
\end{enumerate}


\subsection{\Glsdesc{pmf}}
\index{\glsdesc{pmf}}
\label{sec:pmf}

Outro conceito importante e diretamente relacionado à \gls{pdf} é a
\gls{pmf}.

No caso da \gls{va} $X$, sua \gls{pmf} $F_X(x)$ é definida como

\begin{equation}
	P \left\{ X \leq x\right\} = F_X(x) = \int_{-\infty}^{x}\!f_X(u)\,\mathrm{d}u.
	\label{eq:pmf}
\end{equation}



\subsection{Histograma}
\index{histograma}
\label{sec:histogram}

Quando a \gls{pdf} de uma \gls{va} não é conhecida e se deseja estudar seu comportamento
é preciso estimá-la e para isso o histograma é uma das técnicas mais antigas e amplamente utilizadas.

O histograma divide o universo das observações, possíveis realizações $X_1, X_2,\cdots, X_n$
da \gls{va}, em compartimentos (\emph{bins}).

Dados uma origem arbitrária $x_0$ e uma largura $h$ de cada um, os compartimentos
são definidos como os intervalos $[x_0 + (j -1)h,\; x_0 + jh[$
com $j\in\mathbb{Z}$, um identificador para cada um deles.

Considere um determinado intervalo $[-h/2, h/2[$.
A probabilidade (\eqref{eq:pdf}) de que uma observação qualquer venha a pertencer a esse intervalo é

\begin{equation}
	P \left\{ X \in [-h/2,h/2[ \; \right\} = \int_{-h/2}^{h/2}\!f_X(x)\,\mathrm{d}x.
	\label{eq:hist01}
\end{equation}

E um estimador natural $\hat{f}_X$ para a densidade $f_X$ seria contar o número de observações

\begin{equation}
	P \left\{ X \in [-h/2,h/2[ \; \right\} \approx \frac{\# \left\{ X_i \in [-h/2,h/2[ \; \right\}}{n} =
	\int_{-h/2}^{h/2}\!\hat{f}_X(x)\,\mathrm{d}x ,
	\label{eq:hist02}
\end{equation}
de onde
\begin{equation}
	\hat{f}_X(x) = \frac{\# \left\{ X_i \in [-h/2,h/2[ \; \right\}}{nh},
	\label{eq:hist_03}
\end{equation}
para todo $x \in [-h/2,h/2[$.

De modo geral, sejam $X_1, \cdots, X_n$ observações \gls{iid} da \gls{va} $X$ com densidade desconhecida $f$.
Considere $N_I$ intervalos de comprimento $h$ e o conjunto de compartimentos $C_j = [x_0 + (j -1)h,\; x_0 + jh[,\;
j=1..N_I$.
Defina
\[
	I_A(x) := \begin{cases}
		1 & \text{se } x \in A \\
		0 & \text{caso contrário}
	\end{cases}
\]
e
\[	n_j := \sum_{i=1}^{n}I_{C_j}(X_i)\; \text{tal que} \;  \sum_{j=1}^{N_I}n_j = n. \]

Dessa forma a estimativa $\hat{f}$ parametrizada pela largura $h$ para a densidade $f$ seria
\begin{equation}
	\hat{f}(x \arrowvert\, h) = \frac{1}{nh} \sum_{j=1}^{N_I}n_jI_{C_j}(x)
	\label{eq:hist_04}
\end{equation}
para toda realização possível $x$ de $X$.






\section{Sismicidade}
\index{sismicidade}
\label{sec:sismicidade}

A sismicidade é a ocorrência dos tremores de terra. Como, quando, onde, de que tamanho?

É sabido que pequenos abalos são mais frequentes que os tremores de terra
muito grandes e catastróficos cujos registros são extremamente raros.

A figura \ref{f:m9} apresenta os sismos de magnitude acima de nove conhecidos.

\begin{figure}[H]
   \centering
   \includegraphics[width=0.5\textwidth]{M9}
   \caption[Sismos com magnitude acima de 9,0.]
   		   {Sismos com magnitude acima de 9,0. Fonte: \gls{isc}}
   \label{f:m9}
\end{figure}


Tremores de terra, abalos, \glspl{equake}, sismos são a ocorrência de
fenômenos geológicos de ruptura, instantânea, por certo mecanismo, de certa dimensão, na
crosta terrestre.

\subsection{Ocorrência}
\index{\gls{equake}!ocorrência}
\label{sec:ocorrencia}

Os tremores acontecem por uma ruptura geológica (figura \ref{f:rupture})
num instante \gls{sym:t}, num lugar \gls{sym:r} e cada um
com sua magnitude \gls{sym:m} associada.

\begin{figure}[H]
   \centering
   \includegraphics[width=0.50\textwidth]{rupture}
   \caption[Ilustração da área de ruptura em um falhamento geológico]
   		   {Ilustração da área de ruptura em um falhamento geológico\footnotemark}
   \label{f:rupture}
\end{figure}
\footnotetext{\citet{opensha_team_2010}}


O local em que se iniciou a ruptura que deu origem ao tremor é um \gls{hypocenter},
enquanto sua projeção na superfície, desconsiderando-se a profundidade, é o \gls{epicenter}.


\subsection[Magnitude]{Magnitude (da ruptura)}
\index{magnitude}
\label{sec:magnitude}

A magnitude de um tremor de terra é um valor medido numa escala que versa sobre a energia liberada pelo sismo.
Essa energia é proporcional à área rompida e ao deslocamento geológico relativo entre os blocos de rocha na superficie
de ruptura.

O desenvolvimento experimental de escalas de magnitude, para medir o tamanho dos tremores,
é marcado pelo trabalho do sismólogo Charles \citet{richter_1935}. Existem, entretanto,
uma série de diferentes escalas de magnitude, baseadas em diversos tipos de medidas.

A escolha de qual usar fica a critério
de cada sismólogo (ou analista) e de cada rede sismográfica.
Geralmente usam escalas diferentes
para avaliar a magnitude dos tremores ou até mesmo divulgam mais de um tipo de magnitude para
um mesmo evento.

As escalas são calibradas para fornecerem valores similares, de acordo com
o intervalo de utilidade para o qual foram desenvolvidas, mas apresentam diferenças consideráveis para um mesmo evento.
Isso pode comprometer as análises estatísticas baseadas em um catálogo
cujas magnitudes não tenham sido calculadas de maneira uniforme.


\subsubsection{Magnitude Richter}
\index{magnitude!Richter}
\label{sec:magnitude_richter}

As escalas de magnitude mais comuns são as que derivam da definição de \citet{richter_1935}
fundada na relação empírica entre o logarítmo da amplitude do registro das ondas sísmicas e a distância onde foram
registradas. Em 1935 Richter notou a seguinte proporcionalidade

\begin{equation}
	\gls{eqn:richter},
	\label{eq:richter}
\end{equation}
\glsdesc*{eqn:richter}.


A amplitude máxima de sua
escala foi definida pela amplitude máxima observada em um sismômetro Wood-Anderson, com período de 0.8s, registrando a 100km
do tremor.

Algumas correções poderiam ser cogitadas, principalmente pelo fato da escala estar intimamente
relacionada a um determinado equipamento, hoje obsoleto, e porque sismos locais (a menos de 100km) têm sua magnitude
melhor calculada usando frequências mais altas que as registráveis pelo sismômetro da época.


Outras escalas foram desenvolvidas a partir da medida da amplitude máxima de determinadas fases
de diferentes tipos de onda sísmica e apresentam bons resultados para a maior parte dos sismos,
mas não refletem, com precisão,
o tamanho dos maiores e mais destrutivos eventos, com magnitude acima de 7 ou 8 porque geralmente saturam.


\subsubsection{Magnitude de Momento Sísmico}
\index{magnitude!momento sísmico}
\label{sec:risco_sismico}

O evento de natureza sismológica ocorre num
instante \gls{sym:t} liberando uma certa quantidade de energia na forma de \glsdesc{sym:M_0}
\gls{sym:M_0}. A \glsdesc{sym:MW} \gls{sym:MW} desse evento é proporcional ao logarítmo dessa energia gls{sym:M_0}.

O \glsdesc{sym:M_0} é apresentado na equação \eqref{eq:M_0}:

\begin{equation}
	\gls{eqn:M_0}
	\label{eq:M_0}
\end{equation}
\glsdesc*{eqn:M_0}.

O momento sísmico é estimado geralmente pela inversão duplamente acoplada de um tensor de momento aos registros em
forma de onda do movimento do chão causado pelo terremoto. Ou, em casos de tremores muito bem registrados, ele pode
ser estimado a partir de algum modelo numérico para a ruptura.

A \glsdesc{sym:MW} \gls{sym:MW} \citep{hanks_1979} é baseada no
logarítmo do \glsdesc{sym:M_0} \gls{sym:M_0}, e não se satura no caso de grandes eventos.
Sua definição é dada pela equação \eqref{eq:M_W}

\begin{equation}
	\gls{eqn:M_W}
	\label{eq:M_W}
\end{equation}
\glsdesc*{eqn:M_W}.


\subsubsection{Intensidade Macrossísmica}
\index{instensidade macrossísmica}
\label{sec:intensidade}

A intensidade macrossísmica é uma escala para medir, não a energia proporcional
à ruptura que originou o tremor de terra, mas para retratar a percepção humana, e os efeitos sobre
construções e objetos, do movimento do chão onde quer este tenha produzido seus efeitos.

Uma das mais difundidas é a escala Modificada de Mercalli \citep{richter_1958} apresentada em sua versão simplificada
na tabela \ref{tab:mercalli}:

\begin{table}[H]
\begin{center}
\begin{scriptsize}
\noindent\begin{tabular}{c|c|p{12cm}}
\hline
Categoria  	& Sensação & Efeitos \\
\hline
I 			&	Imperceptível 	&	Não sentido. Apenas registado pelos sismógrafos.
\\II 		&	Muito fraco 	&	Sentido por um muito reduzido número de pessoas em
								repouso, em especial pelas que habitam em andares
								elevados.
\\III 		&	Fraco 			&	Sentido por um pequeno número de pessoas. Bem sentido nos andares elevados.
\\IV 		&	Moderado 		&	Sentido dentro das habitações, podendo despertar do sono um pequeno número de pessoas.
								Nota-se a vibração de
								portas e janelas e das loiças dentro dos armários.
\\V 		&	Forte 			&	Praticamente sentido por toda a população, fazendo acordar muita gente.
								Há queda de alguns objectos menos estáveis e param os pêndulos dos relógios.
								Abrem-se pequenas fendas nos estuques das paredes.
\\VI 		&	Bastante forte 	&	Provoca início de pânico nas populações. Produzem-se leves danos nas habitações,
								caindo algumas chaminés. O mobiliário menos pesado é deslocado.
\\VII 		&	Muito forte 	&	Caem muitas chaminés. Há estragos limitados em edifícios de boa construção,
								mas importantes e generalizados nas construções mais frágeis.
								Facilmente perceptível pelos condutores de veículos automóveis em trânsito.
								Desencadeia pânico geral nas populações.
\\VIII 		&	Ruinoso  		&	Danos acentuados em construções sólidas. Os edifícios de muito boa construção
								sofrem alguns danos. Caem campanários e chaminés de fábricas.
\\IX 		&	Desastroso 		&	Desmoronamento de alguns edifícios. Há danos consideráveis em construções muito sólidas.
\\X 		&	Destruidor 		&	Abrem-se fendas no solo. Há cortes nas canalizações, torção nas vias de caminho
								de ferro e empolamentos e fissuração nas estradas.
\\XI 		&	Catastrófico 	&	Destruição da quase totalidade dos edifícios, mesmo os mais sólidos.
								Caem pontes, diques e barragens. Destruição das redes de canalização e das vias de comunicação.
								Formam-se grandes fendas no terreno, acompanhadas de desligamento. Há grandes escorregamentos de terrenos.
\\XII 		&	Cataclismo 		&	Destruição total. Modificação da topografia. Nunca foi presenciado no período histórico. \\
\hline
\end{tabular}
\caption{Escala simplificada de intensidade sísmica,
modificada em 1956 derivada da escala original de Giuseppe Mercalli de 1902.}
\label{tab:mercalli}
\end{scriptsize}
\end{center}
\end{table}

Existem estudos \citep{bakun_1999} que propõem a inferência sobre o tamanho da ruptura, e sua
magnitude, a partir de observações macrossísmicas, ou dos efeitos relatados pela escala de intenside, georreferenciados.


%% ------------------------------------------------------------------------- %%
\subsection{Catálogos}
\index{catálogos}
\label{sec:catalogos}

Os catálogos podem ser vistos como uma coleção de parâmetros sobre os tremores.
Podem ser classificados em três categorias \citep{woessner_2010} enumeradas a seguir:

\begin{itemize}\setlength{\itemsep}{0em}
	\item Pré-históricos: baseados na coleta de dados feitas por
	geólogos estruturais em trincheiras ou campos de subsidência. Podem conter registros de tremores que ocorreram
	há milhares de anos.
	\item Históricos: catálogos formados a partir de relatos históricos e inferência de valores de intensidade
	(seção \ref{sec:intensidade}), de análises de forma de onda com instrumentos antigos (registros em papel), eventualmente
	digitalizados.
	Cobrem o período das primeiras descrições humanas até os catálogos intrumentais.
	\item Instrumentais: são os catálogos de sismicidade definidos por dados produzidos por uma rede sismográfica bem estabelecida
	 gerando localizações continuamente (que começam a existir a partir de 1960).
\end{itemize}

Os catálogos instrumentais são uma listagem onde se epera encontrar, para cada evento, as
seguintes informações:

\begin{itemize}\setlength{\itemsep}{0em}
	\item algum identificador,
	\item a localização (\gls{hypocenter}) do evento (longitude, latitude, profundidade) em algum sistema de referência,
	\item o tempo de origem (data e hora) com precisão de pelo menos décimos de segundo e
	\item uma ou várias informações de \glsdesc{sym:m}.
\end{itemize}

Adicionalmente, embora não seja muito frequente, podem ser fornecidas informações adicionais obtidas pela análise das
formas de onda, como:

\begin{itemize}\setlength{\itemsep}{0em}
	\item incertezas sobre as magnitudes,
	\item incertezas sobre a localização (erro padrão, elipses de erro, cobertura dos sismogramas em diversas distâncias, cobertura
	dos sismogramas em vários ângulos azimutais, acurácia do modelo de velocidades utilizado, para enumerar alguns),
	\item intensidade máxima,
	\item intensidade no epicentro,
	\item número de, e as vezes as próprias, informações usadas para a determinação do hipocentro e hora de origem,
	\item sobre o mecanismo (alinhamento, mergulho e sentido do deslocamento na falha geológica) focal, entre outras.
\end{itemize}

É importante salientar \citep{woessner_2010} que cada um dos parâmetros determinados é fruto de uma série de decisões
e etapas de processamento. Começam pela escolha dos sismômetros e pelos locais onde serão instalados para registrar as
formas de onda. Sinais acima do nível de ruído são associados à chegadas de fases quando registradas em mais de uma estação.
A localização e o tempo de origem são determinados juntando-se os tempos de chegadas das fases a um modelo de velocidade
das ondas ao longo de camadas da crosta (ao qual a localização é extremamente dependente). As magnitudes, por fim, são
computadas a partir das amplitudes e/ou da duração do sinal, dependendo profundamente da calibração dos instrumentos.



\subsection{\glsdesc{mfd}}
\index{MFD}
\label{sec:mfd}

Observa-se que os sismos menores são muito mais frequentes.
Entretanto, os maiores e mais raros são os que trazem a maior ameaça e os que causam as maiores perdas.
Em virtude disso, uma análise conveniente seria explorar como se distribuem as magnitudes.

\subsubsection{\glsdesc*{mfd} (\gls{mfd}) de Gutenberg-Richter}
\index{Gutenberg-Richter MFD}
\label{sec:grmfd}

\citet{gutenberg_1944} observaram empiricamente que a distribuição da frequência de ocorrência dos tremores e
das magnitudes seguiam uma distribuição (de Pareto) cuja versão clássica é apresentada na equação \eqref{eq:gr_mfd} a
seguir:

\begin{equation}
	\gls{eqn:gr_mfd}
	\label{eq:gr_mfd}
\end{equation}
\glsdesc*{eqn:gr_mfd}.

Com uma simples transformação de variáveis ($\alpha = 10^a$ e $\beta = b\ln{10}$), observa-se que o número de sismos que ocorrem
com magnitudes dentro de um pequeno intervalo $[m, m+\mathrm{d}m]$ é

\begin{equation}
	\begin{split}
		\gls{sym:N_m} &= 10^{\gls{sym:a} - \gls{sym:b}\gls{sym:m}} \\
					  &= \alpha e^{-\beta m}
	\end{split}
	\label{eq:gr_exp}
\end{equation}

A distribuição cumulativa, ou seja, o número de eventos com magnitude maior que um certo valor $m_{min}$, é apresentada
na equação \eqref{eq:gr_cum}:

\begin{equation}
	\begin{split}
		N(m > m_{min}) &= \alpha \int\limits_{m_{min}}^{\infty}e^{-\beta m}\mathrm{d}m \\
					   &= \frac{\alpha}{\beta} e^{-\beta m} \\
					   &= \alpha_{cum} e^{-\beta m}.
	\end{split}
	\label{eq:gr_cum}
\end{equation}
onde $\alpha_{cum} = \alpha / \beta $ é o valor cumulativo da atividade sísmica.


Entretanto, a distribuição clássica de \gls{gr} não impunha restrições sobre um limite inferior $m_{min}$
ou superior $m_{max}$ à validade da distribuição.



\subsubsection{MFD Truncada}
\index{MFD Truncada}
\label{sec:TMFD}

Variações da distribuição clássica de \gls{gr} foram propostas em vista de melhor representar as MFD estudadas à partir de
catálogos de diversas regiões.

A equação \eqref{eq:gr_max} versão incremental truncada com um limite superior $m_{max}$:

\begin{equation}
		\gls*{sym:N_m} = \frac{e^{-\beta m}}{1 -e^{-\beta m_{max}} }, m \leq m_{max}
	\label{eq:gr_max}
\end{equation}

Na equação \eqref{eq:gr_dtr} versão incremental duplamente truncada com um limite inferior $m_{min}$ e superior $m_{max}$ :

\begin{equation}
		\gls*{sym:N_m} = \frac{e^{-\beta (m - m_{min})}}{1 -e^{-\beta (m_{max} - m_{min}) } } , m_{min} \leq m \leq m_{max}
	\label{eq:gr_dtr}
\end{equation}

A figura \ref{f:mfd} ilustra essas distribuições.

\subsubsection{MFD Limitada}
\index{MFD Limitada}
\label{sec:BMFD}

Outra possibilidade, é limitar suavemente a parte final da curva (ver figura \ref{f:mfd}). A equação \eqref{eq:gr_bounded} apresenta
a distribuição:

\begin{equation}
		\gls*{sym:N_m} = \alpha [ e^{-\beta (m - m_{min})} - e^{-\beta (m_{max} - m_{min}) } ], m_{min} \leq m \leq m_{max}
	\label{eq:gr_bounded}
\end{equation}


\subsubsection{MFD com decaimento exponencial}
\index{MFD com decaimento exponencial}
\label{sec:KMFD}

Yan Kagan \citep{kagan_2002} propôs uma distribuição de magnitude mais adequada e acoplada à energia liberada pelos sismos, que
pode ser descrita como na equação \eqref{eq:gr_tapered}:

\begin{equation}\ensuremath{
		\gls*{sym:N_m} = [\gls*{sym:beta_p} + \frac{m}{m_{min}}]
				m_{min}^{\gls*{sym:beta_p}}
				\gls*{sym:m_corner}^{-1 -\gls*{sym:beta_p}}
				e^{\frac{m_{min} - m}{\gls*{sym:m_corner}}},
				m_{min} \leq m < \infty
		}
	\label{eq:gr_tapered}
\end{equation}
onde \glsdesc*{sym:beta_p} e \gls*{sym:m_corner} \glsdesc*{sym:m_corner}

\begin{figure}[H]
   \centering
   \includegraphics[width=0.95\textwidth]{mfd}
   \caption[Distribuições de frequência e magnitude]
   		   {Distribuições de frequência e magnitude}
   \label{f:mfd}
\end{figure}

A figura \ref{f:mfd} apresenta um comparativo de algumas distribuições. Para ilustração,
há também na figura um histograma de um catálogo de
uma pequena região do norte do Chile, onde se pode observar que tanto a porção inferior (em torno de $m=5$), como a porção
posterior ($m > 7$) do histograma não seguem perfeitamente a distribuição. Há essencialmente duas zonas críticas em que é preciso
estar atento à física do problema:
(i) na parte inferior, muitos sismos de magnitude pequena não são registrados, seja por não terem energia suficiente
para sensibilizar um conjunto razoável de estações que permitam determinar suas localizações, seja porque o número de
estações é insuficiente na região onde os pequenos tremores ocorrem; (ii) a parte superior, por sua vez, é crítica
por se acoplar diretamente aos limites físicos do tamanho da maior ruptura possível, relacionada diretamente ao limite de liberação de energia na forma de momento sísmico $M_0$.

Nas distribuições de magnitude e frequência é importante que se possa reconhecer claramente alguns parâmetros
fundamentais.

\subsection{Valor-b}
\index{valor-b}
\label{sec:b_value}

O \emph{valor-b} foi apresentado na seção \ref{sec:grmfd} como sendo a inclinação da reta que representa a parte linear
descrescente da distribuição. Representa a proporção entre sismos pequenos e catastróficos que uma determinada fonte
sísmica é capaz de produzir (figura \ref{f:occurrence}).

\begin{figure}[H]
   \centering
   \includegraphics[width=0.95\textwidth]{occurrence}
   \caption[Distribuição incremental e cumulativa de frequência e magnitude dos sismos presentes no catálogo ISC-GEM
   para a América do Sul unido com o \gls{bsb2013}]
   {Distribuição incremental e cumulativa de frequência e magnitude dos sismos presentes no catálogo ISC-GEM
   para a América do Sul unido com o \gls{bsb2013}}
   \label{f:occurrence}
\end{figure}
%\footnotetext{\citet{img_opensha_rupture}}




\subsection{Taxa de Sismicidade}
\index{taxa de sismicidade}
\label{sec:seismic_rate}

A taxa de sismicidade é a medida da ocorrência dos tremores por uma determinada unidade de tempo (geralmente anos).
Representa para cada magnitude, a frequência média de ocorrência de sismos.

\subsection{Valor-a}
\index{valor-a}
\label{sec:a_value}

O \emph{valor-a} é a interseção da \gls{mfd} no eixo das frequências e representa o nível geral de sismos
que as fontes observadas pelo catálogo são capazes de produzir.

Costuma ser confundido pela forma de representação adotadas para a distribuição (incremental e/ou
cumulativa) e pelos truncamentos onde por vezes se apresenta o valor-a como a taxa de sismicidade da
magnitude mínima ou de completude do catálogo.

No presente trabalho o \emph{valor-a} significará sempre o da distribuição cumulativa de sismos por unidade de
tempo com magnitudes positivas.

\subsection{Magnitude de Completude $M_c$}
\index{magnitude de completude}
\label{sec:completeness}

A magnitude de completude é o valor mínimo para o qual a distribuição é capaz de observar completamente o conjunto
de sismos. Em outras palavras representa o limite de observação completa do catálogo.

Sua identificação é bem simples quando se observa a distribuição incremental de magnitudes. É facilmente notado o valor
de magnitude na porção inferior na qual o número de sismos registrados começa a divergir da tendência geral da
distribuição.

Seu mapeamento é importante uma vez que os métodos de ajuste e determinação dos parâmetros da distribuição baseados na
máxima verossimilhança \citep{aki_1965, weichert_1980} dependem fundamentalmente desse valor mínimo.


\section{Risco Sísmico}
\index{Risco Sísmico}
\label{sec:risco_sismico}

A redução do risco sísmico é um problema complexo que envolve geralmente muitas pessoas, informações, decisões e ações.

A palavra risco, ao pé da letra, significa a exposição à possibilidade de injúria ou perda. E geralmente é usada como
sinônimo de ameaça. Na literatura acerca do tema risco, inclusive, as palavras risco e ameaça são usadas com certa
confusão.

No glossário da \gls{eeri} (EERI Committee on Seismic Risk, 1984) a definição de risco sísmico é a probabilidade de
que perdas sociais ou econômicas aconteçam como decorrência de tremores por superarem limiares estabelecidos para
determinado local ou região durante um certo período de exposição.

A ameaça sísmica, por outro lado, é qualquer fenômeno físico (oscilação, falhamento) associado à terremotos que possam
produzir efeitos adversos às atividades humanas. Na prática são avaliados por dadas probabilidades de ocorrência.

Pode-se deduzir que o risco sísmico é então uma combinação da ameaça sísmica com outros fatores:

\begin{equation}
		\text{Risco Sísmico} = \text{Ameaça Sísmica} \ast \text{Vulnerabilidade} \ast \text{Valor Exposto},
	\label{eq:risk}
\end{equation}

onde a vulnerabilidade é a quantidade de danos induzidos por um dado grau de ameaça e expressa como uma fração do
valor exposto ao dano e varia de acordo com o modelo proposto.

Frequentemente, o fator vulnerabilidade advém das análises das (ii) respostas das estruturas edificadas ao espectro de
acelerações produzidos pela (i) provável ameaça sísmica e da análise de possíveis (iii) danos estruturais à edificação.

A decisão de alterar ou não o desenho estrutural das edificações é feito a partir da análise dos (iv) prejuízos
(quantidade de moeda, mortes, tempo inoperante) causados caso as estruturas sejam danificadas conforme as análises anteriores.


\section{Ameaça Sísmica}
\index{ameaça sísmica}
\label{sec:ameaca_sismica}

A ameaça sísmica poderia ser definida de modo geral como a possibilidade de ocorrerem efeitos potencialmente destrutivos
de um terremoto em uma particular localização. Com exceção de \textit{tsunamis} ou falhamentos geológicos superficiais,
todos os efeitos destrutivos de um tremor de terra estão diretamente relacionados ao movimento do chão induzido pela
passagem das ondas sísmicas. Existem, entretanto diferenças de abordagem para a avaliação da ameaça sísmica.

A \glsdesc{psha} foi introduzida por \citet{cornell_1968}, aprimorada por \citet{mcguire_1976} e se tornou técnica mais
amplamente utilizada para a avaliação da ameaça sísmica. Também é possível fazer essa avaliação deterministicamente,
através de cenários definidos pelo espectro de movimento forte do chão que pode ser causado pela ocorrência de um
determinado tremor de terra em certa localização e de certa magnitude. O possível espectro de movimento no local de
interesse é avaliado através de relações de atenuação ou \glspl{gmpe}.

Os mecanismos da \gls{psha} \citep{bazzurro_1999, abrahamson_2006} são menos óbvios do que os da \gls{dsha}
\citep{reiter_1991, kramer_1996}, e em essência significam identificar todos os possíveis tremores que podem afetar o local de interesse,
incluíndo todas as possíveis combinações de distâncias e caracterizar a frequência de ocorrência das diferentes magnitudes através de relações de recorrência. As equações de
atenuação são utilizadas para calcular os parâmetros do movimento do chão no local de interesse devido a esses tremores
e consequentemente a taxa com que diferentes níveis de movimento do chão ocorram no local de interesse.

Seus resultados também apresentam certa distinção. Se por um lado a \gls{psha} traz consigo o aspecto
temporal, ou a taxa com que diferentes níveis de aceleração excederão determinado limiar em determinado local de interesse,
por outro, a \gls{dsha} apresenta o movimento do chão esperado quando ocorra determinado evento de controle.


\section{Projeção da Ocorrência de Rupturas}
\index{projeção de ocorrência de rupturas}
\label{sec:projecao}

As projeções (\textit{forecasting}) são feitas para se estimar a ocorrência de futuros tremores
\citep{kagan_2000,marzocchi_2011}, principalmente dos maiores, com grandes chances de provocar perdas.

Nas de curto prazo, estimam-se os próximos tremores
numa escala de dias ou horas considerando uma taxa de sismicidade variável
com o tempo como no caso dos pré e pós-abalos, ou de quando
acontece um enxame sísmico, período de maior atividade numa região.
Sua principal aplicação é auxiliar a tomada de decisões de curto período,
como evacuação de edifícios.

Nas de longo prazo, foco desse texto, a principal consideração feita é de que a
\gls{seismic_rate} não varie ao longo do tempo, servindo para estimar as acelerações
provocadas por tremores, mesmo que possam ocorrer
muito raramente, de grandes proporções.

São geralmente aplicadas quando
se deseja saber o nível de segurança e resistência estrutural que devem ser impostos
às edificações em geral, ou quando se deseja estimar o valor de um contrato de resseguro
de algum outro grande investimento.


\section{Análise Probabilística de Ameaça Sísmica }
\index{PSHA, Análise Probabilística de Ameaça Sísmica}
\label{sec:psha}


Na \gls{psha} são considerados todos os possíveis tremores, as rupturas que os originaram e os movimentos do chão
resultantes conjuntamente com suas probabilidades de ocorrência associadas de modo a encontrar o nível de movimento do
chão que será excedido, numa janela de tempo, com uma pré-definida baixa tolerância \citep{baker_2008-1}.

Uma das formas de se enxergar o resultado de uma análise de ameaça sísmica
é como uma estimativa da pequena probabilidade $\xi$, em uma janela de tempo dada,
com que determinada medida de intensidade $I$ é raramente excedida.

Na análise probabilística de ameaça sísmica o nível de confiança $\xi$
e a janela de tempo são fixadas (por exemplo, $t$ anos). A tarefa é então estimar
o valor da intensidade $I$, em um determinado local $S$, de forma que a probabilidade
do evento
\begin{equation} \label{eventEt}
\begin{array}{lll}
 E_t(I, S)&=&\{\mbox{Haja pelo menos um evento causando intensidade }  \\
 &&  \;\;\mbox{maior que $I$ em }S  \mbox{ nos próximos }t \mbox{ anos}\} \\
\end{array}
\end{equation}
seja $\xi$.


\subsection{Método de Zoneamento}
\index{PSHA!zoneamento}
\label{sec:zoneamento}

Na ausência de informações geológicas mais precisas, o método introduzido por \citet{cornell_1968} e
\citet{mcguire_1976} para modelar e resolver essse problema consiste
primeiramente em identificar quais zonas sísmicas (que em geral não se sobrepõem)
podem ter impacto sobre o valor de intensidade em $S$.

O número de tremores que povocam intensidade em $S$ maior que $I$ em $t$ anos depende da frequência
de tremores em cada zona. O valor da intensidade $I$ provocada por cada tremor depende da magnitude,
aleatória, desses tremores e de sua localização, também aleatória.

Para considerar esses fatores Cornell \& McGuire propõem que
\begin{itemize}
\item[(i)] em cada zona $i$, o processo de ocorrência de tremores seja
modelado como um processo de Poisson com taxa $\lambda_i$, assumindo que os tremores em diferentes zonas são
independentes.
\item[(ii)] Numa zona $i$, a magnitude dos tremores é modelada como uma \gls{va}
com densidade $f_{M_i}(\cdot)$.
\item[(iii)] A distância entre cada tremor da zona $i$ e o local $S$ é modelada como uma \gls{va}
com densidade $f_{D_i}(\cdot)$.
\item[(iv)] O modelo de predição do movimento do chão (\gls{gmpe}) é expresso pela regressão da medida de intensidade
em magnitude, distância, condições geológicas do local $S$ e outros fatores.
\end{itemize}


A habilidade de se calcular a probabilidade do evento
\eqref{eventEt} para qualquer $I$ faz com que seja possível a estimativa, por dicotomia, de uma intensidade $I$ que satisfaça
$P\left\{ E_t(I, S) \right\} \geq \xi$.


\subsection{Identificação das Fontes Sísmicas}
Para identificar fontes sísmicas são utilizados desde registros históricos de sismicidade à evidências geológicas de
falhamentos/rupturas datados com deslocamento e magnitudes inferidos e busca-se aproveitar de toda informação relevante
disponível, como a medida secular de deslocamento relativo entre observações geodésicas contínuas ou mesmo da
sismicidade recente.

Quando se identifica uma fonte sísmica é comum representá-la por uma forma geométrica simples mas consistente com o
conjunto das observações disponíveis para descrever as possíveis rupturas \citep{crowley_2013}.

\subsubsection{Ponto}
\index{fonte sísmica!ponto}
\label{sec:point_source}

Se toda informação disponível é uma localização isolada de um tremor antigo, com magnitude e com mecanismo de
falhamento conhecido, é possível representá-lo como uma fonte sísmica de tipo pontual.
Nesse tipo de fonte são definidos os limites superior e inferior da ruptura,
sua orientação e tipo de falhamento (quando disponível)
e o hipocentro era definido como o centro de cada ruptura. Recentemente porém é possível definir uma distribuição de
progundidades hipocentrais.

\subsubsection{Área}
\index{fonte sísmica!área}
\label{sec:area_source}

Quando o conhecimento sobre a geologia, a tectônica, ou mesmo a correlação espacial dos tremores no catálogo permitam
o delineamento de zonas ou áreas com características sísmicas comuns se costuma representar por um polígono na
superfície.

Essas áreas, para efeito de cálculo, são discretizadas como um conjunto de fontes sísmicas de
característica pontual distribuídas uniformemente por toda área.

\subsubsection{Falha Simples}
\index{fonte sísmica!falha simples}
\label{sec:simple_fault_source}

Muitas vezes os parâmentros de um falhamento ativo são claramente conhecidos e monitorados. Isso permite uma maior
especificidade na representação da fonte sísmica, restringindo mais, por exemplo, as flutuações na orientação das
rupturas.
Nesse caso a geometria da falha se caracteriza pela projeção do traço de falha na superfície e pelos limites superior e
inferior da ruptura no plano de mergulho (ver figura \ref{f:rupture}).


\subsubsection{Falha Complexa}
\index{fonte sísmica!falha complexa}
\label{sec:complex_fault_source}

Casos de sismicidade em zonas de subducção ou encontro de placas, de contexto geológico mais complexo
geralmente apresentam variações laterais, de mergulho, de acúmulo de esforços, de orientação, etc. Fontes sísmicas em
situações como essa são modeladas por um poliedro unido de forma suave.


\subsection{Caracterização da \glsdesc*{mfd}}
\index{caracterização da \glsdesc*{mfd}}
\label{sec:psha_mfd}

Conhecida a fonte sísmica e sua representação geométrica, é preciso caracterizar sua capacidade sismogênica determinando
uma (ou mais) possíveis \gls{mfd}s que se ajustam às observações. Isso inclui a espressão matemática da distribuição, a
taxa geral de sismicidade (\emph{valor-a}) e frequentemente as magnitudes mínima e máxima.


A densidade $f_{M_i}(\cdot)$ usada para a distribuição de magnitude dos tremores na zona $i$ depende do
histórico de magnitudes na mesma zona, e possui uma das formas funcionais da seção \ref{sec:mfd},
como por exemplo a distribuição duplamente truncada,
com $M_i$ sendo o intervalo entre as magnitudes mínima e máxima $[M_{\min}(i), M_{\max}(i)]$
em cada zona $i$.


\subsection{Caracterização da Distribuição de Distâncias}
\index{caracterização da distribuição de distâncias}
\label{sec:psha_distances}

Dados um local de interesse e uma provável ruptura em uma fonte sísmica é necessário calcular a distribuição das
distâncias da fonte, isto é, das possíveis rupturas, ao local em que se deseja avaliar a ameaça.

Em cada zona as rupturas são consideradas como tendo distribuição espacial uniforme $f_{D_i}(\cdot)$.
A distribuição das distâncias, usadas pelas equações de atenuação, podem então ser calculadas
analiticamente ou por aproximação.


\subsection{Predição do Movimento do Chão}
\index{predição do movimento do chão}
\label{sec:gmpe}

Para se estimar os possíveis níveis de movimento do chão causados por eventos de uma determinada magnitude à uma certa
distância do local de interesse são utilizadas as equações de predição de movimento do chão (\glspl{gmpe}).

As GMPEs são modelos (equações e coeficientes) de regressão representando certo valor de
intensidade $I$ induzida por um tremor de magnitude $M$ a uma distância $D$ do epicentro (ruptura, hipocentro, etc, dependendo da
modelagem da GMPE) mas que depende também de outros fatores $\theta$ que considerem as condições
do local de estudo e do tipo de falhamento por exemplo. Geralmente as GMPEs tem a forma

\begin{equation} \label{pgamodel}
\ln I = \overline{\ln I}(M, D, \theta) + \sigma(M, D, \theta) \varepsilon.
\end{equation}

Nessa relação, $\overline{\ln I}(M, D, \theta)$ (e respectivamente $\sigma(M, D, \theta)$)
é a média condicional (e desvio padrão) de $\ln I$ para certa magnitude $M$, distância $D$
e condições $\theta$, enquanto $\varepsilon$ é uma \gls{va} gaussiana padrão.

Essa média $\overline{\ln I}(M, D, \theta)$ deve crescer com $M$ já que quanto maior a magnitude,
maior a intensidade provocada, e, por outro lado, decrescer com a distância $D$ pois quanto maior a distância, menor
a intensidade, como é possível observar nas figuras \ref{fig:gmpe_magnitude} e \ref{fig:gmpe_distance}.

\begin{figure}[H]
	\centering
	\begin{subfigure}[t]{0.47\textwidth}
		\centering
		\includegraphics[width=1.0\textwidth]{gmpe_magnitude}
		\caption{Variação da intensidade \gls{pga} com a magnitude.}
		\label{fig:gmpe_magnitude}
	\end{subfigure}
	\quad
	\begin{subfigure}[t]{0.47\textwidth}
		\centering
		\includegraphics[width=1.0\textwidth]{gmpe_distance}
		\caption{Variação da intensidade \gls{pga}) com a distância de Joyner-Boore.
		  para um valor de magnitude de 5.05 }
		\label{fig:gmpe_distance}
	\end{subfigure}
	\caption{Variação das intensidades com a magnitude e distância, para diferentes GMPEs}
	\label{fig:gmpe}
\end{figure}

A distância de \citet{joyner_1981} é uma das várias medidas de distância utilizadas e corresponde especificamente à
menor distância entre o local de avaliação da ameaça $S$ à projeção da ruptura na superfície da topografia.

Apenas como exemplo, a forma funcional que \citet{toro_1997} dão à GMPE é

\begin{equation}
\begin{array}{lcl}
\ln I(M, R_M, \theta)
&=& \theta_1 + \theta_2(M-6) + \theta_3(M-6)^2  \\
& & - \theta_4\ln (R_M) -(\theta_5 - \theta_4)\max\left[ \ln\left( \frac{R_M}{100} \right), 0 \right] -\theta_6 R_M \\
& & + \varepsilon_e + \varepsilon_a, \\
\end{array}
\end{equation}
com
$$
R_M =  \sqrt{ R_{jb}^2 + \theta_7^2 }
$$
onde $R_{jb}$ é a distância de Joyner-Boore, $M$ é a magnitude de momento,
$\theta_i$ são as componentes do vetor de coeficientes e $\varepsilon_e$ + $\varepsilon_a$
são as incertezas epistêmica e aleatória do modelo.

Nesse exemplo
\begin{equation}
\begin{array}{lcl}
\overline{\ln I}(M, D, \theta) & = & \theta_1 + \theta_2(M-6) + \theta_3(M-6)^2  \\
& & - \theta_4\ln (R_M)
-(\theta_5 - \theta_4)\max\left[ \ln\left( R_M / 100 \right), 0 \right]
-\theta_6 R_M
\end{array}
\end{equation}
e
\begin{equation}
	\sigma(M, D, \theta)=\varepsilon_e(M,D) + \varepsilon_a(M,D).
\end{equation}


\subsection{Combinação de Incertezas e Avaliação da Ameaça Sísmica}
\index{cálculo da ameaça}
\label{sec:hazard}

A distribuição do número de tremores $N_{t i}$, na zona $i$, na janela de tempo $t$, por Poisson,
é dada por
$$
P(N_{t i}=k)=e^{-\lambda_i t} \frac{(\lambda_i t)^k}{k!},\;k \in \mathbb{N},
$$
onde a taxa $\lambda_i$ representa o número médio de tremores na zona
$i$ por unidade de tempo, por exemplo, por ano.

Fixado um valor de intensidade $I$, o evento
\begin{equation} \label{formulapi}
\begin{array}{lll}
E(I, S, i)  & =  & \{ \mbox{Um tremor da zona }i \mbox{ gera uma intensidade} \\
&  &  \;\;\mbox{maior ou igual a} I \mbox{ em }S \}
\end{array}
\end{equation}
tem probabilidade $p_i=P \left\{ E(I, S, i)\right\}$.

Para cada tremor na zona $i$, o evento $E(I, S, i)$ ocorre, ou não.
Como resultado, é possível definir dois novos processos de contagem
em cada zona $i$: o processo $\tilde N_{t i}$, contando os tremores que
causam intensidade maior ou igual a $I$ em $S$ e os que causam intensidade menor que $I$ em $S$.

\begin{figure}[H]
	\centering
	\begin{tabular}{l}
	\includegraphics[width=0.80\textwidth]{poisson}
	\end{tabular}
	\caption{Separação do processo de chegada dos tremores da zona $i$
	em um processo causando, em $S$, intensidade maior que $I$ (bolas pretas) e intensidade
	menor ou igual a $I$ (bolas brancas).}
\label{fig:poisson}
\end{figure}

É preciso utilizar o seguinte
\begin{lemma}
Considere um processo de Poisson $N_t$ com taxa de ocorrência $\lambda$.
Assumindo que as chegadas são de dois tipos: tipo 1 com probabilidade $p$,
e tipo 2 com probabilidade  $1 - p$ e que elas são independentes,
então o processo $\tilde N_t$ do tipo 1 é um proceso de Poisson com taxa
$\lambda p$.
\end{lemma}
\begin{proof} Para cada $k \in \mathbb{N}$, calcule
$$
\begin{array}{lll}
P\Big(\tilde N_{t} =k \Big) & = & \displaystyle \sum_{j=k}^{+\infty} P\Big(\tilde N_{t} =k | N_{t}
=j\Big)  P \Big( N_{t} =j  \Big)\;\;\;\mbox{[Teorema da Probabilidade Total]}\\
&=& \displaystyle \sum_{j=k}^{+\infty} C_j^k p^k (1-p)^{j-k} e^{-\lambda t} \frac{(\lambda t)^j}{j!} \\
&=&e^{-\lambda t} \frac{(\lambda p t)^k}{k!} \displaystyle \sum_{j=0}^{+\infty} \frac{[\lambda (1-p)]^{j}}{j!} = e^{-\lambda p t} \frac{(\lambda p t)^k}{k!},
\end{array}
$$
o que mostra que $\tilde N_{t}$, usando a independência dos tipos de chegada em
intervalos disjuntos, é uma \gls{va} de Poisson com parâmetro
$\lambda p t$.\hfill
\end{proof}

Esse lema mostra que o processo $(\tilde N_{t i})_t$ é um processo de Poisson com taxa $\lambda_i p_i$.

Sendo $\mathcal{N}$ o número de zonas sísmicas, segue que a probabilidade de que haja $k$ sismos causando
intensidade maior que $I$ em $S$ na janela de tempo $t$ é
$$
\displaystyle
\begin{array}{lll}
P\Big(\displaystyle \sum_{i=1}^{\mathcal{N}} \tilde N_{t i} = k\Big)
&= &\displaystyle \sum_{x_1+\ldots +x_{\mathcal{N}}=k} P\Big(\tilde N_{t 1} = x_1; \ldots; \tilde N_{t \mathcal{N}} = x_{\mathcal{N}} \Big)\\
&= &\displaystyle{\sum_{x_1+\ldots +x_{\mathcal{N}}=k} \prod_{i=1}^{\mathcal{N}}}P\Big(\tilde N_{t i} = x_i \Big)\\
&= &\displaystyle \sum_{x_1+\ldots +x_{\mathcal{N}}=k} \prod_{i=1}^{\mathcal{N}} e^{-\lambda_i p_i t} \frac{(\lambda_i p_i t)^{x_i}}{x_{i}!}
\end{array}
$$
em que para a segunda igualdade foi utilizada a independência de  ${\tilde N}_{t 1}, \ldots, {\tilde N}_{t
\mathcal{N}}$.


Para $k=0$ na relação acima obtêm-se que
\begin{equation} \label{probinterest}
1-P( E_t(I, S) )= P(\overline{E_t(I, S)})=e^{-(\sum_{i=1}^{\mathcal{N}} \lambda_i p_i) t}.
\end{equation}

Com $\tilde N_t=\sum_{i=1}^{\mathcal{N}} \tilde N_{t i}$,
o valor esperado $\tilde N_t$ é o número médio de tremores causando intensidade maior que $I$ em $S$
 nos próximos $t$ anos e pode ser espresso por
\begin{equation} \label{formlambdatA}
\lambda_t(I, S)=\mathbb{E}\Big[ \tilde N_t \Big] =\sum_{i=1}^{\mathcal{N}} \mathbb{E}\Big[ \tilde N_{t i}\Big] =
\left( \sum_{i=1}^{\mathcal{N}} \lambda_i p_i \right) t.
\end{equation}

Usando essa relação e a equação \eqref{probinterest}, a probabilidade do evento $E_t(I, S)$
pode ser reescrita
\begin{equation}
P \left\{   E_t(I, S) \right\} =1-e^{-\lambda_t(I, S)}
\end{equation}
com $\lambda_t(I, S)$ dado por \eqref{formlambdatA}.


\subsubsection{Combinando todos os elementos}

Assumindo que as variáveis aleatórias da distância $D_i$ entre o tremor na zona $i$ e o local $S$ e magnitude $M_i$
são independentes e usando o Teorema da Probabilidade Total, obtém-se que
\begin{equation}
p_i= \displaystyle \int_{m_i =M_{\min}(i)}^{M_{\max}(i)}
\int_{x_i =0}^{\infty}  P \left\{ \mbox{\small intensidade \normalsize }>I| M_i = m_i; D_i = x_i \right\}f_{M_i}(m_i) f_{D_i}( x_i )
\mathrm{d}m_i \mathrm{d}x_i
\end{equation}
onde $P \left\{ \mbox{\small intensidade \normalsize }>I| M_i = m_i; D_i = x_i \right\}$
é dada pelo de predição do movimento do chão \eqref{pgamodel}.

Para implementação, a integral acima é geralmente estimada discretizando as distribuições contínuas de
magnitude $M_i, i=1,\ldots, \mathcal{N}$, e distância $D_i, i=1,\ldots, \mathcal{N}$.

         % associado ao arquivo: 'cap-conceitos.tex'
%% ------------------------------------------------------------------------- %%
\chapter{Região de Estudo}
\label{cap:regiao_de_estudo}


Esse capítulo apresenta a região de estudo sob o ponto de vista tectônico e sismológico.

%% ------------------------------------------------------------------------- %%
\section{Contexto Geológico e Tectônico Sul-Americano}
\index{\gls{tectonic}!América do Sul}
\label{sec:03_america_do_sul}

A placa Sul-Americana, como mostra a figura \ref{fig:sa_plate} \citet{bizzi_2003}, tem ao norte a placa do Caribe e a
placa Norte-Americana.
Ao sul estão a placa de Scotia e a placa Antártica. Todas elas se deslocam 
majoritariamente tangencialmente
à placa Sul-Americana.

\begin{figure}[H]
  \centering
  \includegraphics[width=.95\textwidth]{placas_sa} 
  \caption{Placa Sul-Americana em seu contexto global}
  \label{fig:sa_plate} 
\end{figure}


Na divisa com placa Africana à leste está a Dorsal Meso-Atlântica que é resultado
do processo de abertura dos oceanos e separação dos continentes. A abertura dos Atlântico na 
dorsal é responsável por um considerável esforço de compressão horizontal na placa Sul-Americana.

E há também a subducção da placa de Nazca sob a placa Sul-Americana, à oeste,
responsável, entre outros processos, pelo surgimento das Fossas Oceânicas e da
cordilheira dos Andes com suas altitudes e vulcanismo.

Olhando um pouco mais de perto para a parte continental da placa Sul-Americana
(figura \ref{fig:sa_tec}) é interessante destacar três grupos principais de rochas: 
(i) o Embasamento Pré-Cambriano, (ii) as Coberturas Fanerozóicas e (iii) a 
Cadeia Andina.

\begin{figure}[H]
  \centering
  \includegraphics[width=.85\textwidth]{lithology_sa} 
  \caption{Mapa geológico da América do Sul. Fonte \url{http://onegeology.org}}
  \label{fig:sa_tec} 
\end{figure}

As rochas do embasamento pré-cambriano se originaram a mais de 500Ma. Por serem mais antigas
são mais estáveis. A cobertura Fanerozóica é resuldado da sedimentação ocorrida a menos de 250Ma. Formam as bacias
sedimentares. A Cadeia Andina com 30Ma é resultado da subducção e embora exponha rochas pré-cambrianas
em algumas partes é a parte mais ativa e interessante tectonicamente.


\subsubsection{Sismicidade Sul Americana}

A sismicidade Sul-Americana é marcada fortemente pela subducção à oeste e pela 
separação dos oceanos à leste. América Central, Caribe e a parte Antartica ao sul
são placas menores e seus movimentos merecem estudo de maior detalhe.

\begin{figure}[H]
  \centering
  \includegraphics[width=.70\textwidth]{seismicity_sa} 
  \caption{Sismicidade da América do Sul, Catálogo \gls*{iscgem}. 
  		   Geologia: \gls*{cgmw} via OneGeology.
		   Sismos mais profundos foram registrados no interior da placa, inclusive sobre o Acre.
  		   }
  \label{fig:sa_seis} 
\end{figure}

A figura \ref{fig:sa_seis} apresenta a sismicidade da América do Sul pelo catálogo \gls{iscgem}
(seção \ref{sec:data_source}). É possivel notar claramente a subducção, ou seja, do mergulho, da placa de Nazca sob a placa
Sul-Americana. Fica mais claro quando se observa que os sismos com profundidade variando cerca de centenas de
quilômetros e que vão se tornando mais profundos para interior da placa Sul-Americana.

Isso acontece porque parte da quantidade de rocha fria, oceânica e continental, está afundando sob o manto
e lentamente se incorporando à ele. Mas ainda existem atrito, compressões e processos de ruptura nessas
profundidades e que ocorrem majoritariamente na interface entre essas placas pelo acúmulo de tensão e deslocamentos 
mínimos durante milhares de anos e que são liberados instantaneamente na ruptura. 
Esse processo é também o responsável pelo soerguimento da 
cordilheira dos Andes e de boa parte do vulcanismo na região. 

Sismos profundos, de 70 e 700km, geralmente provocam acelerações de baixa intensidade em seus epicentros 
devido à atenuação das ondas.

Também é latente a constatação da diferença de distribuição da sismicidade sul-americana nas bordas de placa e do
resto do continente como um todo. A maior parte do Brasil é praticamente inativa, não desprezível, no histórico
comparado.

%% ------------------------------------------------------------------------- %%
\section{Contexto Geológico e Tectônico Brasileiro}
\index{\gls{tectonic}!Brasil}
\label{sec:geotec_bras}

Tomando-se como referência 500Ma, destacam-se dois grandes grupos de rochas na figura \ref{fig:br_tec}
\citet{bizzi_2003} a seguir.

\begin{figure}[H]
  \centering
  \includegraphics[width=.70\textwidth]{tectonico_brasil} 
  \caption{Mapa Geológico do Brasil em escala 1:1.000.000}.
  \label{fig:br_tec} 
\end{figure}

O embasamento, mais antigo, exposto sob a Amazônia e em porções menores e mais recentes também expostos 
no sudeste e nordeste, cedem espaço à um segundo grupo de rochas mais jovens, fruto da sedimentação e metamorfismos
mais recentes \citep{bizzi_2003}.


%% ------------------------------------------------------------------------- %%
\section{Sismicidade do Brasil}
\index{Sismicidade do Brasil}
\label{sec:sismicidade_brasil}

No Brasil não há terremotos. Não ao menos em quantidade proporcional a 10\% dos sismos sul-americanos.
Mesmo assim, ou por isso mesmo, a ocorrência de um sismo é ainda mais ameaçadora. Onde se espera, 
já se está preparado. Por outro lado onde nunca se espera é sempre uma surpresa.

É fato que o Brasil por estar numa área continental, mais no interior da placa \citep{talwani_2014} e geologia
com uma formação antiga e estável tem um número reduzido, não desprezível, de tremores. 
A figura \ref{fig:br_seis} mostra o detalhe da sismicidade brasileira com a litologia ao fundo.


\begin{figure}[H]
  \centering
  \includegraphics[width=.90\textwidth]{seismicity_br} 
  \caption{Sismicidade do Brasil. Catálogo \gls{bsb2013} (seção \ref{sec:data_source2}).}
  \label{fig:br_seis} 
\end{figure}

É importante notar que já houve registro de sismos com magnitudes pouco acima de 6,
e que sismos de magnitude acima de 4, rasos, em área urbana e em um continente estável,
com baixa atenuação das amplitudes das ondas sísmicas, podem ser danosos.


%% ------------------------------------------------------------------------- %%
\subsection{Sul, Sudeste e Litoral Leste}
\index{Sismicidade do Brasil!sul, sudeste e litoral leste}
\label{sec:z_se}

A sismicidade do sudeste e seu litoral possui características diferentes.
Enquanto no litoral, a principal sismiciade ocorre na área do talude continental
(porção dos fundos marinhos com declive muito pronunciado que fica entre a plataforma continental e
a margem continental e onde começam as planícies abissais), com destaque nessa sismicidade para um dos maiores sismos
que se tem registro no Brasil (figura \ref{fig:z_se}).

\begin{figure}[H]
  \centering
  \includegraphics[width=0.6\textwidth]{z_se} 
  \caption{Zona sísmica do SE. \citet{dourado_2014}}
  \label{fig:z_se} 
\end{figure}

O continente é marcado por uma sismicidade difusa na área do cráton que se extende pelo norte de Minas
Gerais até quase o sul da Bahia e outra parte a nordeste da bacia do Paraná. 
Há também uma pequena sismicidade acompanhando a costa. Para maiores referencias, veja 
\citet{assumpcao_2004}.

É nessa região o único registro no Brasil de vítimas fatais decorrentes de tremores de terra, em Itacarambi, MG.

%% ------------------------------------------------------------------------- %%
\subsection{Centro-Norte}
\index{Sismicidade do Brasil!centro-norte}
\label{sec:z_cn}

É uma área com sismicidade peculiar. Na figura \ref{fig:z_cn}, de norte a sul, a sismicidade acompanha grosseiramente o
contato entre o cráton e parte da bacia do Parnaíba. O mesmo ocorre ambém no que seria a 
área central próxima à Chapada dos Veadeiros em uma outra formação cratônica.

\begin{figure}[H]
  \centering
  \includegraphics[width=0.45\textwidth]{z_co} 
  \caption{Zona sísmica do Centro-Norte. \citet{dourado_2014}}
  \label{fig:z_cn} 
\end{figure}

A última porção, ao sul, a sismicidade ocorre na área sedimentar da bacia do Pantanal.

%% ------------------------------------------------------------------------- %%
\subsection{Mato-Grosso}
\index{Sismicidade do Brasil!centro-norte}
\label{sec:z_mt}

A sismicidade do Mato-Grosso, mais precisamente de Porto-de-Gaúchos (\ref{fig:z_mt}), é emblemática para o Brasil.
Poderia ser considerada com características similares a Nova Madri, EUA.

\begin{figure}[H]
  \centering
  \includegraphics[width=0.45\textwidth]{z_mt} 
  \caption{Zona sísmica do Centro-Norte. \citet{dourado_2014}}
  \label{fig:z_mt} 
\end{figure}

A região sofreu um dos maiores sismos já registrados no Brasil, com magnitude pouco acima de 6.
Não há registros de falhas geológicas neo-tectônicas e mais complexa de ser explicada \citep{barros_2009}.

%% ------------------------------------------------------------------------- %%
\subsection{Extremo Oeste e Acre}
\index{Sismicidade do Brasil!extremo-oeste}
\label{sec:z_ac}

No extremo oeste do Brasil, na região do Acre, a sismicidade tem uma característica distinta das outras.
É possível reparar, na figura \ref{fig:z_ac}, primeiramente na quantidade de sismos, e em
seguida perceber a influência dos sismos sul americanos, desde os mais profundos aos intermediários
e relativamente rasos de 70km. 


\begin{figure}[H]
  \centering
  \includegraphics[width=0.5\textwidth]{z_ac} 
  \caption{Zona sísmica do Acre. \citet{dourado_2014}, Catálogo \gls*{iscgem}.}
  \label{fig:z_ac} 
\end{figure}

Note que a escala de profundidade da figura \ref{fig:z_ac} é diferente das demais que seguem a
mesma escala da figura \ref{fig:br_seis}.


%% ------------------------------------------------------------------------- %%
\subsection{Amazonas}
\index{Sismicidade do Brasil!Amazonas}
\label{sec:z_am}

É a região com menor quantidade de conhecimento e informação disponível.
Ainda assim, na figura \ref{fig:z_am} é possível observar a ocorrência de sismos
na direção norte-sul.

\begin{figure}[H]
  \centering
  \includegraphics[width=0.5\textwidth]{z_am} 
  \caption{Zona sísmica de Manaus. \citet{dourado_2014}.}
  \label{fig:z_am} 
\end{figure}

O registro de um sismo de magnitude 5 determinada por dados macrossísmicos,
é o indício mais marcante na região.


%% ------------------------------------------------------------------------- %%
\subsection{Nordeste}
\index{Sismicidade do Brasil!nordeste}
\label{sec:z_ne}

A regisão nordeste brasileira é sismicamente a mais ativa (figura \ref{fig:z_ne}). 

\begin{figure}[H]
  \centering
  \includegraphics[width=0.6\textwidth]{z_ne} 
  \caption{Zona sísmica do NE. \citet{dourado_2014}}
  \label{fig:z_ne} 
\end{figure}

Destaca-se a sismicidade na região de João Câmara no Rio Grande do Norte,
com um enxame sísmico em meados da década de 1980 \citep{takeya_1989, bezerra_1998}. 
Além disso são bem conhecidas as atividades sísmicas na região de Sobral-CE,
na região de Pernambuco e um pouco mais ao sul já na Bahia.


        	  % associado ao arquivo: 'cap-conclusoes.tex'
%% ------------------------------------------------------------------------- %%
\chapter{Contexto Teórico}
\label{cap:teoria}


%% ------------------------------------------------------------------------- %%
\section{Apresentação}
%\index{área do trabalho!fundamentos}
\label{sec:c04_apresentacao}

Este capítulo apresenta a formalização das teorias aplicadas na fase de
processamento. 

Trata-se essencialmente das \gls{smoothing} que, em geral, permitem extrair 
feições importantes do conjunto de dados.

Quando aplicadas à caracterização das \glspl{seismic_source} em \gls{psha}
tornam possível gerar um conjunto regular de \glspl{point_source} 
singularmente definidas pela suavização das \glspl{seismic_rate} 
nas células de uma malha regular.


%% ------------------------------------------------------------------------- %%
\section{\Gls{smoothing}}
\index{suavização!fundamentos, metodologia}
\label{sec:04_smoothing_general}

A ideia, no fundo, é estimar a distribuição espacial da taxa anual de sismicidade $R$
ou sua \gls{pdf}.
O método mais simples conhecido para essa estimativa seria o histograma.

\subsection{Histograma 2D: uma possível \glsdesc{pdf} para a taxa de sismicidade}

Numa malha regular a taxa anual de sismicidade em cada célula seria calculada 
contando, à partir de um catálogo com tempo de observação conhecido,
\begin{equation}
\scriptsize
	\ensuremath{
	\frac{\text{o número observado de sismos na célula}}
		 {\text{área/volume da célula} \times 
		  \text{número total de sismos observados}}
	/
	\text{tempo de observação em anos}
	}
\normalsize.
\label{eq:rate_count}
\end{equation}

Isso seria equivalente a preparar um histograma normalizado 
dos tremores em duas (ou três, considerando a profundidade) dimensões.

O que se busca, em geral, por essas técnicas é suavizar justamente essas contagens ou esse histograma
normalizado que representa uma estimativa da função de densidade de probabilidade da taxa de ocorrência espacial de tremores. 

%\newtheorem{prop}{Proposição}
%\begin{prop}

\subsection{Regressão e Suavizadores}


Para os $n$ pares (célula, taxa de sismicidade) $(x_1, R_1), (x_2, R_2), \cdots, (x_n, R_n)$
obtidos pela contagem anterior, considere um modelo para a taxa de sismicidade $R$ 
em uma determinada célula $x_i$ a partir dessa amostra dado por
\begin{equation}
	\ensuremath{
		R(x_i) = \lambda(x_i) + \epsilon(x_i),\;\;\; i=1,\dots,n
	}
\label{eq:rate_model}
\end{equation}

onde os $\epsilon_i$ são \gls{va} não-correlacionadas que representam os erros 
tais que $E(\epsilon_i \arrowvert X = x_i) = 0$ 
e a  $Var(\epsilon_i \arrowvert X = x_i) = \sigma^2(x_i)$. A função  
$\lambda(x_i) = E(R_i \arrowvert X = x_i)$ é uma função de regressão.
%\end{prop}

É possível definir um estimador ou suavizador linear $\hat{\lambda}$ para $\lambda$, 
se para todo $x \in \mathbb{R}$ existe uma sequência de pesos $w_1(x), w_2(x),\cdots,w_n(x)$ tais que
$\sum_{i=1}^{n}w_i(x) = 1$, como sendo

\begin{equation}
	\ensuremath{
		\hat{\lambda}(x) = \sum_{i=1}^{n}w_i(x)\,R_i.
	}
\label{eq:rate_estim}
\end{equation}

A questão passa a ser então como encontrar essa sequência de pesos $w_i$.

\subsection{Função de Núcleo e Estimadores de Nadaraya-Watson}
\label{sec:nadaraya}
Uma função de núcleo $K$ (\emph{Kernel}) é qualquer função par, contínua e limitada que satisfaz as seguintes
propriedades:

%\begin{enumerate}[(i)]
%	\item $\int \lvert K(\boldsymbol{u})\rvert \,\mathrm{d}\boldsymbol{u} < \infty $
%	\item $\underset{ \lvert\boldsymbol{u} \rvert \to \infty }{\lim} \lvert \boldsymbol{u} K(\boldsymbol{u})\rvert = 0$
%	\item $\int \! K(\boldsymbol{u}) \,\mathrm{d}\boldsymbol{u} = 1 $
%\end{enumerate}
\begin{center}
(i) $\int \lvert K(\boldsymbol{r})\rvert \,\mathrm{d}\boldsymbol{r} < \infty $
\;\;\;\;\;(ii) $\underset{ \lvert\boldsymbol{r} \rvert \to \infty }{\lim} 
				\lvert \boldsymbol{r} \, K(\boldsymbol{r})\rvert =0$ 
\;\;\;\;\;(iii) $\int \! K(\boldsymbol{r})	\,\mathrm{d}\boldsymbol{r} = 1 $.
\end{center}

Uma das possíveis maneiras de se encontrar os pesos $w_i$ é o 
Estimador de Nadaraya-Watson \citep{nadaraya_1965}. 

Seja $h \in \mathbb{R}, h > 0$ e $K$ uma função de núcleo. 
Nadaraya e Watson propõem, para a estimativa \eqref{eq:rate_estim}, os pesos
\begin{equation}
	\ensuremath{
		w_i(x) = \frac{ K\left( \frac{x - x_i}{h} \right)}
					  {\sum_{j=1}^{n} K\left( \frac{x - x_j}{h} \right) },
	}
\label{eq:rate_wi}
\end{equation}
onde $h$ é conhecida como largura de banda ou \emph{bandwidth}.



\subsection{Formas das funções de núcleo}

Dentre as possíveis expressões para as funções de núcleo é relevante destacar duas \citep{kagan_2000}. 

A equação \eqref{eq:kernel_gs} apresenta a expressão da função de núcleo gaussiana:

\begin{equation}
	\ensuremath{
		K_{gs}(\gls{sym:r}\arrowvert h) = \eta_1(h)
			e^{- \frac{\|\gls{sym:r}\|^2}
 				 	  {2 h^2 }},
 	}
\label{eq:kernel_gs}
\end{equation}
onde $h$ é a largura de banda definida para a função de núcleo e $\eta_1(h)$ um fator de normalização
para que sua integral seja igual à unidade.


Outra forma possível para a função de núcleo é uma Lei de Potência (\emph{power-law}), que decai com o 
inverso do cubo da distância, como na equação
\eqref{eq:kernel_pl} a seguir:

\begin{equation}
	\ensuremath{
		K_{pl}(\gls{sym:r}\arrowvert h) = 
			\frac{\eta_2(h)}
 				 {\left( \|\gls{sym:r}\|^2 + h^2 \right)^{\frac{3}{2}} },
 	}
\label{eq:kernel_pl}
\end{equation}
onde $h$ é a largura de banda definida para a função de núcleo e $\eta_2(h)$ um fator de normalização
para que sua integral também seja igual à unidade.


Quaisquer dessas duas funções podem ser usadas como função de núcleo 
para os estimadores de Nadaraya-Watson apresentados anteriormente.

\subsection{Contribuição de uma função de núcleo bidimensional}

É importante notar que há uma outra abordagem possível para a estimativa da taxa de sismicidade.

Se, em vez de suavizar o histograma do catálogo sísmico numa malha regular, a proposta for
avaliar a contribuição em probabilidade pela \gls{pdf} de cada tremor em uma determinada célula da malha, 
é possível considerar uma função de núcleo como a \gls{pdf} da ocorrência de cada tremor, e,
essa contribuição numa célula $j$, devido ao tremor $i$ em $\boldsymbol{r}_i$, dada pela integral da
função de núcleo na área/volume da célula \citep{zechar_2010-2}: 
\begin{equation}
	\ensuremath{
		R_{j}(\boldsymbol{r}_i \arrowvert h) = \int\limits_{\mathrm{cell}\;j}\!K( \gls{sym:r} -
		\gls{sym:ri}\,\arrowvert\,h)\,\mathrm{d}\gls{sym:r} },
\label{eq:kernel_int}
\end{equation}
onde $h$ é a largura de banda da suavização.


Em todos os casos apresentados, é fácil notar que, a determinação da largura de banda influencia diretamente
a estimativa da taxa de sismicidade. Os métodos estudados variam essencialmente 
na forma com que a escolha da largura de banda é concebida e proposta.



%% ------------------------------------------------------------------------- %%
\section{Frankel, 1995}
\index{Frankel, 1995}
\label{sec:frankel}

A proposta de Arthur \citet{frankel_1995} foi usar uma largura de banda fixa, 
nomeada distância de correlação $d_F$, e aplicar o estimador de Nadaraya-Watson (\ref{sec:nadaraya})
para suavizar o histograma 2D da sismicidade utilizando uma função de núcleo gaussiana:
\begin{equation}
	\ensuremath{
		\tilde{n}_j = \frac{ \sum_{i} n_i \,e^{ - \left(\frac{\gls{sym:dij}}{\gls{sym:dF}}\right)^2}}
						   { \sum_{i}     e^{ - \left(\frac{\gls{sym:dij}}{\gls{sym:dF}}\right)^2}},
	}
	\label{eq:ni}
\end{equation}
onde $\tilde{n}_j$ é a taxa de sismicidade (número de sismos com magnitude $m$ maior que a mínima magnitude
\gls{sym:Md} do catálogo) suavizada
na célula $j$, $n_i$ é o número de sismos em cada outra célula $i$ e
	\gls{sym:dij} é \glsdesc{sym:dij}.


\citet{zechar_2010-2} propuseram uma variação da abordagem de Frankel avaliando a contribuição
de cada tremor $i$ na célula $j$ em vez de suavizar o histograma 2D, mas para isso
é preciso avaliar essa contribuição como na equação \eqref{eq:kernel_int}.


%% ------------------------------------------------------------------------- %%
\section{Woo, 1996}
\index{Woo, 1996}
\label{sec:woo}

Já Gordon \citet{woo_1996} propôs avaliar a contribuição de uma função de núcleo 
de cada sismo $i$ ocorrido em $\boldsymbol{r}_i$ 
na célula centrada em $\boldsymbol{r}$ que dependa também de sua
magnitude $m$:
\begin{equation}
	\ensuremath{
		\gls{sym:Rrm} = \sum_{i=1}^{N} \frac{ K(\gls{sym:r} - \gls{sym:ri}, m)}
											{T({\gls{sym:ri})}},
	}
	\label{eq:Rrm}
\end{equation}
onde $N$ é o número de tremores $i$ no catálogo 
e $T(\gls{sym:ri})$ é o período em que todo sismo de magnitude acima de $m$ é completamente observado 
em \gls{sym:ri}.

A função de núcleo utilizada por Woo foi a proposta por \citet{kagan_knopoff_1980}
para um domínio espacial infinito:
\begin{equation}
	\ensuremath{
		K_{KJ}(\gls{sym:r}, m \arrowvert \gls{sym:aW}) =  \frac{  \gls{sym:aW}  -1}{\pi\gls{sym:hm}^2}
							\left( 1 + \frac{\gls{sym:r}^2}{\gls{sym:hm}^2} \right)^{-\gls{sym:aW}},
	}
	\label{eq:k_kj}
\end{equation}
onde \gls{sym:aW} é, segundo \citet{verejones_1992}, \glsdesc{sym:aW}.

Uma variante limitada para a função de núcleo foi proposta por \citet{verejones_1992}:\begin{equation}
	\ensuremath{
		K_{VJ}(\gls{sym:r}, m \arrowvert \gls{sym:DW}) = 
		\begin{cases}
			\frac{\gls{sym:DW}}{2\pi\,\gls{sym:hm}} 
			\left( \frac{\gls{sym:hm}}{\gls{sym:r}} \right)^{2 - \gls{sym:DW}} 
			  & \gls{sym:r} \leq \gls{sym:hm} \\
			0 & \gls{sym:r} > \gls{sym:hm}
		\end{cases},
	}
	\label{eq:k_vj}
\end{equation}
onde \gls{sym:DW} é \glsdesc{sym:DW}.

Para a largura de banda \gls{sym:hm}, nos dois casos, Gordon Woo propôs utilizar a relação
\begin{equation}
	\ensuremath{
		h(m\arrowvert \gls{sym:a0}, \gls{sym:a1}) = \gls{sym:a0}e^{\gls{sym:a1}m},
	}
	\label{eq:hm}
\end{equation}
em que $a_0$ e $a_1$ são determinados pela regressão entre a 
distância média $h$ de cada tremor ao vizinho mais próximo em cada faixa de magnitude $m \pm \mathrm{d}m$.
É uma largura de banda dependente da magnitude.


%% ------------------------------------------------------------------------- %%
\section{Helmstetter, 2012}
\index{hemlstetter, 2012}
\label{sec:helmstetter}

O trabalho de  Agnès Helmstetter e Maximilian Werner \citep{helmstetter_2012} é focado principalmente em 
projeção de longo-prazo para ocorrência de tremores/rupturas. 
O principal pressuposto dessa técnica (seção \ref{sec:projecao}) é que nesse período a taxa
de sismicidade se mantenha invariante, que também é o pressuposto da \gls{psha}.

\subsection{Taxa de sismicidade}
Para eles a taxa de sismicidade em uma localização
$\boldsymbol{r}$ e em um instante $t$ 
em função dos tempos $T_i$ e localizações $\mathbf{r}_i$ dos tremores $i$ em um catálogo 
poderia ser expresso por
\begin{equation}
	\ensuremath{\gls{sym:R} = \sum_{i=1}^{N}{ \frac{1}{h_i\,{d_i}^2} \gls{sym:Kt}\gls{sym:Kr} }},
	\label{eq:helms01}
\end{equation}
onde \gls{sym:R} é \glsdesc{sym:R}, 
	  $K_t$ é a \glsdesc{sym:Kt}, 
	  $K_r$ é a \glsdesc{sym:Kr}.

Como seu interesse é em projeção, restringiram os tempos do catálogo aos $t_i
< t$ e ponderaram segundo um peso $w$ da seguinte maneira:
\begin{equation}
\ensuremath{\gls{sym:R} = \gls{sym:Rmin} + \sum_{t_i < t}{ 
	\frac{2\,w(\boldsymbol{r}_i,t_i)}{h_i\,{d_i}^2}
			\gls{sym:Kt}\gls{sym:Kr} }},
	\label{eq:helms02}
\end{equation}
onde \gls{sym:Rmin} é a \glsdesc{sym:Rmin}, positiva, permitindo que ocorram eventos-surpresa 
onde não se tem registro de sismicidade.

Os pesos $w(\boldsymbol{r}_i,t_i)$, calculados em cada tremor $i$, propostos por Helmstetter são a transformação de
Gutemberg-Richter. Consideraram uma contribuição maior para a taxa de sismicidade daqueles sismos 
que ocorreram onde e quando a magnitude de completude $M_c$ era maior que 
a mínima magnitude $M_d$ do catálogo 
em locais onde a magnitude de completude e o \emph{valor-b} variam. 
Note que o modelo de pesos acomoda bem variações espaciais e
temporais tanto da magnitude de completude como do \emph{valor-b}. A expressão é a seguinte:
\begin{equation}
	\ensuremath{ w(\boldsymbol{r},t) = 10^{ \gls{sym:b}(\boldsymbol{r},t) \left[ \gls{sym:Mc_rt} - \gls{sym:Md}
	\right] } },
	\label{eq:helms_wi}
\end{equation}
onde  \gls{sym:wi} é o \glsdesc{sym:wi} na localização $\boldsymbol{r}$ e no instante $t$, 
	  \gls{sym:b}$(\boldsymbol{r},t)$ é o \glsdesc{sym:b}, 
	  \gls{sym:Mc_rt} é a \glsdesc{sym:Mc_rt}, 
	  \gls{sym:Md} é a \glsdesc{sym:Md}.

\subsection{Método acoplado dos vizinhos mais próximos}

Em sismologia, as funções de núcleo no espaço e no tempo são usualmente estimadas
pela distância entre cada tremor e seu $k^{simo}$ vizinho mais próximo. 

A definição de cada $h_i$ e $d_i$ é feita por otimização, 
pelo método acoplado do $k^{ésimo}$ vizinho mais próximo
proposto por \citet{choi_1999}
com uma modificação para a função de núcleo temporal assimétrica.

Para cada tremor $i$, as larguras de banda  $h_i$ e $d_i$ são escolhidas para
minimizar a soma $h_i + a_{cnn}\,d_i$ sob a condição de que hajam ao menos $k_{cnn}$
eventos a uma distância espacial menor que $d_i$ e, simultaneamente, estejam
no intervalo de tempo $[t_i - h_i, t_i[$. O parâmetro $k_{cnn}$ controla a suavização 
geral do modelo, e $a_{cnn}$ controla a importância relativa entre o tempo e o espaço.
Valores altos de $a_{cnn}$ apresentam alta resolução espacial e, ao mesmo tempo, 
são mais suaves no domínio do tempo. Em outras palavras:

\begin{equation}
	\ensuremath{
%		h_i, d_i = \underset{d_i \ge \gls{sym:dk}, h_i \ge \gls{sym:hk}}{\argmin} 
		h_i, d_i = \argmin_{\substack{h_i \ge \gls{sym:hk} \\
						              d_i \ge \gls{sym:dk}}
				           } 
		\left[ s \left(h_i,d_i 
			 		  \arrowvert
					  \gls{sym:k_cnn},\gls{sym:a_cnn}
			     \right) 
			   := h_i + \gls{sym:a_cnn}d_i 
	    \right],
	}
	\label{eq:helms_cnn}
\end{equation}
onde \gls{sym:k_cnn} é o \glsdesc{sym:k_cnn},
	 \gls{sym:a_cnn} é o \glsdesc{sym:a_cnn},
	 \gls{sym:dk} é o \glsdesc{sym:dk} e 
	 \gls{sym:hk} é o \glsdesc{sym:hk}.

As larguras de bandas são escolhidas localmente, podendo ser menor, aumentando a resolução 
onde há maior densidade de informação, ou crescer onde há escassez de informação e menor resolução.


\subsection{Taxa de sismicidade estacionária}
Mesmo havendo modelado a dependência com o tempo para distribuição da taxa de sismicidade,
a taxa estacionária $\bar{R}$ que se preserva no longo-prazo em uma determinada localização  
$\boldsymbol{r}_0$ é a que mais interessa à \gls{psha} e,
segundo o método proposto por \citet{helmstetter_2012}, será definida
pelo valor da mediana da taxa fornecida pelo modelo no
período considerado:

\begin{equation}
	\ensuremath{
		\bar{R}(\boldsymbol{r}_0) = \text{Mediana}\left[R(\boldsymbol{r}_0, t)\right].
	}
	\label{eq:helms_mediana}
\end{equation}

O fato de se tomar a mediana faz com que variações na taxa de sismicidade devido a pré e pós-eventos
não interfiram significativamente no valor da taxa estacionária. A decorrência mais importante desse fato
é que o método dispensa a remoção de agrupamentos (seção \ref{sec:declustering}) do catálogo, um método paramétrico e
quase sempre controverso.

\subsection{Verossimilhança}

Para otimizar os parâmetros do modelo de Helmstetter, é preciso dividir o catálogo de sismos.
Uma parte serve para o aprendizado do modelo e a outra para avaliação.

Se a ocorrência de tremores pode ser modelada por um processo de Poisson com taxa $N_p$, 
então a probabilidade de se observar exatamente $n$ eventos no período de tempo considerado é dada por

\begin{equation}
	\ensuremath{
		\gls{sym:pNn} = \frac{{N_p}^n e^{-N_p}}{n!}.
	}
	\label{eq:loglik}
\end{equation}

E o logarítmo, a ser maximizado, da verossimilhança entre 
o que o modelo predisse e o que foi observado 
pode ser escrito como

\begin{equation}
	\ensuremath{
		\gls{sym:L} = \sum_{i_x=1}^{N_x}\sum_{i_y=1}^{N_y}\log p\left[  \gls{sym:Np}, \gls{sym:nxy}  \right]
	}
	\label{eq:loglik}
\end{equation}
onde \gls{sym:Np} é \glsdesc{sym:Np},
	\gls{sym:nxy} é \glsdesc{sym:nxy}.

Os parâmetros $R_{min}$, $a_{cnn}$ e $k_{cnn}$ do modelo são otimizados 
pela maximização da verossimilhança $L$.

\subsection{Ganho}

Uma distribuição de Poisson espacialmente uniforme para a sismicidade com taxa $N_u$
teria sua verossimilhança $L_u$ expressa por
\begin{equation}
	\ensuremath{
		\gls{sym:Lu} = -N_t + 
		\sum_{i_x = 1}^{N_x}\sum_{i_y=1}^{N_y}
		\gls{sym:nxy}\log N_u - \log \left[ \gls{sym:nxy}! \right]
	}
	\label{eq:lu}
\end{equation}
onde $N_u = N_t/N_c$ com $N_t$ o número de sismos no catálogo-alvo e $N_c$ o número de células, 


O ganho de probabilidade $G$ por tremor predito no catálogo-alvo (de teste)
sobre o que seria predito por uma distribuição espacialmente uniforme 
foi definido por \citet{kagan_knopoff_1977} como
\begin{equation}
	\ensuremath{
		\gls{sym:G} = e^{ \frac{\gls{sym:L} - \gls{sym:Lu}}{\gls{sym:Nt}}   }.
	}
	\label{eq:gain}
\end{equation}

Quando se compara dois modelos frente o mesmo catálogo-alvo, com a mesma quantidade de tremores $N_t$,
a equação \eqref{eq:gain} pode ser simplificada, pois 
\begin{equation}
	\ensuremath{
	\begin{align}
		L - L_u & = \sum_{i_x = 1}^{N_x}\sum_{i_y=1}^{N_y}
				  \gls{sym:nxy}\log \left[ \frac{\gls{sym:Np}}{\gls{sym:Nu}} \right] \\
				& = \sum_{i = 1}^{N_t}\log \left[\frac{\gls{sym:Npi}}{\gls{sym:Nu}} \right]
	\end{align}}
	\label{eq:llu}
\end{equation}

e o ganho $G$ do modelo passa a ser o mesmo que média geométrica da taxa prevista pelo modelo sobre 
a taxa uniforme em todas as células:
\begin{equation}
	\ensuremath{
	\begin{align}
		G & = e^{\sum_{i = 1}^{N_t}
					\frac{\log \left[  \gls{sym:Npi} / \gls{sym:Nu}  \right]}
						 {\gls{sym:Nt}}
			  } \\
		  & = {\langle  \gls{sym:Npi} / \gls{sym:Nu}  \rangle}_{geom}
	\end{align}}
	\label{eq:G}
\end{equation}
onde $\langle\cdot\rangle_{geom}$ significa média geométrica.

\subsection{Testes}

\citet{helmstetter_2012} usam os testes propostos por \citet{schorlemmer_2007}
e \citet{rhoades_2011} para comparar diferentes modelos.

O teste-T avalia quando o ganho de informação de um modelo é significativamente diferente
de um outro.

Para isso as taxas preditas pelos modelos são re-normalizadas para que o valor da taxa de sismicidade
prevista pelos modelos seja igual à taxa observada.

Sendo \gls{sym:NAi} a \glsdesc{sym:NAi} e \gls{sym:NBi} a \glsdesc{sym:NBi}, o ganho de informação $I_{inf}$
por tremor se reduz à 
\begin{equation}
	\ensuremath{
		\gls{sym:I} = \frac{1}{\gls{sym:Nt}} 
					  \sum_{i = 1}^{N_t}\log\left[ \frac{\gls{sym:NAi}}
					  								  {\gls{sym:NBi}}  \right].
	}
	\label{eq:gain_info}
\end{equation}

O ganho de informação $I_{inf}$ se relaciona com o ganho dos modelos (eq. \eqref{eq:gain}) por
\begin{equation}
	\ensuremath{
		I_{inf} = \log\left(\frac{G_A}{G_B}\right).
	}
	\label{eq:gain_info_G}
\end{equation}


A variância da amostra de Rhoades apresentada por Helmstetter é 

\begin{equation}
	\ensuremath{
		\sigma^2 = 	\frac{1}{N_t - 1}
					{\left(
					\sum_{i-1}^{N_t}
						{x_i}^2 
					\right)}
					- 
					\frac{1}{{N_t}^2 - N_t}
					{\left(
						\sum_{i=1}^{N_t}{x_i}
					\right)}^2,
	}
	\label{eq:var}
\end{equation}
onde $x_i = \log\left[ \gls{sym:NAi} / \gls{sym:NBi}  \right]$.

Seja o valor $T_s$ também definido por Rhoades e apresentado por Helmstetter

\begin{equation}
	\ensuremath{
		\gls{sym:Ts} = \frac{I_{inf}\sqrt{N_t}}{\sigma}.
	}
	\label{eq:T}
\end{equation}

Se $x_i$ é independente e tem distribuição normal,
então $T_s$ deve ter distribuição de Student com $N_t - 1$ graus de liberdade. Para grandes valores de $N_t$
a distribuição converge rapidamente para a normal.

O valor do ganho de informação é considerado significante se $T_s > 2$ no intervalo de 95\% de confiança.

O teste de Wilcoxon é usado no caso de que $x_i$ não tenha distribuição normal
mas permanece simétrica e independente \citep{rhoades_2011}.
Esse teste avalia o quanto a mediana de $x_i$ difere significativamente de zero.
Isso é o mesmo que avaliar o quanto a taxa prevista por um modelo é significativamente diferente da prevista por 
outro modelo.

A probabilidade $p_W$ de se observar um valor maior que a mediana de $x_i$ é retornada pelo teste.
Valores de $p_W < 0.05$ indicam que a mediana de $x_i$ é significativamente diferente de 0.


       % associado ao arquivo: 'cap-introducao.tex'
%% ------------------------------------------------------------------------- %%
\chapter{Processamento}
\label{cap:processamento}

Esse capítulo apresenta como as teorias do capítulo anterior foram utilizadas para obtenção 
dos resultados que serão discutidos no próximo capítulo.

%% ------------------------------------------------------------------------- %%
\section{Conjunto de Dados}
\index{Conjunto de Dados}
\label{sec:dados}

Os dados utilizados foram os dados do catálogo \gls{iscgem} \citep{storchak_2013} 
para a América do Sul e 
os dados do \glsdesc{bsb2013} \citep{bsb_2013} para o Brasil.

O dados são texto e formatados em arquivos de \gls{csv}.

A figura \ref{fig:eq_record} ilustra o número de registros de tremores por ano nos dois catálogos.

\begin{figure}[H]
	\centering
	\begin{subfigure}[b]{0.48\textwidth}
		  	\centering
			\includegraphics[width=1.00\textwidth]{hmtk_sa3_rate}
			\subcaption{Número de tremores registrados por ano, \gls{iscgem}}
			\label{fig:sa_eq_record}
	\end{subfigure}%
	\quad %~ %add desired spacing between images, e. g. ~, \quad, \qquad, \hfill etc.
	\begin{subfigure}[b]{0.48\textwidth}
		  	\centering
			\includegraphics[width=1.00\textwidth]{hmtk_bsb2013_rate}
			\subcaption{Número de tremores registrados por ano, \gls{bsb2013}}
			\label{fig:br_eq_record}
    \end{subfigure}%
	\caption{Número de tremores registrados por ano após 1900}
	\label{fig:eq_record}
\end{figure}

Em ambos é possível notar um incremento do número de registros de sismos com o passar dos anos, 
além de algumas variações bem destacadas em períodos reduzidos de poucos anos.



%% ------------------------------------------------------------------------- %%
\subsection{Catálogo ISC-GEM}
\index{Conjunto de Dados!ISC-GEM}
\label{sec:data_source}

O \glsdesc{iscgem} versão 1.04 possui licença CC-BY-SA e é fruto da redeterminação de parâmetros
dos terremotos com os dados disponíveis no ISC-GEM e de um detalhado estudo para que ao menos um valor de
magnitude $\gls{sym:Mw}$ estivesse disponível com incertezas.

Possui pouco mais de 110 000 registros.  

\subsection{Boletim Sísmico Brasileiro}
\index{Conjunto de Dados!BSB}
\label{sec:data_source2}

O \glsdesc{bsb} \citep{bsb_2013}, versão 2013.08 possui licença CC-BY e é fruto do esforço de compilação de
dados e determinação de epicentros e magnitudes que contou com a colaboração de várias instituições
como o \gls{obsis} da \gls{unb}, o \gls{ipt}, a \gls{unesp}, a \gls{ufrn} liderados pelo \gls{iag} da \gls{usp}.


Possui algo em torno de 900 registros.

%% ------------------------------------------------------------------------- %%

\section{Ferramentas}
\index{Ferramentas}
\label{sec:ferramentas}

As ferramentas utilizadas para as análises e cálculos apresentados nesse texto
foram de código aberto.
Utilizou-se o Latex para a edição do texto, o software de processamento de dados
geoespacial QGIS, o console do Linux BASH, a IDE Eclipse, o git/GitHub para o controle de versao,

O iPython como console interativo, a MatplotLib e Basemap para o gráficos, 
a SciPy para um conjunto de funções científicas e estatísticas, a shapely e gdal
para lidar com objetos com atributos geométricos.


%% ------------------------------------------------------------------------- %%
\subsection{Linguagens de Programação para o Cálculo da Taxa de Sismicidade}
\index{Linguagens de Programação}
\label{sec:linguagens}

A principal linguagem de programação utilizada foi Python.

Os códigos-fonte dos modelos de Woo e de Helmstetter foram providos pelos autores
e estão em Fortran. O método de Frankel estava disponível em Python.


%% ------------------------------------------------------------------------- %%
\subsection{Programas de Computador para Cálculo de Ameaça Sísmica}
\index{Ferramentas!programas}\index{software}
\label{sec:software}

O programa mais comumente utilizado para análise de ameaça sísmica é o CRISIS \citep{crisis_2007} nas suas
mais diversas versões. Como nesse trabalho a opção foi por código aberto, poderia ter se optado pelo 
\gls{opensha}\footnote{\url{http://www.opensha.org}} que é escrito em Java.
A intenção no entanto foi provar o conjunto de tecnologias mantido pela fundação \gls{gem}
para o cálculo de ameaça e rísco sísmico que é o \gls{oq} \citep{pagani_2014}.

O código do \gls{oq} está abertamente disponível para consulta e clonagem no GitHub
\footnote{\url{https://github.com/GEM}} e seu \emph{ecossistema} é ilustrado na figura a seguir:

\begin{figure}[!h]
  \centering
  \includegraphics[width=.80\textwidth]{oq_ecosystem} 
  \caption{Ecossistema de módulos, bibliotecas e utilitários do OpenQuake}
  \label{fig:oq} 
\end{figure}

O destaque em vermelho na figura \ref{fig:oq} fica para o \gls{oqe} que executa as simulações de cada tronco da
arvore-lógica e as usa para calcular o cenário de risco e armazenar o resultado.
Faz isso distribuindo o processamento em tarefas e permitindo escalar o cálculo.

A \gls{oqp} permite interagir com os dados armazenados e gerenciados na nuvem, como catálogos, 
falhas, dados de esforços, fontes sismogênicas e suas geometrias usados em cálculos de ameaça/risco.
Permite também interagir com o resultado do cálculo, visualizando mapas de ameaça e curvas de intensidade.

%% ------------------------------------------------------------------------- %%
\subsection{Bibliotecas de Funções}
\index{ferramentas!bibliotecas}
\label{sec:bibliotecas}

Em amarelo na figura \ref{fig:oq} está a \gls{hl} que contém toda lógica e ciência para o cálculo de ameaça,
como os tipos de fontes sísmogênicas, as \gls{mfd}, as \glspl{gmpe}, etc. 

Está também a \gls{rl}
que contém os modelos de vulnerabilidade e exposição para a análise de risco. 

E por último,
a \gls{nrml} que é a sintaxe da linguagem de representação das árvores-lógicas, fontes sísmicas e
resultados, como mapas de risco, espectro de alguma medida de intensidade, etc. 


%% ------------------------------------------------------------------------- %%
\subsection{Implementações e Novos Códigos}
\index{Implementações e novos códigos }
\label{sec:implementacao}

Incluído no suporte da fundação \gls{gem} está a manutenção de um Comitê Científico
que desenvolve ferramentas auxiliares adicionais como conversores, ferramentas para trabalho com 
o catálogo, ferramentas gráficas, entre outras para interação com o
\gls{oq} como pode ser visto na figura \ref{fig:oq} em azul, cinza e bege.

Uma dessas ferramentas em especial é o \gls{hmtk} \citep{weatherill_2012, weatherill_2014-1} que facilita todo o
processo de modelagem da \gls{psha} como a remoção de agrupamentos, a caracterização de zonas sísmicas,  
a visualização da evolução da taxa de sismicidade, a análise da magnitude de completude e
estimativa do \emph{valor-b}.

O \gls{hmtk} já trazia um módulo para trabalhar com a sismicidade que usava as \gls{smoothing}
implementando o método de \citet{frankel_1995}. E é na \gls{hmtk} que se pretende
contribuir com a implementação dos métodos de \citet{woo_1996} e de \citet{helmstetter_2012}.
 
%% ------------------------------------------------------------------------- %%
\section{Pré-Processamento}
\index{pré-processamento}
\label{sec:pre_processamento}

Para aplicar as \gls{smoothing} no conjunto de dados é necessário alguns primeiros procedimendos.

%% ------------------------------------------------------------------------- %%
\subsection{Controle de Qualidade}
\index{pre-processamento!controle de qualidade}
\label{sec:qualicontrol}

A primeira coisa a fazer no conjunto de dados é uma checagem geral.

Nessa hora é preciso observar se não há pontos com coordenadas erradas, invertidas, faltando valores de dias ou horas.

É recomendado também fazer uma varredura em busca de vieses no catálogo \citep{van_stiphout_2010}.
Isso é feito com, por exemplo, um histograma do dia da semana da ocorrência dos tremores em busca
de algum descréscimo em fins de semana que seriam um provável indicativo de contaminação do catálogo 
por atividade humana, por exemplo explosões em pedreiras. 
Também é possível fazer um histograma para observar a
distribuição do horário de ocorrência durante o dia, 
ou mesmo da profundidade para estimar a resolução do catálogo
nesse quesito. A figura \ref{fig:qc_histograms} apresenta alguns dos histogramas sugeridos para a análise
exploratória dos dados dos catálogos.

\begin{figure}[H]
	\centering
	\begin{subfigure}[b]{0.45\textwidth}
		  	\centering
			\includegraphics[width=1.00\textwidth]{hmtk_sa3_weekday}
			\subcaption{Distribuição dos tremores nos dias da semana, \gls{iscgem}}
			\label{fig:sa_week_hist}
	\end{subfigure}%
	\quad %~ %add desired spacing between images, e. g. ~, \quad, \qquad, \hfill etc.
	\begin{subfigure}[b]{0.45\textwidth}
		  	\centering
			\includegraphics[width=1.00\textwidth]{hmtk_bsb2013_weekday}
			\subcaption{Distribuição dos tremores nos dias da semana, \gls{bsb2013}}
			\label{fig:br_week_hist}
    \end{subfigure}%
        %~ %add desired spacing between images, e. g. ~, \quad, \qquad, \hfill etc.
          %(or a blank line to force the subfigure onto a new line)
          
 	\begin{subfigure}[b]{0.45\textwidth}
		  	\centering
			\includegraphics[width=1.00\textwidth]{hmtk_sa3_hour}
			\subcaption{Distribuição do horário (GMT) de ocorrência dos tremores, \gls{iscgem}}
			\label{fig:sa_hour_hist}
	\end{subfigure}%
	\quad %~ %add desired spacing between images, e. g. ~, \quad, \qquad, \hfill etc.
	\begin{subfigure}[b]{0.45\textwidth}
		  	\centering
			\includegraphics[width=1.00\textwidth]{hmtk_bsb2013_hour}
			\subcaption{Distribuição do horário (GMT) de ocorrência dos tremores, \gls{bsb2013}}
			\label{fig:br_hour_hist}
    \end{subfigure}%
        %~ %add desired spacing between images, e. g. ~, \quad, \qquad, \hfill etc.
          %(or a blank line to force the subfigure onto a new line)

	\begin{subfigure}[b]{0.45\textwidth}
		  	\centering
			\includegraphics[width=1.00\textwidth]{dep_sa_hist}
			\subcaption{Distribuição da profundidade dos tremores, \gls{iscgem}}
			\label{fig:sa_dep_hist}
	\end{subfigure}%
	\quad %~ %add desired spacing between images, e. g. ~, \quad, \qquad, \hfill etc.
	\begin{subfigure}[b]{0.4\textwidth}
		  	\centering
			\includegraphics[width=1.00\textwidth]{dep_br_hist}
			\subcaption{Distribuição da profundidade dos tremores, \gls{bsb2013}}
			\label{fig:br_dep_hist}
        \end{subfigure}%
        %~ %add desired spacing between images, e. g. ~, \quad, \qquad, \hfill etc.

  \caption{Checagem de qualidade.}
  \label{fig:qc_histograms} 
\end{figure}

Nas figuras \ref{fig:sa_week_hist} e \ref{fig:br_week_hist} não se observam grandes vieses aparentes 
em relação aos dias da semana, nem aos horários de ocorrência durante o dia 
(figuras \ref{fig:sa_week_hist} e \ref{fig:br_week_hist}) nos dois catálogos.

No que diz respeito à distribuição de profundidade, o catálogo do ISC tem as
profundidades mais distribuidas, mesmo com grande maioria sendo eventos rasos 
também se observa um número razoável de sismos a mais de 600km de profundidade.
O BSB por sua vez é composto praticamente todo de sismos rasos, com menos de 10km
de profundidade, ou na prática, com profundidade desconhecida.



%% ------------------------------------------------------------------------- %%
\subsection{Conversão de Magnitudes}
\index{magnitudes!conversão}
\label{sec:mag_conv}

Para que a magnitude dos tremores possam ser analizadas em termos do momento sísmico e do 
tamanho da ruptura que geram é preciso que os sismos do catálogo apresentem pelo menos um valor
\gls{sym:MW} para a magnitude.

No catálogo do ISC-GEM isso é assunto resolvido pois as magnitudes \gls{sym:MW} já foram computadas \citep{storchak_2013}, mas no
caso do \gls{bsb} a maior parte das magnitudes são regionais $m_R$ que se assemelham às magnitudes $m_b$ 
ou magnitudes estimadas a partir de dados macrossísmicos. Nos dois casos é preciso avaliar funções que relacionem os
vários valores de magnitude e \gls{sym:MW}.

Para o catálogo \gls{bsb} as relações conhecidas\footnote{por Assumpção,M. e Drouet, S. (não-publicado)} para serem
usadas são as de equivalência entre $m_R$ e $m_b$:
\begin{equation}
	\ensuremath{
		m_R \equiv m_b,
	}
\label{eq:mrmb}
\end{equation}
as de conversão de $m_b$ em $\gls{sym:MW}$
\begin{equation}
	\ensuremath{
		M_W(m_b) = 1.12 m_b - 0.76,
	}
\label{eq:mwmb}
\end{equation}
as de conversão entre $A_f$ e $M_W$
\begin{equation}
	\ensuremath{
		M_W(A_f) = 0.6 + 0.8\log A_f,
	}
\label{eq:mwaf}
\end{equation}
onde \gls{sym:Af} é \glsdesc{sym:Af}; e por fim as estimativas de $m_b$ a partir da \glsdesc{sym:I0} (\gls{sym:I0})
presentes no catálogo.

NOTA: Essa etapa de pré-processamento, por simplicidade, não foi feita no escopo desse trabalho mas é aqui apresentada 
como referência para trabalhos futuros.


%% ------------------------------------------------------------------------- %%
\subsection{Remoção de agrupamentos}
\index{remoção de agrupamentos}\index{declustering}
\label{sec:declustering}

A maior parte dos modelos de sismicidade estudados assumem que a ocorrência de
tremores segue um processo de Poisson.

Mas sabe-se que os predecessores e sucessores de tremores principais não são independentes.
O aumento do número de sismos pouco antes e pouco depois de um grande tremor contamina
o catálogo com mais sismos do que seriam realmente esperados se os tremores fossem realmente independentes.
Por exemplo, logo após um sismo de magnitude 8, geralmente ocorrem durante um curto 
espaço de tempo mais alguns sismos de magnitude próximo a 7 e cerca de dezenas de sismos 
de magnitude menores que 5, que não ocorreriam com essa mesma frequência na ausência do sismo
principal, de magnitude 8.

Existem outros modelos que usam exatamente essa variação das taxas de sismicidade com o tempo
para fazerem projeções de curto-prazo, mas que não estão no escopo desse trabalho.

Os métodos mais difundidos para a remoção desses agrupamentos não-Poissonianos são métodos que consistem em definir
janelas de tempo e espaço em função da magnitude \citep{gardner_1974}, dentro das quais, os sismos são considerados como pertencentes ao mesmo
agrupamento. 

A seguir estão três formulações implementadas no HMTK para a remoção de agrupamentos.
A primeira de \citet{gardner_1974} é 
\begin{equation}\begin{split} 
\mbox{d(m)} = &10^{0.1238 m + 0.983}\\
\mbox{t(m)} = & 
\begin{cases} 10^{0.032 m + 2.7389} & \text{se $m \geq 6.5$} \\ 
              10^{0.5409 m - 0.547} & \mbox{caso contrário}  \end{cases}\end{split}
\end{equation}

Uma formulação alternativa foi proposta por Gr\"unthal \citep{marsan_david_2012} :

\begin{equation}\begin{split} 
\mbox{d(m)} = & e^{1.77 + \left( {0.037 + 1.02 m} \right)^2} \\ 
   \mbox{t(m)} = & \begin{cases}   |e^{-3.95+ \left( {0.62 + 17.32 m}
    \right)^2}|    & \text{se $m \geq 6.5$ } \\ 10^{2.8 + 0.024 m} & 
    \text{caso contrário}  \end{cases}\end{split}
\end{equation}

E outra sugerida por \citet{uhrhammer_1986}
%
\begin{equation}
\mbox{d(m)} = e^{-1.024 + 0.804 m} \quad \mbox{t(m)} = 
    e^{-2.87 + 1.235 m}
\end{equation}

Enquanto os métodos acima definem explicitamente a janela de tempo, \citet{musson_1999}, propôs que as janelas de tempo
fossem janelas móveis em vez de fixas. Sua teoria é consistente com a de \citet{gardner_1974}.

A figura \ref{fig:eq_decluster_cum} apresenta os resultados de distintos métodos e janelas aplicados aos 
catálogos.

\begin{figure}[H]
	\centering
	\begin{subfigure}[b]{0.48\textwidth}
		  	\centering
			\includegraphics[width=1.00\textwidth]{decluster_sa}
			\subcaption{Número cumulativo de tremores registrados por ano para o \gls{iscgem}
			original e para diferentes métodos/janelas de remoção de agrupamentos.}
			\label{fig:sa_eq_record}
	\end{subfigure}%
	\quad %~ %add desired spacing between images, e. g. ~, \quad, \qquad, \hfill etc.
	\begin{subfigure}[b]{0.48\textwidth}
		  	\centering
			\includegraphics[width=1.00\textwidth]{decluster_br}
			\subcaption{Número cumulativo de tremores registrados por ano para o \gls{bsb2013}
			original e para diferentes métodos/janelas de remoção de agrupamentos.}
			\label{fig:br_eq_record}
    \end{subfigure}%
	\caption{Número cumulativo de tremores registrados por ano após 1900. A magnitude usada foi a do catálogo. 
	As diferentes respostas vêm das diferentes janelas utilizadas por cada método.}
	\label{fig:eq_decluster_cum}
\end{figure}

No caso do \gls{bsb2013} os resultados são relativamente equivalentes quando comparados aos resultados no caso do 
\glsdesc{iscgem}, onde é possível notar claramente as diferenças nos resultados de cada método de remoção de
agrupamento. Isso demonstra que uma discussão mais aprofundada sobre os métodos de 
remoção de agrupamentos é necessária mas está além dos propósitos desse texto.

A figura \ref{fig:eq_decluster} ilustra os agrupamentos de sismos não-Poissonianos encontrados 
pelo método de AFTERAN \citep{musson_2000} e janelas de Gr\"uenthal quando aplicados
aos catálogos \gls{iscgem} e \gls{bsb2013}. As sequencia de cores indica o número de cada agrupamento, e sua
distribuição espacial não apresenta qualquer padrão. 

\begin{figure}[H]
	\centering
	\begin{subfigure}[t]{0.46\textwidth}
		  	\centering
			\includegraphics[width=1.00\textwidth]{hmtk_sa3_pp_decluster}
			\subcaption{Número de tremores registrados por ano, \gls{iscgem}}
			\label{fig:sa_decluster}
	\end{subfigure}%
	\quad %~ %add desired spacing between images, e. g. ~, \quad, \qquad, \hfill etc.
	\begin{subfigure}[t]{0.50\textwidth}
		  	\centering
			\includegraphics[width=1.00\textwidth]{hmtk_bsb2013_pp_decluster}
			\subcaption{Número de tremores registrados por ano, \gls{bsb2013}}
			\label{fig:br_decluster}
    \end{subfigure}%
	\caption{Número de tremores registrados por ano após 1900}
	\label{fig:eq_decluster}
\end{figure}

Os catálogos com os agrupamentos removidos passam a ser os catálogos utilizados nas etapas posteriores.

%% ------------------------------------------------------------------------- %%
\subsection{Análise da Magnitude de Completude}
\index{Magnitude de Completude}
\label{sec:completeness}

Uma evidência da necessidade de se avaliar a magnitude de completude \gls{sym:Mc}
pode ser percebida ao contar a quantidade de sismos registrados ao longo do tempo
para cada intervalo de magnitude. A figura \ref{fig:qc_time_mag_count} apresenta
esses gráficos para os catálogos considerados.

\begin{figure}[H]
	  \centering
	  \begin{subfigure}[b]{0.7\textwidth}
		  	\centering
			\includegraphics[width=1.00\textwidth]{time_mag_count_sa}
			\caption{Catálogo \gls{iscgem} 1900-2012}
			\label{fig:tmf_sa}
        \end{subfigure}%
        %~ %add desired spacing between images, e. g. ~, \quad, \qquad, \hfill etc.
          %(or a blank line to force the subfigure onto a new line)

	  \begin{subfigure}[b]{0.7\textwidth}
		  	\centering
  			\includegraphics[width=1.00\textwidth]{time_mag_count_br}
			\caption{Catálogo \gls{bsb2013} 1900-2012}
			\label{fig:tmf_br}
       \end{subfigure}%

	   \begin{subfigure}[b]{0.7\textwidth}
		  	\centering
  			\includegraphics[width=1.00\textwidth]{time_mag_count_br_1960}
			\caption{Catálogo \gls{bsb2013} no período de 1960-2012}
			\label{fig:tmf_br_1960}
       \end{subfigure}%

  \caption{Contagem de sismos em tempo e em magnitude.}
  \label{fig:qc_time_mag_count} 
\end{figure}

Na figura \ref{fig:qc_time_mag_count} é possível perceber, nos dois catálogos,
que a quantidade de sismos no mesmo intervalo de magnitude varia bastante ao longo do tempo.
Isso se deve a uma série de fatores, mas principalmente à capacidade de registro
da rede sismográfica, implantação de redes globais e 
à modernização e aumento de sensibilidade de equipamentos.

A magnitude de completude é importante \citep{woessner_2005} por sua influência direta no cálculo dos parâmetros $a$ e
$b$ de Gutemberg-Richter.
Ela pode variar ao longo do tempo e do espaço (variação da densidade de estações sismográficas).


\begin{p}
\textbf{Stepp Test}
\end{p}

O método de \citet{stepp_1971} permite obter a variação temporal da magnitude
de completude \gls{sym:Mc} diretamente a partir do catálogo com um teste simples.

Esse procedimento analítico foi um dos primeiros a serem propostos para esse fim.
Sejam $\lambda_1, \lambda_2, \cdots, \lambda_n$ o número de sismos por unidade de tempo.
Assumindo que os tremores de certa faixa de magnitude têm distribuição de Poisson, a estimativa da taxa de
média de sismicidade $\lambda$ por unidade de tempo é expressa por

\begin{equation}
	\ensuremath{
		\lambda =  \frac{1}{n}\sum_{i=1}^{n} \lambda_i.
	}
\label{eq:mwaf}
\end{equation}
Sua variância é $\sigma_{\lambda}^2 =  \lambda/n$, onde $n$ o é número de intervalos.

Com intervalos de 1 ano, $n$ intervalos é o período $T$ de observação com desvio padrão
\begin{equation}
	\ensuremath{
		 \sigma_{\lambda} = \frac{\sqrt{\lambda}}{\sqrt{T}}.
	}
\label{eq:mwaf}
\end{equation}

A taxa de sismicidade sendo estacionária, seu desvio padrão deve se comportar como $1/\sqrt{T}$ em 
tempos de observação crescentes $T$.

Os diagramas abaixo tomam intervalos de observação crescentes com passo de 2.5 anos,
em intervalos de 0.5 de magnitude.
\begin{figure}[H]
	\centering
	\begin{subfigure}[b]{0.47\textwidth}
		  	\centering
			\includegraphics[width=1.00\textwidth]{stepp_sa}
			\subcaption{Diagrama de Stepp para o \gls{iscgem} (\emph{declustered})}
			\label{fig:sa_stepp}
	\end{subfigure}%
	\quad %~ %add desired spacing between images, e. g. ~, \quad, \qquad, \hfill etc.
	\begin{subfigure}[b]{0.47\textwidth}
		  	\centering
			\includegraphics[width=1.00\textwidth]{stepp_br}
			\subcaption{Diagrama de Stepp para o \gls{bsb2013} (\emph{declustered})}
			\label{fig:br_stepp}
    \end{subfigure}%
	\caption{Diagrama de Stepp para análise da magnitude de completude \gls{sym:Mc}}
	\label{fig:eq_stepp}
\end{figure}

Com esses parâmetros, as magnitudes de completude \gls{sym:Mc}, determinadas pelo método
de Stepp, estão listadas nas tabelas \ref{tab:mc_sa} e \ref{tab:mc_br}.

	\begin{table}[h]
	  	\centering
		\begin{tabular}{l|*{11}{c}}
		$M_c$ & 3.0  & 3.5  & 4.0  & 4.5  & 5.0  & 5.5  & 6.0  & 6.5  & 7.0  & 7.5  & 8 \\  \hline
		Ano   & 1986 & 1986 & 1986 & 1960 & 1958 & 1958 & 1927 & 1898 & 1885 & 1885 & 1885   \\
		\end{tabular}
		\caption{Magnitude de completude, \gls{iscgem}}
		\label{tab:mc_sa}
	\end{table}

	\begin{table}[h]
	  	\centering
		\begin{tabular}{l|*{7}{c}}
		$M_c$ & 3.0  & 3.5  & 4.0  & 4.5  & 5.0  & 5.5  & 6.0  \\  \hline
		Ano   & 1980 & 1975 & 1975 & 1965 & 1965 & 1860 & 1860 \\
		\end{tabular}
		\caption{Magnitude de completude, Cat.BSB2013}
		\label{tab:mc_br}
	\end{table}

Os parâmetros a serem utilizados podem ser controversos, inclusive a opção de forçar com que 
os anos de completude de magnitudes maiores sejam ao menos iguais aos de magnitude menores, 
mas novamente, a discussão sobre os melhores parâmetros para o método de Stepp, sua validade, ou 
mesmo outros métodos estão além da proposta desse texto.

Os valores de \gls{sym:Mc} das tabelas \ref{tab:mc_sa} e \ref{tab:mc_br} são os valores utilizados
no cálculo da recorrência no resto do trabalho.


%% ------------------------------------------------------------------------- %%
\section{Frankel, 1995}
\index{Frankel, 1995!processamento}
\label{sec:proc_frankel}

Tanto no método de Frankel como nos demais métodos avaliados,
a malha para a estimativa de taxa de sismicidade para o Brasil 
foi de $1^o \times 1^o$, o \emph{valor-b} foi fixado em 1.

Para o processamento pelo método de Frankel, utilizou-se o catálogo com os agrupamentos removidos,
e a tabela de magnitude de completude mencionada. A distância de correlação utilizada apenas como exemplo
foi de 150km a 68\% (totalizando uma influência total da ordem de 500km). A zona de influência foi limitada em 3 vezes a
distância de correlação.

%% ------------------------------------------------------------------------- %%
\section{Woo, 1996}
\index{Woo, 1996!processamento}
\label{sec:proc_woo}

O processamento pelo método de Woo também foi feito com a remoção dos agrupamentos no catálogo.

A largura de banda dependente da magnitude é ajustada pelo próprio método.
O ajuste da função de largura de banda é apresentado na figura~\ref{fig:woo_b} para o \gls{bsb2013}:

\begin{figure}[H]
  \centering
  \includegraphics[width=.80\textwidth]{woo_bandwidth} 
  \caption{Ajuste da largura de banda para o método de Woo1996.
  	Os valores de 1.38 e 1.18 para $a_0$ e $a_1$ foram obtidos por mínimos quadrados.}
  \label{fig:woo_b} 
\end{figure}

Apenas como comparação estão apresentados também os ajustes citados por \citet{beauval_2003} para a Espanha e Noruega.
Como é possível observar o ajuste brasileiro apresenta distâncias médias sistematicamente maiores, provavelmente
devido à escala continental do Brasil ou ao baixo volume de registros.

A função de núcleo utilizada foi a proposta pela equação \eqref{eq:k_kj}.

%% ------------------------------------------------------------------------- %%
\section{Helmstetter, 2012}
\index{Helmstetter!processamento}
\label{sec:proc_helmstetter}

Utilizou-se, para a projeção da taxa de sismicidade, como catálogo de teste
o período 1950-2007. Os sismos de 2007-2012 juntamente
com os sismos ocorridos antes de 1950 foram colocados no catálogo-alvo.

A escolha por colocar os sismos anteriores a 1950 no catálogo se deve ao baixo número
de sismos no Brasil e de pouca capacidade histórica de observação, de forma que esses eventos trazem informações
importantes sobre essas fontes sismogênicas que não aparecem no catálogo no periodo escolhido para a aprendizagem.

Para a aprendizagem foram utilizados sismos com magnitudes acima de 3.8 buscando projeções para
sismos com magnitudes acima de 4.0. O pressuposto é de que sismos com magnitudes grandes
ocorrem onde geralmente ocorrem sismos de magnitude inferior.

A figura \ref{fig:h_catalogue} apresenta os catálogos de treino e de teste usados na otimização.

\begin{figure}[H]
  \centering
  \includegraphics[width=.80\textwidth]{helmstetter_catalogues} 
  \caption{Catálogos de aprendizado e de teste para o método de \citet{helmstetter_2012}}
  \label{fig:h_catalogue} 
\end{figure}

As magnitudes de completude, embora haja espaço no modelo para sua representação espacial e temporal
e, teoricamente, pudessem ser calculadas para o instante e posição de cada sismo para que fossem
considerados com os pesos adequados, não foram consideradas dessa forma. Em vez disso a magnitude
de completude utilizada foi a mínima magnitude no catálogo de teste ($M_c = M_d = 3.8$) 
uniformemente para todo o catálogo.

Os cálculos da largura de banda $h_i$ e $d_i$ de cada sismo é ilustrado no caso de um evento particular na figura
\ref{fig:h_hidi}.

\begin{figure}[H]
  \centering
  \includegraphics[width=.80\textwidth]{helmstetter_hidi} 
  \caption{Exemplo da largura de banda para um determinado evento para o método de Helmstetter, com $k_{cnn} = 5$ e
  $a_{cnn} = 100$ arbitrários, para um determinado tremor $i$. A linha pontilhada vermelha marca a distância temporal
  $h_i$ e a linha pontilhada azul representa a distância espacial $d_i$}
  \label{fig:h_hidi} 
\end{figure}

Nota-se que as larguras de banda são escolhidas de forma a deixar $k_{cnn}$ eventos entre $h_i$ e $d_i$.
O mais importante é notar que como $k_{cnn} = 5$ existem exatamente 5 outros eventos (bolinhas) no retângulo
definido pelo estremo inferior esquerdo (0,0) e o superior direito ($h_i$,$d_i$). Fica evidente
que os valores determinados para $h_i$ e $d_i$ são os que, preservando ao menos 5 eventos, minimiza a soma
($h_i + a_{cnn}d_i$) mas que seriam outros valores caso a $a_{cnn}$ apontasse para outra proporção espaço-temporal.

Em seguida, na figura \ref{fig:hidi_hist} são apresentadas as distribuições de $h_i$ e $d_i$ calculadas
para os parâmetros já otimizados do modelo.

\begin{figure}[H]
	\centering
	\begin{subfigure}[b]{0.47\textwidth}
		  	\centering
			\includegraphics[width=1.00\textwidth]{hi_histogram}
			\subcaption{Distribuição das larguras de banda temporais}
			\label{fig:hi_hist}
	\end{subfigure}%
	\quad %~ %add desired spacing between images, e. g. ~, \quad, \qquad, \hfill etc.
	\begin{subfigure}[b]{0.47\textwidth}
		  	\centering
			\includegraphics[width=1.00\textwidth]{di_histogram}
			\subcaption{Distribuição das larguras de banda espaciais}
			\label{fig:di_hist}
    \end{subfigure}%
	\caption{Distribuição das larguras de banda calculadas com os parâmetros otimizados do modelo}
	\label{fig:hidi_hist}
\end{figure}


A taxa estacionária em cada célula é calculada a partir da mediana da taxa prevista pelo modelo em cada célula,
como na figura \ref{fig:h_stationary}.
\begin{figure}[H]
  \centering
  \includegraphics[width=.80\textwidth]{helmstetter_stationary_a} 
  \caption{Taxa de sismicidade estacionaria calculada a partir da mediana da taxa de sismicidade
  modelada pelo método de Helmstetter para uma determinada célula $r_0$}
  \label{fig:h_stationary} 
\end{figure}
A célula em questão fica próxima à região de Mato Grosso e o pico de sismicidade em torno de 1955,
ocorre justamente na época do grande tremor da região.

A função de núcleo utilizada na dimensão do espaço foi a gaussiana \eqref{eq:kernel_gs} multivariada, 
e univariada na dimensão do tempo.

%% ------------------------------------------------------------------------- %%
\section{Pós-Processamento}
\index{pós-processamento}
\label{sec:pos_proc}

Como o objetivo desse trabalho é a caracterização das fontes sísmogênicas,
o cálculo da ameaça pelo método clássico propriamente dito
acaba figurando como pós-processamento.

A figura \ref{fig:classical_psha} apresenta o fluxograma de trabalho para uma análise clássica de \gls{psha} conforme
implementado no Openquake \citep{pagani_2010, weatherill_2012}.

\begin{figure}[H]
	\centering
	\begin{tabular}{l}
	\includegraphics[width=0.70\textwidth]{classical_psha_workflow}
	\end{tabular}
	\caption{Fluxo de trabalho clássico para a \gls{psha} \citep{crowley_2013}.}
\label{fig:classical_psha}
\end{figure}
     % associado ao arquivo: 'cap-conceitos.tex'
%% ------------------------------------------------------------------------- %%
\chapter{Resultados}
\label{cap:resultados}

Os resultados obtidos através do processamento descrito no capítulo anterior
serão enumerados a seguir.


%% ------------------------------------------------------------------------- %%
\section{Resultados Anteriores}
\index{resultados anteriores}
\label{sec:old_results}

Antes dos resultados obtidos pelos modelos discutidos, 
serão apresentados alguns já conhecidos que poderão servir como referência ou ponto de partida.

%% ------------------------------------------------------------------------- %%
\subsection{GSHAP}
\index{GSHAP}
\label{sec:gshap}

Um dos resultados mais conhecidos de avaliação de ameaça sísmica global é o GSHAP \citep{giardini_1999}.

A figura \ref{fig:gshap} apresenta os resultados do GSHAP para América do Sul.
E escala está limitada para que sejam claras as variações no território brasileiro.

\begin{figure}[H]
  \centering
  \includegraphics[width=.80\textwidth]{pga_gshap} 
  \caption{Resultado do GSHAP para o Brasil: \gls{PGA} (10\%/50anos) em unidades de $g$}
  \label{fig:gshap} 
\end{figure}

É possível observar no GSHAP a grande influência de sismos provenientes da zona de subdução nos Andes 
sob o território do Acre. Também é destacada a sismicidade no Nordeste, com um pico de 0.16g para a PGA, e por
último, chama a atenção o resto do país por haver um vazio de informação.


%% ------------------------------------------------------------------------- %%
\subsection{Zoneamento Sísmico}
\index{zoneamento}
\label{sec:zonning}

Existem também esforços \citep{dourado_2014} para se criar um modelo para a ameaça sísmica brasileira
usando a metodologia clássica de Cornell \& McGuire com zoneamento sísmico. 

\citet{dourado_2014}, aplicou critérios de especialista para definir e caracterizar as zonas sísmicas
apresentadas na figura \ref{fig:a_dourado}.

\begin{figure}[H]
  \centering
  \includegraphics[width=.80\textwidth]{a_dourado} 
  \caption{Zoneamento sísmico e caracterização das zonas sísmicas por \citep{dourado_2014}.
  Os valores para a magnitude mínima foram de 3.0, e os \emph{valores-a} correspondem a magnitude maior que 0.}
  \label{fig:a_dourado} 
\end{figure}

As zonas sísmicas foram apresentadas em detalhe de sismicidade e geologia no capítulo \ref{cap:regiao_de_estudo}.

O cálculo feito por Dourado com o programa Crisis \citep{crisis_2007} gerou os resultados da figura \ref{fig:pga_dourado}
para o mapa de ameaça sísmica (PGA, poe 10\% em 50 anos). Os valores originais \emp{Gal} $[cm/s^2]$ foram convertidos
para unidades de \emph{g} $[m/s^2]$.

\begin{figure}[H]
  \centering
  \includegraphics[width=.80\textwidth]{pga_dourado_crisis} 
  \caption{Resultado de \citet{dourado_2014} para o Brasil: PGA (10\%/50anos) em unidades de $g$ calculadas com o programa
  Crisis-2007}
  \label{fig:pga_dourado} 
\end{figure}

As mesmas fontes sísmicas e suas respectivas características foram utilizadas para o cálculo no
Openquake e os resultados estão apresentados na figura \ref{fig:pga_dourado_oq}.

\begin{figure}[H]
  \centering
  \includegraphics[width=.80\textwidth]{pga_dourado_oq} 
  \caption{Mapa de ameaça sísmica, PGA(poe 0.1, 50y)[Dourado, 20014] OpenQuake-Engine }
  \label{fig:pga_dourado_oq} 
\end{figure}

Os resultados preservam as feições mas apresentam quantidade significativamente diferentes.
Provavelmente devido ao fator de escala entre a magnitude e a área de ruptura utilizada,
que sobrevalora a área de ruptura para magnitudes pequenas (menores que 5). Uma sugestão
para o futuro é separar, na definição das fonte, a \gls{mfd} em duas partes: na porção inferior
usar um fator de escala que considere toda a ruptura ocorrendo no hipocentro. Na porção superior,
a partir de uma certa magnitude, toma-se a ruptura tendo como geometria mais realista.


%% ------------------------------------------------------------------------- %%
\section{Suavização da Sismicidade}
\index{suavização da sismicidade}
\label{sec:suavizacao_resultados}

Os métodos avaliados resultaram nos mapas de 
\emph{valor-a} e de ameaça sísmica
que serão apresentados nessa seção.

%% ------------------------------------------------------------------------- %%
\subsection{Frankel, 1995}
\index{Frankel,1995!resultados}
\label{sec:frankel_resultados}

Como resultado do processamento apresentado na seção \ref{sec:proc_frankel},
com distância de correlação de 150km e utilizando-se do catálogo BSB,
a taxa de sismicidade suavizada gerada pelo modelo pode ser vista na figura \ref{fig:a_fran_br}.
\begin{figure}[H]
  \centering
  \includegraphics[width=.80\textwidth]{a_frankel_br} 
  \caption{Mapa do valor-a $[\log\left($\#tremores$/$ano$)/$área$\right)]$, usando o catálogo \gls{bsb2013} calculado
  pelo método de Frankel, 1995 }
  \label{fig:a_fran_br} 
\end{figure}

Apenas para comparação, o mesmo método foi empregado com o catálogo \gls{iscgem}, como pode ser visto na figura 
\ref{fig:a_fran_sa}.

\begin{figure}[H]
  \centering
  \includegraphics[width=.80\textwidth]{a_frankel_sa} 
  \caption{Mapa do valor-a, catálogo \gls{iscgem} calculado pelo método de Frankel, 1995}
  \label{fig:a_fran_sa} 
\end{figure}

Nos dois casos é possível perceber a compatibilidade entre os resultados na maior parte do Brasil, mas com
diferenças significativas na região do Acre, Rondônia e na porção norte na Dorsal Atlântica. 
Note que pela limitação da escala, os valores em vermelho são quaisquer valores acima de 2.5

Os valores da ameaça calculados conforme a seção \ref{sec:hazard} estão na figura \ref{fig:pga_fran}.

\begin{figure}[H]
  \centering
  \includegraphics[width=.80\textwidth]{pga_frankel} 
  \caption{Mapa de ameaça sísmica, PGA (poe 10\%, 50y) [Frankel, 1995] }
  \label{fig:pga_fran} 
\end{figure}

Como técnica de suavização, é importante notar os principais altos de ameaça no nordeste e no centro do país.
A ameaça é recuperada razoávelmente na plataforma continental.

%% ------------------------------------------------------------------------- %%
\subsection{Woo, 1996}
\index{Woo, 1996!resultados}
\label{sec:woo_resultados}

O ajuste dos parâmetros da função de largura de banda resultou em $h(m)=1.39e^{1.18\,m}$km.

Aplicando-se o método de Woo ao catálogo brasileiro desagrupado, a taxa de sismicidade suavizada 
pode ser avaliada pela figura \ref{fig:a_woo}.

\begin{figure}[H]
  \centering
  \includegraphics[width=.80\textwidth]{a_woo} 
  \caption{Mapa do valor-a, usando o catálogo \gls{bsb2013} calculado pelo método de Woo, 1996 }
  \label{fig:a_woo} 
\end{figure}

A sismicidade de regiões como o nordeste, Goiás/Tocantins, sudeste e plataforma continental 
foram bem recuperadas, enquanto partes da região norte e centro-oeste não tiveram tanto destaque. 

O cálculo da ameaça (figura \ref{fig:pga_woo_inc}) foi feito com o resultado direto do método, utilizando
diretamente uma \gls{mfd} discreta.
\begin{figure}[H]
  \centering
  \includegraphics[width=.80\textwidth]{pga_woo_inc} 
  \caption{Mapa de ameaça sísmica, PGA (poe 10\%, 50y) 
  		   calculado com o Openquake a partir das fontes sísmicas
  		   determinas pelo método de Woo, 1996, usando uma \gls{mfd}
  		   discreta e incremental.
  }
  \label{fig:pga_woo_inc} 
\end{figure}

Coerentemente com a taxa de sismicidade modelada. Não houve local em que a taxa
da figura \ref{fig:a_woo} fosse baixa e que tenha apresentado grande ameaça.
As regiões norte, oeste e sul do país quase não figuraram 
enquanto parte do sudeste, plataforma continental, Goiás e nordeste foram
destacadas. Merece ser notado o maior valor de ameaça localizado em Pernambuco, dentre as três principais regiões mais ameaçadas do nordeste.

Para quantificar as diferenças nos cálculos de ameaça em decorrência da \gls{mfd} utilizada, em seguida é apresentado o
mapa de ameaça (figura \ref{fig:pga_woo_cum}) calculado a partir das taxas geradas pelo modelo, mas utilizando-se do
acumulado nos valores de magnitude acima do mínimo do catálogo e aplicando a transformação de Gutemberg-Richter 
para obter o \emph{valor-a}. Nesse procedimento, a partir de certa magnitude minima de referencia, toma-se o cumulativo
das taxas para magnitudes acima da referência, e, se projeta (por \gls{GR}) o \emph{valor-a}.

Essa projeção acumula incertezas que não são inerentes ao método. A proposta de Woo é fornecer
diretamente a distribuição de magnitudes em cada ponto da malha enquanto essa projeção é feita
assumindo um \emph{valor-b} arbitrário.


\begin{figure}[H]
	\centering
	\begin{subfigure}[t]{0.47\textwidth}
		\centering
		\includegraphics[width=1.0\textwidth]{pga_woo_cum} 
		\subcaption{Mapa de ameaça sísmica, PGA (poe 10\%, 50y), 
  		   calculado com o Openquake a partir das fontes sísmicas
  		   determinas pelo método de Woo, 1996, usando uma \gls{mfd}
  		   truncada usando o valor-a como o valor cumulativo
  		   contado a partir da \gls{sym:m_min}.
		}
		\label{fig:pga_woo_cum} 
	\end{subfigure}
	\quad
	\begin{subfigure}[t]{0.47\textwidth}
		\centering
		\includegraphics[width=1.0\textwidth]{pga_woo_dif} 
		\subcaption{Mapa diferencial de ameaça, PGA (poe 10\%, 50y), 
		   entre os modelos usando \gls{mfd} truncada e \gls{mfd}
		   discreta. Diferença entre os mapas das figuras 
		   \ref{fig:pga_woo_inc} e \ref{fig:pga_woo_cum}.
		   }
		\label{fig:pga_woo_dif} 
	\end{subfigure}
	\caption{Variação do resultado da ameaça em função do uso de diferentes 
			\glspl{mfd} no OpenQuake.}
	\label{fig:pga_woo} 
\end{figure}

A figura \ref{fig:pga_woo_dif} quantifica a diferença nos valores de ameaça obtitos quando 
utilizadas as diferentes distribuições de frequência e magnitude. Nota-se que enquanto
o modelo com \gls{mfd} incremental (método original) destaca principalmente o nordeste, o modelo utilizando o
\emph{valor-a} cumulativo (projetado) e uma \gls{mfd} truncada, destacou principalmente as ameaças no centro-oeste e
sudeste.


%% ------------------------------------------------------------------------- %%
\subsection{Helmstetter, 2012}
\index{Helmstetter, 2012!resultados}
\label{sec:helmstetter_resultados}

A otimização dos parâmetros do modelo de Helmstetter conforme descrito na seção \ref{sec:proc_helmstetter}
resultou, com ganho\footnote{equação \eqref{eq:gain}} $G = 2.43$ sobre uma distribuição uniforme, nos valores
apresentados na tabela \ref{tab:hemlstetter}.

\begin{table}[H]
	\centering
	\begin{tabular}{c|c}
		Parâmetro & Valor \\ \hline
		$R_{min}$ & $0.1\times10^{-13}$ \\
		$a_{cnn}$ & 325 \\
		$k_{cnn}$ & 1 \\
	\end{tabular}
	\caption{Parâmetros otimizados (método simplex) para o modelo de Helmstetter a partir do catálogo \gls{bsb2013}.}
	\label{tab:hemlstetter}
\end{table}

Usando os parâmetros otimizados, os valores calculados para a taxa de sismicidade
de longo-prazo pelo método proposto por Helmstetter pode ser visto na figura \ref{fig:helm_r}.

\begin{figure}[H]
  \centering
  \includegraphics[width=.80\textwidth]{a_helmstetter} 
  \caption{Mapa do valor-a, usando o catálogo \gls{bsb2013} calculado pelo método de Helmstetter, 2012.}
  \label{fig:helm_r} 
\end{figure}

Observa-se que a sismicidade de regiões como amazônia, Mato-Grosso e Pará foram destacadas,
enquanto no sudeste houve maior diluição. A sismicidade do nordeste aparece evidente, principalmente
nas regiões do Ceará, Rio Grande do Norte e Pernambuco. 

É fato também que a sismicidade do centro-oeste rumo ao norte pelo estado de Goiás praticamente desapareceu.
Isso se deve provavelmente ao fato de que grande parte dos sismos dessa região possuem magnitudes não 
muito elevadas, tendo sido em grande parte removidas pela magnitude de completude como se pode notar nos catálogos
de aprendizado e testes na figura \ref{fig:h_catalogues}.

Na figura~\ref{fig:helm_h} estão os valores calculados para o mapa de ameaça 
com as taxas de sismicidade suavizadas pelo método.

\begin{figure}[H]
  \centering
  \includegraphics[width=.80\textwidth]{pga_helmstetter} 
  \caption{Mapa de ameaça sísmica, PGA (poe 10\%, 50y), 
  		   calculado com o OpenQuake a partir das fontes sísmicas
  		   determinas pelo método de Helmstetter, 2012.}
  \label{fig:helm_h} 
\end{figure}

Nota-se que ao passo em que houve um grande realçe da ameaça em algumas regiões como amazônia e Pará (próximo a Belém),
em regiões como a faixa sísmica no norte de Goiás e Tocantins não apresentaram o mesmo contraste.


        % associado ao arquivo: 'cap-conclusoes.tex'
%% ------------------------------------------------------------------------- %%
\chapter{Discussão}
\label{cap:conclusoes}

Apesar de possuirem implementações distintas de um mesmo fundamento matemático e estatítstico
que é a estimativa de densidade por funções de núcleo, os métodos, no geral, conseguiram
estimar das características da sismicidade a partir da sismicidade histórica catalogada.

Alguns distribuiram mais a taxa de sismicidade e a ameaça, uns destacando certas feições, outros outras.
Pode-se perceber que a caracterização da distribuição de frequência e magnitude, o uso da sua forma discreta 
ou truncada, pode ter influência no valor calculado para a ameaça e merece atenção.

Não parece haver um modelo mais correto, mesmo
porque todas as formulações são coerentes ao que se propuseram. Mesmo assim
os resultados, considerando suas diferentes proposições variam bastante.
O modelo de projeção de ocorrência de rupturas de Helmstetter talvez nem
fosse aplicável num universo de poucas amostras distribuídas em uma área tão vasta.

Essa caracterização suavizada da taxa de sismicidade, mesmo que executada de forma
distinta por critérios distintos, certamente terá relevãncia e deverá ser considerada
de alguma forma pelos sismólogos ou engenheiros modeladores de ameaça sísmica,
mesmo que apenas para cumprir certo percentual de uma árvore lógica.

Ficou claro também, mesmo não tendo sido o foco específico desse trabalho, 
não ser possível negligenciar a ocorrência de sismos profundos da zona de 
subducção para a ameaça sísmica do extremo oeste do país. Mesmo os sismos muito profundos (da ordem de centenas de km)
não provocando grandes impactos nas estruturas, sismos maiores e um pouco mais distantes podem ter razoáveis amplitudes em determinados perídos do espectro de aceleração.

O openquake como calculador de ameaças foi positivo do ponto de vista metodológico,
e o material bem documentado serviu de apoio e esclarecimento do cálculo.

O resultado discrepante entre os valores apresentados pelo Crisis-2007 e o Openquake,
precisam ser investigados com maior detalhe, mas os resultados do openquake foram compatíveis com o
modelo global, mesmo os valores calculados com o mesmo modelo de fontes sísmicas.

O HMTK se mostrou uma ferramenta essencial para a modelagem da ameaça. 
Muitas funcionalidades permitem facilmente a implementação dos principais fluxos de trabalho.

Por fim, talvez seja possível dizer que os objetivos propostos foram cumpridos.

%------------------------------------------------------
\section{Considerações Finais} 

Foi possível de certa forma aplicar as técnicas de suavização, estimadoras das taxas suavizadas de sismicidade, 
gerar uma grade regular de fontes sísmicas pontuais, e então calcular a ameaça sísmica obtendo 
valores razoáveis, sem com isso definir zonas sísmicas, é o que se conhece também como métodos de \emph{zoneless}.

As diferenças nas formas de escolher a largura de banda das funções de núcleo de cada método,
podem talvez render ao método de Hemlstetter alguma vantagem, por ser localmente adaptável,
quando define claramente feições no nordeste. as outros métodos, com escolhas mais rígidas,
também o fizeram, privilegiando outras regiões.

O ajuste do método de Woo para o Brasil forneceu enormes larguras de banda, com cerca de 1500km para
as maiores magnitudes (são poucas e o método se baseia em vizinhos mais próximos), isso dilui
a influência dos grandes sismos nas taxas de sismicidade.

No caso do Brasil, essas primeiras estimativas sugerem que estudos de maior detalhe 
sejam feitos nas regiões de maior destaque.



%------------------------------------------------------
\section{Sugestões para Pesquisas Futuras} 

Ainda há muito o que explorar.

A variação temporal e espacial da magnitude de completude no Brasil seria a principal fonte de informação 
para melhoria significativa da resposta dos métodos que dependem dessas correções para darem bons resultados.
Seu conhecimento permite também um mapeamento dos \emph{valores-b}. \citet{vorobieva_2013} aponta caminhos muito
interessantes.

Consideração de modelos que levem em consideração a distribuição da perda de tensão na forma de momento sísmico,
e mesmo a suavização da distribuição de momento acumulado espacialmente pela placa também podem ser outra alternativa.

Na modelagem da ameaça sísmica, um passo importante, seria trabalhar daqui por diante também com uma metodologia
para seleção das relações de atenuação.

Nas regiões de maior sismicidade, talvez seja possível estudar modelos onde a taxa de sismicidade varia com o tempo
como os modelos de sequência epidêmica de pós-abalos, que podem dar melhor resposta com pouco volume de dados.

Coletar incertezas e elaborar um cenário de ameaça,
detalhando a árvore lógica de possibilidades e variações
tanto nos modelos de fonte sísmica, incluíndo eventos históricos característicos, 
como nos modelos de atenuação. 

Estudos de desagregação \citep{pagani_2007} também precisarão ser feitos no futuro.

        % associado ao arquivo: 'cap-conclusoes.tex'

% ============================================================================ %


% cabeçalho para os apêndices
\renewcommand{\chaptermark}[1]{\markboth{\MakeUppercase{\appendixname\ \thechapter}} {\MakeUppercase{#1}} }
\fancyhead[RE,LO]{}
\appendix

% ============================================================================ %
%\include{a01-conjuntos}      % associado ao arquivo: 'ape-conjuntos.tex'
% ============================================================================ %


% ---------------------------------------------------------------------------- %
% Bibliografia
\backmatter \singlespacing   				% espaçamento simples
\bibliographystyle{styles/plainnat-ime} 	% citação bibliográfica textual
\bibliography{bib/bibliografia}  			% associado ao arquivo: 'bibliografia.bib'


% ---------------------------------------------------------------------------- %
% Índice remissivo
%\index{TBP|see{periodicidade região codificante}}
%\index{DSP|see{processamento digital de sinais}}
%\index{STFT|see{transformada de Fourier de tempo reduzido}}
%\index{DFT|see{transformada discreta de Fourier}}
%\index{Fourier!transformada|see{transformada de Fourier}}

% Glossário
\printglossary[title=Gloss{á}rio,toctitle=Gloss{á}rio]

% índice remissivo no documento 
\printindex

\end{document}
