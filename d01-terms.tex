%-----------------------------------------
% terms
%-----------------------------------------
\newglossaryentry{equake}
{
	name={earthquake},
	description={rupture in some geological strutcture},
	plural={earthquakes}
}

\newglossaryentry{seismicity}
{
	name={seismicity},
	description={earthquake occurrence},
}

\newglossaryentry{hypocenter}
{
	name={hypocenter},
	description={geometrical representation of the starting point of the \gls{rupture_process} on the \gls{crust}},
	plural={hypocenters}
}

\newglossaryentry{epicenter}
{
	name={epicenter},
	description={orthogonal projecton over Earth surface of \gls{hypocenter}},
	plural={epicentros}
}


\newglossaryentry{seismotectonic}
{
	name={seismotectonic},
	description={o estudo das relações entre os \glspl{equake} e a \gls{tectonic} recente de uma região.
				 Procura compreender exatamente quais são os mecanismos que levam à uma ruptura geológica 
				 e são responsáveis pela
				 \gls{seismic_activity} em uma certa área. Isso é feito analisando-se de forma combinada 
				 registros recentes de tectonismo global e regional, 
				 considerando também evidências históricas e geomorfológicas},
}


\newglossaryentry{rupture_process}
{
	name={processo de ruptura},
	description={processo que envolve o rompimento de uma região da crosta,
			o deslocamento relativo entre essas regiões, e consequantemente,
			a liberação de uma grande quantidade de energia, de forma praticamente
			instantânea, tomando-se como referência o \gls{geologic_time}},
	plural={processos de ruptura}
}


\newglossaryentry{geologic_time}
{
	name={tempo geológico},
	description={escala de tempo que vai desde a formação do universo até os tempos atuais,
				englobando a formação do planeta e as transformações ocorridas desde então},
}


\newglossaryentry{tectonic}
{
	name={tectonic},
	description={disciplina científica focada nos processos respons\'aveis 
				 pela cria\c{c}\~ao e transforma\c{c}\~ao das estruturas geológicas da Terra e de outros planetas.},
	plural={tectonics}
}


%\newglossaryentry{oq}
%{
%	name={OpenQuake},
%	description={programa de código aberto para o calculo de risco sísmico mantido pela Fundação \gls{gem}},
%	plural={OpenQuake}
%}


\newglossaryentry{crust}
{
	name={crosta terrestre},
	description={parte superficial, rígida e mais externa do planeta Terra},
}

\newglossaryentry{mantle}
{
	name={manto terrestre},
	description={material da por{ç}{ã}o intermediária do planeta, 
		fluido em tempo geológico},
}

\newglossaryentry{core}
{
	name={n{ú}cleo terrestre},
	description={por{ç}{ã}o mais central do planeta, com predomin{â}ncia de compostos metálicos},
}

\newglossaryentry{tectonic_plate_theory}
{
	name={teoria tect{ô}nica das placas},
	description={foi uma teoria revolucionária para a \gls{tectonic},
				propondo que a \gls{crust} terrestre estivesse dividida 
				em placas {à} deriva sobre o \gls{mantle}},
}


\newglossaryentry{litho_plate}
{
	name={placa litosf{é}rica},
	plural={placas litosf{é}ricas},
	description={placa de material da \gls{lithosphere}},
}


\newglossaryentry{lithosphere}
{
	name={litosfera},
	description={região rúptil, mais externa do planeta, formada pela \gls{crust} 
		(continental e ocêanica) e parte do \gls{mantle} superior, com aproximadamente 
		60\gls*{sym:km} de profundidade},
}


\newglossaryentry{astenosphere}
{
	name={astenosfera},
	description={região dúctil entre a \gls{lithosphere} e o \gls{mantle},
				com profundidades que variam de 60 a 700km},
}

\newglossaryentry{smoothing}
{
	name={smoothing techniques},
	description={consiste em capturar importantes feições do conjunto de dados,
				 eliminando ruídos e outras estruturas de curto comprimento de onda
				 presentes nos dados},
}

\newglossaryentry{kernel_function}
{
	name={kernel function},
	description={n-dimentional function, which integral over whole domain is equal one and could be used as a probability density estimation.},
	plural={kernel functions},
}

\newglossaryentry{seismic_rate}
{
	name={seismic rate},
	description={rate within earthquakes are generated in some \gls{seismic_source}},
	plural={seismic rates},
}

\newglossaryentry{seismic_activity}
{
	name={seismic activity},
	description={frequency of \glspl{equake} occurence},
}

\newglossaryentry{poisson_process}
{
	name={processo de Poisson},
	description={uma sequencia de intervalos discretos com um experimento de Bernoulli em cada},
}

\newglossaryentry{seismic_source}
{
	name={seismic source},
	description={geological structure able to produce \glspl{equake}},
	plural={seismic sources}
}

\newglossaryentry{point_source}
{
	name={point seismic source},
	description={geometrical representation as point of some seismogenic source},
	plural={point seismic sources},
}

\newglossaryentry{gmpe}
{
	name={GMPE},
	description={ground motion prediction equation},
	plural={GMPEs},
}


\newglossaryentry{area_source}
{
	name={area seismic source},
	description={representação geométrica por um polígono em superfície, 
				 de uma fonte sísmica},
	plural={farea seismic sources},
}

\newglossaryentry{titulo_da_dissertacao}
{
	name={titulo_da_dissertacao},
	description={Técnicas de suavização aplicadas
					à caracterização de fontes sísmicas e 
					à análise probabilística de ameaça sísmica},
}

\newglossaryentry{isocista}
{
	name={isocist},
	description={border of a region with the same seismic intensity from the same \gls{equake}},
}

