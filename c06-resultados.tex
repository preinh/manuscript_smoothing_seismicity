%% ------------------------------------------------------------------------- %%
\chapter{Resultados}
\label{cap:resultados}

Os resultados obtidos através do processamento descrito no capítulo anterior
serão enumerados a seguir.


%% ------------------------------------------------------------------------- %%
\section{Resultados Anteriores}
\index{resultados anteriores}
\label{sec:old_results}

Antes dos resultados obtidos pelos modelos discutidos, 
serão apresentados alguns já conhecidos que poderão servir como referência ou ponto de partida.

%% ------------------------------------------------------------------------- %%
\subsection{GSHAP}
\index{GSHAP}
\label{sec:gshap}

Um dos resultados mais conhecidos de avaliação de ameaça sísmica global é o GSHAP \citep{giardini_1999}.

A figura \ref{fig:gshap} apresenta os resultados do GSHAP para América do Sul.
E escala está limitada para que sejam claras as variações no território brasileiro.

\begin{figure}[H]
  \centering
  \includegraphics[width=.80\textwidth]{pga_gshap} 
  \caption{Resultado do GSHAP para o Brasil: \gls{PGA} (10\%/50anos) em unidades de $g$}
  \label{fig:gshap} 
\end{figure}

É possível observar no GSHAP a grande influência de sismos provenientes da zona de subdução nos Andes 
sob o território do Acre. Também é destacada a sismicidade no Nordeste, com um pico de 0.16g para a PGA, e por
último, chama a atenção o resto do país por haver um vazio de informação.


%% ------------------------------------------------------------------------- %%
\subsection{Zoneamento Sísmico}
\index{zoneamento}
\label{sec:zonning}

Existem também esforços \citep{dourado_2014} para se criar um modelo para a ameaça sísmica brasileira
usando a metodologia clássica de Cornell \& McGuire com zoneamento sísmico. 

\citet{dourado_2014}, aplicou critérios de especialista para definir e caracterizar as zonas sísmicas
apresentadas na figura \ref{fig:a_dourado}.

\begin{figure}[H]
  \centering
  \includegraphics[width=.80\textwidth]{a_dourado} 
  \caption{Zoneamento sísmico e caracterização das zonas sísmicas por \citep{dourado_2014}.
  Os valores para a magnitude mínima foram de 3.0, e os \emph{valores-a} correspondem a magnitude maior que 0.}
  \label{fig:a_dourado} 
\end{figure}

As zonas sísmicas foram apresentadas em detalhe de sismicidade e geologia no capítulo \ref{cap:regiao_de_estudo}.

O cálculo feito por Dourado com o programa Crisis \citep{crisis_2007} gerou os resultados da figura \ref{fig:pga_dourado}
para o mapa de ameaça sísmica (PGA, poe 10\% em 50 anos). Os valores originais \emp{Gal} $[cm/s^2]$ foram convertidos
para unidades de \emph{g} $[m/s^2]$.

\begin{figure}[H]
  \centering
  \includegraphics[width=.80\textwidth]{pga_dourado_crisis} 
  \caption{Resultado de \citet{dourado_2014} para o Brasil: PGA (10\%/50anos) em unidades de $g$ calculadas com o programa
  Crisis-2007}
  \label{fig:pga_dourado} 
\end{figure}

As mesmas fontes sísmicas e suas respectivas características foram utilizadas para o cálculo no
Openquake e os resultados estão apresentados na figura \ref{fig:pga_dourado_oq}.

\begin{figure}[H]
  \centering
  \includegraphics[width=.80\textwidth]{pga_dourado_oq} 
  \caption{Mapa de ameaça sísmica, PGA(poe 0.1, 50y)[Dourado, 20014] OpenQuake-Engine }
  \label{fig:pga_dourado_oq} 
\end{figure}

Os resultados preservam as feições mas apresentam quantidade significativamente diferentes.
Provavelmente devido ao fator de escala entre a magnitude e a área de ruptura utilizada,
que sobrevalora a área de ruptura para magnitudes pequenas (menores que 5). Uma sugestão
para o futuro é separar, na definição das fonte, a \gls{mfd} em duas partes: na porção inferior
usar um fator de escala que considere toda a ruptura ocorrendo no hipocentro. Na porção superior,
a partir de uma certa magnitude, toma-se a ruptura tendo como geometria mais realista.


%% ------------------------------------------------------------------------- %%
\section{Suavização da Sismicidade}
\index{suavização da sismicidade}
\label{sec:suavizacao_resultados}

Os métodos avaliados resultaram nos mapas de 
\emph{valor-a} e de ameaça sísmica
que serão apresentados nessa seção.

%% ------------------------------------------------------------------------- %%
\subsection{Frankel, 1995}
\index{Frankel,1995!resultados}
\label{sec:frankel_resultados}

Como resultado do processamento apresentado na seção \ref{sec:proc_frankel},
com distância de correlação de 150km e utilizando-se do catálogo BSB,
a taxa de sismicidade suavizada gerada pelo modelo pode ser vista na figura \ref{fig:a_fran_br}.
\begin{figure}[H]
  \centering
  \includegraphics[width=.80\textwidth]{a_frankel_br} 
  \caption{Mapa do valor-a $[\log\left($\#tremores$/$ano$)/$área$\right)]$, usando o catálogo \gls{bsb2013} calculado
  pelo método de Frankel, 1995 }
  \label{fig:a_fran_br} 
\end{figure}

Apenas para comparação, o mesmo método foi empregado com o catálogo \gls{iscgem}, como pode ser visto na figura 
\ref{fig:a_fran_sa}.

\begin{figure}[H]
  \centering
  \includegraphics[width=.80\textwidth]{a_frankel_sa} 
  \caption{Mapa do valor-a, catálogo \gls{iscgem} calculado pelo método de Frankel, 1995}
  \label{fig:a_fran_sa} 
\end{figure}

Nos dois casos é possível perceber a compatibilidade entre os resultados na maior parte do Brasil, mas com
diferenças significativas na região do Acre, Rondônia e na porção norte na Dorsal Atlântica. 
Note que pela limitação da escala, os valores em vermelho são quaisquer valores acima de 2.5

Os valores da ameaça calculados conforme a seção \ref{sec:hazard} estão na figura \ref{fig:pga_fran}.

\begin{figure}[H]
  \centering
  \includegraphics[width=.80\textwidth]{pga_frankel} 
  \caption{Mapa de ameaça sísmica, PGA (poe 10\%, 50y) [Frankel, 1995] }
  \label{fig:pga_fran} 
\end{figure}

Como técnica de suavização, é importante notar os principais altos de ameaça no nordeste e no centro do país.
A ameaça é recuperada razoávelmente na plataforma continental.

%% ------------------------------------------------------------------------- %%
\subsection{Woo, 1996}
\index{Woo, 1996!resultados}
\label{sec:woo_resultados}

O ajuste dos parâmetros da função de largura de banda resultou em $h(m)=1.39e^{1.18\,m}$km.

Aplicando-se o método de Woo ao catálogo brasileiro desagrupado, a taxa de sismicidade suavizada 
pode ser avaliada pela figura \ref{fig:a_woo}.

\begin{figure}[H]
  \centering
  \includegraphics[width=.80\textwidth]{a_woo} 
  \caption{Mapa do valor-a, usando o catálogo \gls{bsb2013} calculado pelo método de Woo, 1996 }
  \label{fig:a_woo} 
\end{figure}

A sismicidade de regiões como o nordeste, Goiás/Tocantins, sudeste e plataforma continental 
foram bem recuperadas, enquanto partes da região norte e centro-oeste não tiveram tanto destaque. 

O cálculo da ameaça (figura \ref{fig:pga_woo_inc}) foi feito com o resultado direto do método, utilizando
diretamente uma \gls{mfd} discreta.
\begin{figure}[H]
  \centering
  \includegraphics[width=.80\textwidth]{pga_woo_inc} 
  \caption{Mapa de ameaça sísmica, PGA (poe 10\%, 50y) 
  		   calculado com o Openquake a partir das fontes sísmicas
  		   determinas pelo método de Woo, 1996, usando uma \gls{mfd}
  		   discreta e incremental.
  }
  \label{fig:pga_woo_inc} 
\end{figure}

Coerentemente com a taxa de sismicidade modelada. Não houve local em que a taxa
da figura \ref{fig:a_woo} fosse baixa e que tenha apresentado grande ameaça.
As regiões norte, oeste e sul do país quase não figuraram 
enquanto parte do sudeste, plataforma continental, Goiás e nordeste foram
destacadas. Merece ser notado o maior valor de ameaça localizado em Pernambuco, dentre as três principais regiões mais ameaçadas do nordeste.

Para quantificar as diferenças nos cálculos de ameaça em decorrência da \gls{mfd} utilizada, em seguida é apresentado o
mapa de ameaça (figura \ref{fig:pga_woo_cum}) calculado a partir das taxas geradas pelo modelo, mas utilizando-se do
acumulado nos valores de magnitude acima do mínimo do catálogo e aplicando a transformação de Gutemberg-Richter 
para obter o \emph{valor-a}. Nesse procedimento, a partir de certa magnitude minima de referencia, toma-se o cumulativo
das taxas para magnitudes acima da referência, e, se projeta (por \gls{GR}) o \emph{valor-a}.

Essa projeção acumula incertezas que não são inerentes ao método. A proposta de Woo é fornecer
diretamente a distribuição de magnitudes em cada ponto da malha enquanto essa projeção é feita
assumindo um \emph{valor-b} arbitrário.


\begin{figure}[H]
	\centering
	\begin{subfigure}[t]{0.47\textwidth}
		\centering
		\includegraphics[width=1.0\textwidth]{pga_woo_cum} 
		\subcaption{Mapa de ameaça sísmica, PGA (poe 10\%, 50y), 
  		   calculado com o Openquake a partir das fontes sísmicas
  		   determinas pelo método de Woo, 1996, usando uma \gls{mfd}
  		   truncada usando o valor-a como o valor cumulativo
  		   contado a partir da \gls{sym:m_min}.
		}
		\label{fig:pga_woo_cum} 
	\end{subfigure}
	\quad
	\begin{subfigure}[t]{0.47\textwidth}
		\centering
		\includegraphics[width=1.0\textwidth]{pga_woo_dif} 
		\subcaption{Mapa diferencial de ameaça, PGA (poe 10\%, 50y), 
		   entre os modelos usando \gls{mfd} truncada e \gls{mfd}
		   discreta. Diferença entre os mapas das figuras 
		   \ref{fig:pga_woo_inc} e \ref{fig:pga_woo_cum}.
		   }
		\label{fig:pga_woo_dif} 
	\end{subfigure}
	\caption{Variação do resultado da ameaça em função do uso de diferentes 
			\glspl{mfd} no OpenQuake.}
	\label{fig:pga_woo} 
\end{figure}

A figura \ref{fig:pga_woo_dif} quantifica a diferença nos valores de ameaça obtitos quando 
utilizadas as diferentes distribuições de frequência e magnitude. Nota-se que enquanto
o modelo com \gls{mfd} incremental (método original) destaca principalmente o nordeste, o modelo utilizando o
\emph{valor-a} cumulativo (projetado) e uma \gls{mfd} truncada, destacou principalmente as ameaças no centro-oeste e
sudeste.


%% ------------------------------------------------------------------------- %%
\subsection{Helmstetter, 2012}
\index{Helmstetter, 2012!resultados}
\label{sec:helmstetter_resultados}

A otimização dos parâmetros do modelo de Helmstetter conforme descrito na seção \ref{sec:proc_helmstetter}
resultou, com ganho\footnote{equação \eqref{eq:gain}} $G = 2.43$ sobre uma distribuição uniforme, nos valores
apresentados na tabela \ref{tab:hemlstetter}.

\begin{table}[H]
	\centering
	\begin{tabular}{c|c}
		Parâmetro & Valor \\ \hline
		$R_{min}$ & $0.1\times10^{-13}$ \\
		$a_{cnn}$ & 325 \\
		$k_{cnn}$ & 1 \\
	\end{tabular}
	\caption{Parâmetros otimizados (método simplex) para o modelo de Helmstetter a partir do catálogo \gls{bsb2013}.}
	\label{tab:hemlstetter}
\end{table}

Usando os parâmetros otimizados, os valores calculados para a taxa de sismicidade
de longo-prazo pelo método proposto por Helmstetter pode ser visto na figura \ref{fig:helm_r}.

\begin{figure}[H]
  \centering
  \includegraphics[width=.80\textwidth]{a_helmstetter} 
  \caption{Mapa do valor-a, usando o catálogo \gls{bsb2013} calculado pelo método de Helmstetter, 2012.}
  \label{fig:helm_r} 
\end{figure}

Observa-se que a sismicidade de regiões como amazônia, Mato-Grosso e Pará foram destacadas,
enquanto no sudeste houve maior diluição. A sismicidade do nordeste aparece evidente, principalmente
nas regiões do Ceará, Rio Grande do Norte e Pernambuco. 

É fato também que a sismicidade do centro-oeste rumo ao norte pelo estado de Goiás praticamente desapareceu.
Isso se deve provavelmente ao fato de que grande parte dos sismos dessa região possuem magnitudes não 
muito elevadas, tendo sido em grande parte removidas pela magnitude de completude como se pode notar nos catálogos
de aprendizado e testes na figura \ref{fig:h_catalogues}.

Na figura~\ref{fig:helm_h} estão os valores calculados para o mapa de ameaça 
com as taxas de sismicidade suavizadas pelo método.

\begin{figure}[H]
  \centering
  \includegraphics[width=.80\textwidth]{pga_helmstetter} 
  \caption{Mapa de ameaça sísmica, PGA (poe 10\%, 50y), 
  		   calculado com o OpenQuake a partir das fontes sísmicas
  		   determinas pelo método de Helmstetter, 2012.}
  \label{fig:helm_h} 
\end{figure}

Nota-se que ao passo em que houve um grande realçe da ameaça em algumas regiões como amazônia e Pará (próximo a Belém),
em regiões como a faixa sísmica no norte de Goiás e Tocantins não apresentaram o mesmo contraste.


