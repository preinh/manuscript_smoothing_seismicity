\documentclass[final]{beamer}
\mode<presentation>

% STEP 1: Change the next line according to your language
\usepackage[english]{babel}

% STEP 2: Make sure this character encoding matches the one you save this file as
% (this template is utf8 by default but your editor may change it, causing problems)
\usepackage[utf8]{inputenc}

% You probably don't need to touch the following four lines
\usepackage[T1]{fontenc}
\usepackage{lmodern}
\usepackage{amsmath,amsthm, amssymb, latexsym}
\usepackage{exscale} % required to scale math fonts properly

% PAUSE: my own packages
% figure numbers, captions and subcaptions
\usepackage[font=scriptsize,format=plain,labelfont=bf,up,textfont=it,up,compatibility=false]{caption}
\usepackage[font=scriptsize,compatibility=false]{subcaption}
\usepackage[caption=true]{subfig}

\setbeamertemplate{caption}[numbered]

% bibliografia
%\usepackage[fixlanguage]{babelbib}
\usepackage[round,sort,nonamebreak]{natbib} % citação bibliográfica textual(plainnat-ime.bst)
%\bibpunct{(}{)}{;}{a}{\hspace{-0.7ex},}{,} % estilo de citação. Veja alguns
% exemplos em http://merkel.zoneo.net/Latex/natbib.php
\bibliographystyle{styles/plainnat-ime} 	% citação bibliográfica textual


% Glossaries
\usepackage[sanitize=none,acronym,toc]{glossaries}
%\usepackage[acronym,toc]{glossaries}
% Define a new glossary type
\newglossary[slg,toc]{symbols}{sym}{sbl}{Lista de Símbolos}
\newglossary[elg]{equations}{eqn}{eql}{Equations}
\makeglossaries


% ---------------------------------------------------------------------------- %
% Dicionario de Termos:
%-----------------------------------------
% terms
%-----------------------------------------
\newglossaryentry{equake}
{
	name={earthquake},
	description={rupture in some geological strutcture},
	plural={earthquakes}
}

\newglossaryentry{seismicity}
{
	name={seismicity},
	description={earthquake occurrence},
}

\newglossaryentry{ED}
{
	name={\gls{ed}},
	description={error diagram},
	plural={error diagrams}
}


\newglossaryentry{hypocenter}
{
	name={hypocenter},
	description={geometrical representation of the starting point of the \gls{rupture_process} on the \gls{crust}},
	plural={hypocenters}
}

\newglossaryentry{epicenter}
{
	name={epicenter},
	description={orthogonal projection over Earth surface of \gls{hypocenter}},
	plural={epicentros}
}


\newglossaryentry{seismotectonic}
{
	name={seismotectonic},
	description={o estudo das relações entre os \glspl{equake} e a \gls{tectonic} recente de uma região.
				 Procura compreender exatamente quais são os mecanismos que levam à uma ruptura geológica 
				 e são responsáveis pela
				 \gls{seismic_activity} em uma certa área. Isso é feito analisando-se de forma combinada 
				 registros recentes de tectonismo global e regional, 
				 considerando também evidências históricas e geomorfológicas},
}


\newglossaryentry{rupture_process}
{
	name={processo de ruptura},
	description={processo que envolve o rompimento de uma região da crosta,
			o deslocamento relativo entre essas regiões, e consequantemente,
			a liberação de uma grande quantidade de energia, de forma praticamente
			instantânea, tomando-se como referência o \gls{geologic_time}},
	plural={processos de ruptura}
}


\newglossaryentry{geologic_time}
{
	name={tempo geológico},
	description={escala de tempo que vai desde a formação do universo até os tempos atuais,
				englobando a formação do planeta e as transformações ocorridas desde então},
}


\newglossaryentry{tectonic}
{
	name={tectonic},
	description={disciplina científica focada nos processos respons\'aveis 
				 pela cria\c{c}\~ao e transforma\c{c}\~ao das estruturas geológicas da Terra e de outros planetas.},
	plural={tectonics}
}


%\newglossaryentry{oq}
%{
%	name={OpenQuake},
%	description={programa de código aberto para o calculo de risco sísmico mantido pela Fundação \gls{gem}},
%	plural={OpenQuake}
%}


\newglossaryentry{crust}
{
	name={crosta terrestre},
	description={parte superficial, rígida e mais externa do planeta Terra},
}

\newglossaryentry{mantle}
{
	name={manto terrestre},
	description={material da por{ç}{ã}o intermediária do planeta, 
		fluido em tempo geológico},
}

\newglossaryentry{core}
{
	name={n{ú}cleo terrestre},
	description={por{ç}{ã}o mais central do planeta, com predomin{â}ncia de compostos metálicos},
}

\newglossaryentry{tectonic_plate_theory}
{
	name={teoria tect{ô}nica das placas},
	description={foi uma teoria revolucionária para a \gls{tectonic},
				propondo que a \gls{crust} terrestre estivesse dividida 
				em placas {à} deriva sobre o \gls{mantle}},
}


\newglossaryentry{litho_plate}
{
	name={placa litosf{é}rica},
	plural={placas litosf{é}ricas},
	description={placa de material da \gls{lithosphere}},
}


\newglossaryentry{lithosphere}
{
	name={litosfera},
	description={região rúptil, mais externa do planeta, formada pela \gls{crust} 
		(continental e ocêanica) e parte do \gls{mantle} superior, com aproximadamente 
		60\gls*{sym:km} de profundidade},
}


\newglossaryentry{astenosphere}
{
	name={astenosfera},
	description={região dúctil entre a \gls{lithosphere} e o \gls{mantle},
				com profundidades que variam de 60 a 700km},
}

\newglossaryentry{smoothing}
{
	name={smoothing techniques},
	description={consiste em capturar importantes feições do conjunto de dados,
				 eliminando ruídos e outras estruturas de curto comprimento de onda
				 presentes nos dados},
}

\newglossaryentry{kernel_function}
{
	name={kernel function},
	description={n-dimentional function, which integral over whole domain is equal one and could be used as a probability density estimation.},
	plural={kernel functions},
}

\newglossaryentry{seismic_rate}
{
	name={seismic rate},
	description={rate within earthquakes are generated in some \gls{seismic_source}},
	plural={seismic rates},
}

\newglossaryentry{seismic_activity}
{
	name={seismic activity},
	description={frequency of \glspl{equake} occurence},
}

\newglossaryentry{poisson_process}
{
	name={processo de Poisson},
	description={uma sequencia de intervalos discretos com um experimento de Bernoulli em cada},
}

\newglossaryentry{seismic_source}
{
	name={seismic source},
	description={geological structure able to produce \glspl{equake}},
	plural={seismic sources}
}

\newglossaryentry{point_source}
{
	name={point seismic source},
	description={geometrical representation as point of some seismogenic source},
	plural={point seismic sources},
}

\newglossaryentry{gmpe}
{
	name={GMPE},
	description={ground motion prediction equation},
	plural={GMPEs},
}


\newglossaryentry{area_source}
{
	name={area seismic source},
	description={representação geométrica por um polígono em superfície, 
				 de uma fonte sísmica},
	plural={farea seismic sources},
}

\newglossaryentry{titulo_da_dissertacao}
{
	name={titulo_da_dissertacao},
	description={Técnicas de suavização aplicadas
					à caracterização de fontes sísmicas e 
					à análise probabilística de ameaça sísmica},
}

\newglossaryentry{isocista}
{
	name={isocist},
	description={border of a region with the same seismic intensity from the same \gls{equake}},
}

       
%-----------------------------------------
% symbols
%-----------------------------------------

\newglossaryentry{sym:t}
{
	name={\ensuremath{t}},
	description={time},
	symbol={\ensuremath{t}},
	type=symbols
}

\newglossaryentry{sym:P}
{
	name={\ensuremath{P}},
	description={probability},
	symbol={\ensuremath{P}},
	type=symbols
}

\newglossaryentry{sym:E}
{
	name={\ensuremath{E}},
	description={expected value},
	symbol={\ensuremath{E}},
	type=symbols
}

\newglossaryentry{sym:Var}
{
	name={\ensuremath{Var}},
	description={variance},
	symbol={\ensuremath{Var}},
	type=symbols
}

\newglossaryentry{sym:epsilon}
{
	name={\ensuremath{\epsilon}},
	description={error},
	symbol={\ensuremath{\epsilon}},
	type=symbols
}

\newglossaryentry{sym:sigma}
{
	name={\ensuremath{\sigma}},
	description={standard deviation},
	symbol={\ensuremath{\sigma}},
	type=symbols
}


\newglossaryentry{sym:r}
{
	name={\ensuremath{\boldsymbol{r}}},
	description={space locallity},
	symbol={\ensuremath{\boldsymbol{r}}},
	type=symbols
}


\newglossaryentry{sym:m}
{
	name={\ensuremath{m}},
	description={magnitude},
	symbol={\ensuremath{m}},
	type=symbols
}


\newglossaryentry{sym:lambda}
{
	name={\ensuremath{\lambda}},
	description={seismic rate regressor},
	symbol={\ensuremath{\lambda}},
	type=symbols
}

\newglossaryentry{sym:M_0}
{
	name={\ensuremath{M_0}},
	description={seismic moment},
	symbol={\ensuremath{M_0}},
	type=symbols
}


\newglossaryentry{sym:mu}
{
	name={\ensuremath{\mu_{stf}}},
	description={stiffness coefficient},
	symbol={\ensuremath{\mu_{stf}}},
	type=symbols
}


\newglossaryentry{sym:A}
{
	name={\ensuremath{A}},
	description={felt area},
	symbol={\ensuremath{A}},
	type=symbols
}


\newglossaryentry{sym:D}
{
	name={\ensuremath{\tilde{D}}},
	description={mean displacement},
	symbol={\ensuremath{\tilde{D}}},
	type=symbols
}


\newglossaryentry{sym:MW}
{
	name={\ensuremath{M_W}},
	description={moment magnitude},
	symbol={\ensuremath{M_W}},
	type=symbols
}

\newglossaryentry{sym:A_richter}
{
	name={\ensuremath{\hat{A}}},
	description={amplitude from an Wood-Anderson seismometer},
	symbol={\ensuremath{\hat{A}}},
	type=symbols
}

\newglossaryentry{sym:d_richter}
{
	name={\ensuremath{\hat{d}}},
	description={distance far 100km from earthquake},
	symbol={\ensuremath{\hat{d}}},
	type=symbols
}


\newglossaryentry{sym:b}
{
	name={\ensuremath{b}},
	description={b-value}, 
	symbol={\ensuremath{b}},
	type=symbols
}


\newglossaryentry{sym:a}
{
	name={\ensuremath{a}},
	description={a-value},
	symbol={\ensuremath{a}},
	type=symbols
}


\newglossaryentry{sym:N_m}
{
	name={\ensuremath{N(m,m+\mathrm{d}m)}},
	description={number of earthquakes with magnitude values between $m$ and $m + \mathrm{d}m$ },
	symbol={\ensuremath{N(m)}},
	type=symbols
}


\newglossaryentry{sym:m_min}
{
	name={\ensuremath{m_{min}}},
	description={minimum magnitude},
	symbol={\ensuremath{m_{min}}},
	type=symbols
}

\newglossaryentry{sym:m_max}
{
	name={\ensuremath{m_{max}}},
	description={maximum magnitude},
	symbol={\ensuremath{m_{max}}},
	type=symbols
}

\newglossaryentry{sym:m_c}
{
	name={\ensuremath{m_c}},
	description={completeness magnitude},
	symbol={\ensuremath{m_c}},
	type=symbols
}



\newglossaryentry{sym:m_corner}
{
	name={\ensuremath{m_{corner}}},
	description={magnitude value which controls the Kagan-MFD behavior},
	symbol={\ensuremath{m_{corner}}},
	type=symbols
}

\newglossaryentry{sym:M}
{
	name={\ensuremath{M}},
	description={magnitude random variable},
	symbol={\ensuremath{M}},
	type=symbols
}

\newglossaryentry{sym:beta}
{
	name={\ensuremath{\beta}},
	description={\beta = \gls{sym:b}\ln{10}},
	symbol={\ensuremath{\beta}},
	type=symbols
}

\newglossaryentry{sym:beta_p}
{
	name={\ensuremath{\beta_p}},
	description={$\beta_p = \frac{2}{3}\gls{sym:b}$ is the Pareto's beta},
	symbol={\ensuremath{\beta_p}},
	type=symbols
}

\newglossaryentry{sym:alpha}
{
	name={\ensuremath{\alpha}},
	description={total number of earthquakes},
	symbol={\ensuremath{\alpha}},
	type=symbols
}


\newglossaryentry{sym:ri}
{
	name={\ensuremath{\boldsymbol{r}_i}},
	description={spatial location of earthquake $i$},
	symbol={\ensuremath{\boldsymbol{r}_i}},
	type=symbols
}


\newglossaryentry{sym:ti}
{
	name={\ensuremath{t_i}},
	description={time location of earthquake $i$},
	symbol={\ensuremath{t_i}},
	type=symbols
}


\newglossaryentry{sym:hi}
{
	name={\ensuremath{h_i}},
	description={temporal bandwidth for earthquake $i$},
	symbol={\ensuremath{h_i}},
	type=symbols
}


\newglossaryentry{sym:di}
{
	name={\ensuremath{d_i}},
	description={spatial bandwidth for earthquake $i$},
	symbol={\ensuremath{d_i}},
	type=symbols
}


\newglossaryentry{sym:wi}
{
	name={\ensuremath{ w }},
	description={weight},
	symbol={\ensuremath{ w }},
	type=symbols
}


\newglossaryentry{sym:Mc_rt}
{
	name={\ensuremath{ M_c\left( \gls{sym:r}, \gls{sym:t} \right)  }},
	description={completenes magnitude on location \gls{sym:r} and \gls{sym:t}},
	symbol={\ensuremath{ M_c\left( \gls{sym:r}, \gls{sym:t} \right) }},
	type=symbols
}


\newglossaryentry{sym:Mc}
{
	name={\ensuremath{M_c}},
	description={completeness magnitude},
	symbol={\ensuremath{M_c}},
	type=symbols
}

\newglossaryentry{sym:Md}
{
	name={\ensuremath{M_d}},
	description={minimum magnitude value on the catalogue},
	symbol={\ensuremath{M_d}},
	type=symbols
}


\newglossaryentry{sym:Rmin}
{
	name={\ensuremath{R_{min}}},
	description={minimum seismic rate},
	symbol={\ensuremath{R_{min}}},
	type=symbols
}


\newglossaryentry{sym:R}
{
	name={\ensuremath{R(\gls{sym:r},\gls{sym:t})}},
	description={seismic rate located \gls{sym:r} distant from the earthquake occurred on the instant \gls{sym:t}},
	symbol={\ensuremath{R(\gls{sym:r},\gls{sym:t})}},
	type=symbols
}


\newglossaryentry{sym:Rrm}
{
	name={\ensuremath{R(\gls{sym:r},\gls{sym:m})}},
	description={seismic rate located \gls{sym:r} distant from the earthquake occurred on the instant \gls{sym:t}},
	symbol={\ensuremath{R(\gls{sym:r},\gls{sym:t})}},
	type=symbols
}


\newglossaryentry{sym:Kt}
{
	name={\ensuremath{K_t \left( \frac{ t - \gls{sym:ti} }{ \gls{sym:hi} } \right) }},
	description={time domain kernel function, where
					\gls{sym:ti} is the \glsdesc{sym:ti} and
					\gls{sym:hi} is the \glsdesc{sym:hi}
				},
	symbol={\ensuremath{K_t \left( \frac{ t - \gls{sym:ti} }{ \gls{sym:hi} } \right)}},
	type=symbols
}

\newglossaryentry{sym:Kr}
{
	name={\ensuremath{K_r \left( \frac{ \| \gls{sym:r} - \gls{sym:ri} \| }{d_i} \right) }},
	description={space domain kernel function, where
					\gls{sym:ri} is the \glsdesc{sym:ri} and
					\gls{sym:di} is the \glsdesc{sym:di}
	},
	symbol={\ensuremath{K_r \left( \frac{ \| \gls{sym:r} - \gls{sym:ri} \| }{d_i} \right)}},
	type=symbols
}


\newglossaryentry{sym:Krm}
{
	name={\ensuremath{K(\gls{sym:r},\gls{sym:m})}},
	description={kernel function for some distance \gls{sym:r} from earthquakes with magnitudes  \gls{sym:m}},
	symbol={\ensuremath{K_1 \left( \frac{ t - \gls{sym:ti} }{ \gls{sym:hi} } \right)}},
	type=symbols
}

\newglossaryentry{sym:a_cnn}
{
	name={\ensuremath{a_{cnn}}},
	description={space-time coupling factor},
	symbol={\ensuremath{a_{cnn}}},
	type=symbols
}

\newglossaryentry{sym:k_cnn}
{
	name={\ensuremath{k_{cnn}}},
	description={$k^{th}$ nearest neighbour},
	symbol={\ensuremath{k_{cnn}}},
	type=symbols
}


\newglossaryentry{sym:dk}
{
	name={\ensuremath{d_k}},
	description={$\max{\left\{ d_j \right\}}, j=1,\ldots,k_{cnn}$},
	symbol={\ensuremath{d_k}},
	type=symbols
}

\newglossaryentry{sym:hk}
{
	name={\ensuremath{h_k}},
	description={$\max{\left\{ h_j \right\} }, j=1,\ldots,k_{cnn}$},
	symbol={\ensuremath{h_k}},
	type=symbols
}

\newglossaryentry{sym:ixiy}
{
	name={\ensuremath{\left(i_x, i_y\right)}},
	description={each grid cell},
	symbol={\ensuremath{\left(i_x, i_y\right)}},
	type=symbols
}


\newglossaryentry{sym:N}
{
	name={\ensuremath{N}},
	description={number of earthquakes on the catalog},
	symbol={\ensuremath{N}},
	type=symbols
}



\newglossaryentry{sym:Np}
{
	name={\ensuremath{N_p\left(i_x, i_y\right)}},
	description={predicted seismic rate on cell \gls{sym:ixiy}},
	symbol={\ensuremath{N_p\left(i_x, i_y\right)}},
	type=symbols
}


\newglossaryentry{sym:Nu}
{
	name={\ensuremath{N_u}},
	description={\gls{sym:Nt}/\gls{sym:Nc}},
	symbol={\ensuremath{N_u}},
	type=symbols
}


\newglossaryentry{sym:Nc}
{
	name={\ensuremath{N_c}},
	description={number of grid cells},
	symbol={\ensuremath{N_c}},
	type=symbols
}


\newglossaryentry{sym:Npi}
{
	name={\ensuremath{N_p(i)}},
	description={predicted seismicity rate on spatial bin where earthquake
	$i$ occurred}, symbol={\ensuremath{N_p(i)}},
	type=symbols
}


\newglossaryentry{sym:NAi}
{
	name={\ensuremath{N_A(i)}},
	description={seismic rate on $i$ predicted by the $A$ model}, 
	symbol={\ensuremath{N_A(i)}},
	type=symbols
}


\newglossaryentry{sym:NBi}
{
	name={\ensuremath{N_B(i)}},
	description={seismic rate on $i$ predicted by the $B$ model},
	symbol={\ensuremath{N_B(i)}},
	type=symbols
}


\newglossaryentry{sym:Ts}
{
	name={\ensuremath{T_s}},
	description={Student distribution $T$-value},
	symbol={\ensuremath{T_s}},
	type=symbols
}

\newglossaryentry{sym:nxy}
{
	name={\ensuremath{n\left(i_x, i_y\right)}},
	description={number of target earthquakes observed on cell \gls{sym:ixiy}},
	symbol={\ensuremath{n\left(i_x, i_y\right)}},
	type=symbols
}


\newglossaryentry{sym:Nt}
{
	name={\ensuremath{N_t}},
	description={number of events on the target catalog},
	symbol={\ensuremath{N_t}},
	type=symbols
}



\newglossaryentry{sym:L}
{
	name={\ensuremath{L}},
	description={log-likelihood},
	symbol={\ensuremath{L}},
	type=symbols
}


\newglossaryentry{sym:Lu}
{
	name={\ensuremath{L_u}},
	description={uniform model likelihood},
	symbol={\ensuremath{L}},
	type=symbols
}


\newglossaryentry{sym:pNn}
{
	name={\ensuremath{p(N_p, n)}},
	description={probability of observe exactly $n$ events with probability \gls{sym:np} each one},
	symbol={\ensuremath{p(N_p, n)}},
	type=symbols
}


\newglossaryentry{sym:G}
{
	name={\ensuremath{G}},
	description={probability gain from each earthquake on target-catalog over a spatialy uniform Poisson model.}, 
	symbol={\ensuremath{G}}, 
	type=symbols
}


\newglossaryentry{sym:I}
{
	name={\ensuremath{ I_{inf}(A,B)}},
	description={information gain from the model $A$ over the model $B$}, 
	symbol={\ensuremath{I_{inf}(A,B)}}, 
	type=symbols
}


\newglossaryentry{sym:aW}
{
	name={\ensuremath{a_W}},
	description={fractal dimension factor, generally about 1.5 and 2}, 
	symbol={\ensuremath{a_W}}, 
	type=symbols
}



\newglossaryentry{sym:DW}
{
	name={\ensuremath{D_W}},
	description={fractal dimenstion of spatial earthquake occurrence $D_W = 2-\gls{sym:aW}$}, 
	symbol={\ensuremath{D_W}}, 
	type=symbols
}



\newglossaryentry{sym:hm}
{
	name={\ensuremath{h(m)}},
	description={fixed kernel bandwidth for earthquakes with magnitude $m$}, 
	symbol={\ensuremath{h(m)}}, 
	type=symbols
}



\newglossaryentry{sym:a0}
{
	name={\ensuremath{a_0}},
	description={\gls{sym:hm} linear parameter}, 
	symbol={\ensuremath{a_0}}, 
	type=symbols
}



\newglossaryentry{sym:a1}
{
	name={\ensuremath{a_1}},
	description={\gls{sym:hm} exponential parameter}, 
	symbol={\ensuremath{a_1}}, 
	type=symbols
}




\newglossaryentry{sym:dF}
{
	name={\ensuremath{d_F}},
	description={correlation distance or \emph{fixed bandwidth}}, 
	symbol={\ensuremath{d_F}}, 
	type=symbols
}


\newglossaryentry{sym:dij}
{
	name={\ensuremath{d_{ij}}},
	description={distance between grid cells $i$ and $j$}, 
	symbol={\ensuremath{d_{ij}}}, 
	type=symbols
}


\newglossaryentry{sym:I0}
{
	name={\ensuremath{I_0}},
	description={maximum intensity}, 
	symbol={\ensuremath{I_0}}, 
	type=symbols
}


\newglossaryentry{sym:Af}
{
	name={\ensuremath{A_{f}}},
	description={felt area in square km}, 
	symbol={\ensuremath{A_{f}}}, 
	type=symbols
}



       

%-----------------------------------------
% acronyms
%-----------------------------------------

\newacronym{psha}{PSHA}{Análise Probabilística de Ameaça Sísmica}
\newacronym{dsha}{DSHA}{Análise Determinística de Ameaça Sísmica}
\newacronym{GMPE}{\gls{gmpe}}{\glsdesc{gmpe}}
\newacronym{mfd}{MFD}{Distribuição de Frequência e Magnitude}
\newacronym{gr}{GR}{Gutenberg-Richter}
\newacronym{cnn}{CNN}{Acoplamento dos Vizinhos mais Próximos}
\newacronym{va}{v.a.}{variável aleatória}
\newacronym{iid}{i.i.d.}{independentes e identicamente distribuidas}
\newacronym{pdf}{densidade}{função de densidade de probabilidade}
\newacronym{pmf}{pmf}{função de distribuição acumulada de probabilidade}

\newacronym{pga}{PGA}{máxima aceleração do chão}

\newacronym{isc}{ISC}{International Seismological Centre}
\newacronym{gem}{GEM}{Global Earthquake Model}
\newacronym{opensha}{openSHA}{Open Source Seismic Hazard Analysis}
\newacronym{oq}{OQ}{Openquake}
\newacronym{eeri}{EERI}{Earthquake Engineering Research Institute}
\newacronym{cgmw}{CGMW}{Comission for Geological Map of the World}



\newacronym{hl}{HazardLib}{Biblioteca de Ameaça}
\newacronym{rl}{RiskLib}{Biblioteca de Risco}
\newacronym{nrml}{NRML}{Natural-hazard and Risk Markup Language}
\newacronym{hmtk}{HMTK}{Hazard Modeller's Toolkit}
\newacronym{oqp}{oq-platform}{Plataforma web de Interação com o OQ}
\newacronym{oqe}{oq-engine}{Motor de Cálculo do OQ}

\newacronym{obsis}{ObSis}{Observatório Sismológico}
\newacronym{unb}{UnB}{Universidade de Brasilia}
\newacronym{iag}{IAG}{Instituto de Astronomia, Geofísica e Ciências Atmosféricas}
\newacronym{usp}{USP}{Universidade de São Paulo}
\newacronym{ipt}{IPT}{Instituto de Pesquisas Tecnológicas}
\newacronym{unesp}{UNESP}{Universidade Estatual Paulista}
\newacronym{ufrn}{UFRN}{Universidade Federal do Rio Grande do Norte}

\newacronym{bsb}{BSB}{Boletim Sísmico Brasileiro}
\newacronym{csv}{CSV}{Valores Separados por Vírgulas}

\newacronym{bsb2013}{{BSB-2013.08}}{Brazilian Seismic Bulletin version 2013.08}
\newacronym{iscgem}{{\gls*{isc}-\gls*{gem}}}{Catálogo ISC-GEM para a América do
Sul}


       
%-----------------------------------------
% equations
%-----------------------------------------

\newglossaryentry{eqn:M_0}
{
	name={\ensuremath{\gls{sym:M_0} = \gls{sym:mu}\gls{sym:A}\gls{sym:D}}},
	description={onde \gls{sym:mu} é \glsdesc{sym:mu}, 
				 \gls{sym:A} é \glsdesc{sym:A} e 
				 \gls{sym:D} é \glsdesc{sym:D}. 
				 Tem unidades de energia [N.m]
		 },
	type=equations
}

\newglossaryentry{eqn:M_W}
{
	name={\ensuremath{\gls{sym:MW} = \frac{2}{3} \log_{10}{\gls{sym:M_0}} - 10.7 }},
	description={onde \gls{sym:M_0} é \glsdesc{sym:M_0} em [N.m]
		 },
	type=equations
}

\newglossaryentry{eqn:richter}
{
	name={\ensuremath{\log{\gls{sym:A_richter}} = 3.37 - 3\log{\gls{sym:d_richter}}}},
	description={onde \gls{sym:A_richter} é \glsdesc{sym:A_richter} e 
					  \gls{sym:d_richter} é \glsdesc{sym:d_richter}
		 },
	type=equations
}

\newglossaryentry{eqn:gr_mfd}
{
	name={\ensuremath{\log{\gls{sym:N_m}} = \gls{sym:a} - \gls{sym:b}\gls{sym:m} }},
	description={onde \gls{sym:N_m} é o \glsdesc{sym:N_m}, 
					  \gls{sym:a} é o \glsdesc{sym:a},
					  \gls{sym:b} é o \glsdesc{sym:b}
    },
	type=equations
}


\newglossaryentry{eqn:Fm_richter}
{
	name={\ensuremath{\log{\gls{sym:N_m}} = \gls{sym:a} - \gls{sym:b}\gls{sym:m} }},
	description={onde \gls{sym:N_m} é \glsdesc{sym:N_m}, 
					  \gls{sym:a} é o \glsdesc{sym:a},
					  \gls{sym:b} é o \glsdesc{sym:b}
    },
	type=equations
}

       
% ---------------------------------------------------------------------------- %


\graphicspath{{./images/}}             % caminho das figuras (recomendável)
\include{template}  % THIS is the line that includes the template!


% ---------------------------------------------------------------------------- %
\DeclareMathOperator*{\argmin}{arg\,min}
\DeclareMathOperator*{\argmax}{arg\,max}
\DeclareMathOperator*{\erf}{erf}
% ---------------------------------------------------------------------------- %


% Orientation is set here
\usepackage[orientation=portrait,size=a0,scale=1.4]{beamerposter}

% STEP 3:
% Change colours by setting \usetheme[<id>, twocolumn]{HYposter}.
\usetheme[iag, threecolumn]{HYposter}
% The different ids are:
%  maa: Faculty of Agriculture and Forestry 
%  hum: Faculty of Arts 
%  kay: Faculty of Behavioural Sciences 
%  bio: Faculty of Biological and Environmental Sciences 
%  oik: Faculty of Law 
%  med: Faculty of Medicine 
%  far: Faculty of Pharmacy 
%  mat: Faculty of Science 
%  val: Faculty of Social Sciences 
%  teo: Faculty of Theology 
%  ell: Faculty of Veterinary Medicine 
%  soc: Swedish School of Social Science 
%  kir: University of Helsinki Library
%  avo: Open University
%  ale: Aleksanteri Institute
%  neu: Neuroscience Institute
%  biot: Bioscience Institute
%  atk: Computer centre
%  rur: Ruralia Institute
%  koe: Laboratory animal centre
%  kol: Collegium for Advanced Studies
%  til: Center for Properties and Facilities
%  pal: Palmenia
%  kie: Language centre
% Without options a black theme without faculty name will be used.


% STEP 4: Set up the title and author info
\titlestart{Smoothing techniques applied to} % first line of title
\titleend{Brazilian seismic sources characterization} % second line of title
\titlesize{\LARGE} % Use this to change title size if necessary. See README
% for details.

\author{Marlon Pirchiner$^{1,2}$\\
\vspace{0.2cm} \scriptsize \alert{\url{marlon@iag.usp.br}} } 

\institute{
$^1$Seismological Centre, IAG-USP,\\
$^2$Applied Math School, EMAp-FGV-RJ
}

% Stuff such as logos of contributing institutes can be put in the lower left corner using this
\leftcorner{
\includegraphics[width=5cm]{images/qr_code}}


\begin{document}
\begin{poster}


%===========================================
\newcolumn
%===========================================


% STEP 5: Add the contents of your poster between \begin{poster} and \end{poster}
%-------------------------------------------
\section{Introduction}
%-------------------------------------------
\footnotesize
In some cases the lack of geological information makes difficult to define and
characterize all potential ground motion seismic sources. On this context, 
this work review and evaluate three seismic rate smoothing techniques,
which will used to define the brazilian smoothed rates as a punctual seismic
sources grid. This input model will used to perform hazard calculation on
\emph{\alert{openquake-engine}}
(\scriptsize
 \url{http://www.globalquakemodel.org/openquake}
 \footnotesize) 
in order to compare them with the previous results for the region. 
For comparission effects an unity \emph{b-value} and the Toro (1997) GMPE was
used.

%-------------------------------------------
\section{Seismicity}
%-------------------------------------------
\footnotesize
The following figure \ref{fig:br_seis} shows the brazilian seismicity.
\begin{figure}[H]
	\scriptsize
	\centering
	\includegraphics[width=1.0\textwidth]{seismicity_br} 
	\caption{Brazilian seismicity. \gls{bsb2013} catalogue.}
	\label{fig:br_seis} 
\end{figure}

\begin{columns}
	\begin{column}[T]{0.50\textwidth}
		\centering
		\begin{figure}[H]
		  	\centering
			\includegraphics[width=1.15\textwidth]{hmtk_bsb2013_rate}
			\caption{Annual Seismic Rate}
			\label{fig:sa_eq_record}
		\end{figure}
	\end{column}

	\begin{column}[T]{0.48\textwidth}
		\centering
		\begin{figure}[H]
		  	\centering
			\includegraphics[width=1.0\textwidth]{occurrence}
			\caption{Recurrence}
			\label{fig:br_eq_record}
	    \end{figure}
	\end{column}
\end{columns}

%-------------------------------------------
\section{Declustering and Completeness}
%-------------------------------------------

\begin{columns}
	\begin{column}[T]{0.49\textwidth}
	\centering
		\begin{figure}[H]
		  	\centering
			\includegraphics[width=1.10\textwidth]{decluster_br}
			\caption{Cumulative EQ records and declustering methods}
			\label{fig:br_eq_record}
		\end{figure}
	\end{column}

	\begin{column}[T]{0.51\textwidth}
		\begin{figure}[H]
		  	\centering
			\includegraphics[width=1.00\textwidth]{stepp_br}
			\caption{Stepp diagram: magnitude completeness evaluation}
			\label{fig:br_stepp}
	    \end{figure}
	\end{column}
\end{columns}



%-------------------------------------------
\section{Previous works}
%-------------------------------------------

\begin{columns}
	\begin{column}[T]{0.49\textwidth}
	\centering
		\begin{figure}[H]
		  \centering
		  \includegraphics[width=1\textwidth]{pga_gshap} 
		  \caption{GSHAP (Giardini \emph{et al}, 1999): PGA(poe 0.1/50y) [$g$]}
		  \label{fig:gshap} 
		\end{figure}
	\end{column}

	\begin{column}[T]{0.49\textwidth}
		\begin{figure}[H]
		  \centering
		  \includegraphics[width=1\textwidth]{pga_dourado_oq} 
		  \caption{\citet{dourado_2014}: PGA(poe 0.1/50y)}
		  \label{fig:pga_dourado_oq} 
		\end{figure}
	\end{column}
\end{columns}


%===========================================
\newcolumn
%===========================================

%-------------------------------------------
\section{Frankel, 1995}
%-------------------------------------------
\footnotesize
\citet{frankel_1995} proposed a gaussian smoothing (\ref{eq:ni}) for the
earthquake grid count.
\small
\begin{equation}
	\ensuremath{
		\tilde{n}_j = \frac{ \sum_{i} n_i \,e^{ - \left(\frac{\gls{sym:dij}}{\gls{sym:dF}}\right)^2}}
						   { \sum_{i}     e^{ - \left(\frac{\gls{sym:dij}}{\gls{sym:dF}}\right)^2}},
	}
	\label{eq:ni}
\end{equation}
\footnotesize
where $d_F$ is the \alert{fixed bandwidth}, called correlation distance,
$\tilde{n}_j$ is the cumulative seismic rate smoothed on cell $j$,
$n_i$ is the number of earthquakes on cell $i$ and \gls{sym:dij} is
the \glsdesc{sym:dij}.

%-------------------------------------------
\section{Woo, 1996}
%-------------------------------------------
\footnotesize
\citet{woo_1996} proposed a magnitude dependent smoothed seismic rate
as shown on equation (\ref{eq:Rrm}).
\footnotesize
		\begin{equation}
			\ensuremath{
				\gls{sym:Rrm} = \sum_{i=1}^{N} \frac{ K(\gls{sym:r} - \gls{sym:ri}, m)}
													{T({\gls{sym:ri})}},
			}
			\label{eq:Rrm}
		\end{equation}
\footnotesize
	where $N$ is number of earthquakes $i$ on the catalogue 
	and $T(\gls{sym:ri})$ is the complete observation time frame for magnitudes
	greather than $m$ on \gls{sym:ri}.

The kernel shape used on this work was
\footnotesize
		\begin{equation}
			\ensuremath{
				K_{KJ}(\gls{sym:r}, m \arrowvert \gls{sym:aW}) =  \frac{  \gls{sym:aW}  -1}{\pi\gls{sym:hm}^2}
									\left( 1 + \frac{\gls{sym:r}^2}{\gls{sym:hm}^2} \right)^{-\gls{sym:aW}},
			}
			\label{eq:k_kj}
		\end{equation}
\footnotesize
	where \gls{sym:aW} is \glsdesc{sym:aW}.

Woo suggested a \alert{magnitude dependent bandwidth} for the kernel function:  
\small
	\begin{equation}
		\ensuremath{
			h(m\arrowvert \gls{sym:a0}, \gls{sym:a1}) = \gls{sym:a0}e^{\gls{sym:a1}m},
		}
		\label{eq:hm}
	\end{equation}
\footnotesize
where $a_0$ and $a_1$ are defined by regression on average nearest neighbour
distance $h$ on the magnitude bin $m \pm \mathrm{d}m$.

\begin{figure}[H]
  \centering
  \includegraphics[width=.9\textwidth]{woo_bandwidth} 
  \caption{Regression of magnitude dependent bandwidth}
  \label{fig:woo_b} 
\end{figure}


%-------------------------------------------
\section{Helmstetter, 2012}
%-------------------------------------------
\footnotesize
	To their forecast's backgroung seismic rate, \citet{helmstetter_2012} suggested
	an space and time long-term model like 
		\begin{equation}
		\ensuremath{\gls{sym:R} = \gls{sym:Rmin} + \sum_{t_i < t}{ 
			\frac{2\,w(\boldsymbol{r}_i,t_i)}{h_i\,{d_i}^2}
					\gls{sym:Kt}\gls{sym:Kr} }},
			\label{eq:helms02}
		\end{equation}
\footnotesize
	where \gls{sym:Rmin} is \glsdesc{sym:Rmin} and
\small
		\begin{equation}
			\ensuremath{ w(\boldsymbol{r},t) = 10^{ \gls{sym:b}(\boldsymbol{r},t) \left[ \gls{sym:Mc_rt} - \gls{sym:Md}
			\right] } },
			\label{eq:helms_wi}
		\end{equation}
\footnotesize
	where \gls{sym:wi} is the \glsdesc{sym:wi} on location $\boldsymbol{r}$ and
	time $t$, \gls{sym:b}$(\boldsymbol{r},t)$ is the \glsdesc{sym:b}, 
		  \gls{sym:Mc_rt} is the \glsdesc{sym:Mc_rt}, 
		  \gls{sym:Md} is the \glsdesc{sym:Md}.

The \alert{local adaptive bandwidth} in space and time for each earthquake on
catalogue could be computed as
\small
		\begin{equation}
			\ensuremath{
		%		h_i, d_i = \underset{d_i \ge \gls{sym:dk}, h_i \ge \gls{sym:hk}}{\argmin} 
				h_i, d_i = \argmin_{\substack{h_i \ge \gls{sym:hk} \\
								              d_i \ge \gls{sym:dk}}
						           } 
				\left[ s \left(h_i,d_i 
					 		  \arrowvert
							  \gls{sym:k_cnn},\gls{sym:a_cnn}
					     \right) 
					   := h_i + \gls{sym:a_cnn}d_i 
			    \right],
			}
			\label{eq:helms_cnn}
		\end{equation}
\footnotesize
	where \gls{sym:k_cnn} is the \glsdesc{sym:k_cnn},
		 \gls{sym:a_cnn} is the \glsdesc{sym:a_cnn},
		 \gls{sym:dk} is the \glsdesc{sym:dk} and 
		 \gls{sym:hk} is the \glsdesc{sym:hk}.


The stationary seismic rate on cell position $r_0$ is computed as 
\footnotesize
		\begin{equation}
			\ensuremath{
				\bar{R}(\boldsymbol{r}_0) = \text{Median}\left[R(\boldsymbol{r}_0, t)\right].
			}
			\label{eq:helms_mediana}
		\end{equation}
which avoid the declustering methods.

\begin{columns}
	\begin{column}[T]{0.48\textwidth}
		\begin{figure}[H]
		  \centering
		  \includegraphics[width=.85\textwidth]{helmstetter_hidi} 
		  \caption{Bandwidth computation for some earthquake ($k_{cnn} = 5$, $a_{cnn}
		  = 100$)}
		  \label{fig:h_hidi} 
		\end{figure}
	\end{column}

	\begin{column}[T]{0.51\textwidth}
		\begin{figure}[H]
		  \centering
		  \includegraphics[width=1.0\textwidth]{helmstetter_stationary_a} 
		  \caption{Stationary seismic rate for some cell $r_0$}
		  \label{fig:h_stationary} 
		\end{figure}
	\end{column}
\end{columns}


\footnotesize
The available kernel shapes are the gaussian and the power-law:
\begin{columns}
	\begin{column}[T]{0.49\textwidth}
		\footnotesize
		\begin{equation}
			\ensuremath{
				K_{gs}(\gls{sym:r}\arrowvert h) = \eta_1(h)
					e^{- \frac{\|\gls{sym:r}\|^2}
		 				 	  {2 h^2 }},
		 	}
		\label{eq:kernel_gs}
		\end{equation}
	\end{column}

	\begin{column}[T]{0.49\textwidth}
		\footnotesize
		\begin{equation}
			\ensuremath{
				K_{pl}(\gls{sym:r}\arrowvert h) = 
					\frac{\eta_2(h)}
		 				 {\left( \|\gls{sym:r}\|^2 + h^2 \right)^{\frac{3}{2}} },
		 	}
		\label{eq:kernel_pl}
		\end{equation}
	\end{column}
\end{columns}
\footnotesize
where $\eta_1(h)$ e $\eta_2(h)$ are normalization factors.

%===========================================
\newcolumn
%===========================================

\footnotesize
The optimization is done by the maximization of 
\small
		\begin{equation}
			\ensuremath{
				\gls{sym:L} = \sum_{i_x=1}^{N_x}\sum_{i_y=1}^{N_y}\log p\left[  \gls{sym:Np}, \gls{sym:nxy}  \right]}
			\label{eq:loglik}
		\end{equation}
\footnotesize
	where \gls{sym:Np} is \glsdesc{sym:Np},
		\gls{sym:nxy} is \glsdesc{sym:nxy}.

\footnotesize
The log-likelyhood could be computed using the probability of observe exacly $n$
events (from target catalog) with the predicted $N_p$ seismic rate
\small
		\begin{equation}
			\ensuremath{
				\gls{sym:pNn} = \frac{{N_p}^n e^{-N_p}}{n!}.
			}
			\label{eq:loglik}
		\end{equation}
\footnotesize
To eval the model information gain over an uniform spatialy distributed seismic
rate, we compute the Kagan and Knopoff (1977) gain:
\small
		\begin{equation}
			\ensuremath{
			\begin{align}
				G & = e^{\sum_{i = 1}^{N_t}
							\frac{\log \left[  \gls{sym:Npi} / \gls{sym:Nu}  \right]}
								 {\gls{sym:Nt}}
					  } \\
				  & = {\langle  \gls{sym:Npi} / \gls{sym:Nu}  \rangle}_{geom}
			\end{align}}
			\label{eq:G}
		\end{equation}	

\vspace{0.3cm}


\begin{columns}[c,totalwidth=\textwidth]
\column{.5\linewidth}
	\begin{figure}[H]
	  \centering
	  \includegraphics[width=.98\textwidth]{helmstetter_catalogues} 
	  \caption{Learning and target catalogues}
	  \label{fig:h_catalogue} 
	\end{figure}
\column{.5\linewidth}
	\begin{table}[H]
		\centering
		\begin{tabular}{c|c}
			Parameter & Value \\ \hline
			$R_{min}$ & $0.1\times10^{-13}$ \\
			$a_{cnn}$ & 325 \\
			$k_{cnn}$ & 1 \\ \hline
			Gain	  & 2.43
		\end{tabular}
		\caption{Optimized parameters}
		\label{tab:hemlstetter}
	\end{table}
\end{columns}


\vspace{0.3cm}
%-------------------------------------------
\section{Rates and Hazard}
%-------------------------------------------
\footnotesize \textbf{\citet{frankel_1995}}:
\begin{columns}[t,totalwidth=\textwidth]
	\column{.5\linewidth}
		\begin{figure}[H]
		  \centering
		  \includegraphics[width=.98\textwidth]{a_frankel_br} 
		  \caption{a-value map}
		  \label{fig:a_fran_br} 
		\end{figure}
	
	\column{.5\linewidth}
		\begin{figure}[H]
		  \centering
		  \includegraphics[width=.98\textwidth]{pga_frankel} 
		  \caption{PGA(poe0.1,50y)}
		  \label{fig:pga_fran} 
		\end{figure}
\end{columns}

\footnotesize \textbf{\citet{woo_1996}}:
\begin{columns}[t,totalwidth=\textwidth]
\column{.5\linewidth}
	\begin{figure}[H]
	  \centering
	  \includegraphics[width=.98\textwidth]{a_woo} 
	  \caption{a-value map}
	  \label{fig:a_woo} 
	\end{figure}
\column{.5\linewidth}
	\begin{figure}[H]
		\centering
		\includegraphics[width=.98\textwidth]{pga_woo_cum} 
		\caption{PGA(poe0.1,50y)}
		\label{fig:pga_woo_cum} 
	\end{figure}
\end{columns}

\footnotesize \textbf{\citet{helmstetter_2012}}:
\begin{columns}[t,totalwidth=\textwidth]
\column{.5\linewidth}
	\begin{figure}[H]
	  \centering
	  \includegraphics[width=.98\textwidth]{a_helmstetter} 
	  \caption{a-value map}
	  \label{fig:helm_r} 
	\end{figure}
\column{.5\linewidth}
	\begin{figure}[H]
	  \centering
	  \includegraphics[width=.98\textwidth]{pga_helmstetter} 
	  \caption{PGA(poe0.1,50y)}
	  \label{fig:helm_h} 
	\end{figure}
\end{columns}


%-------------------------------------------
\section{References}
%-------------------------------------------
	\scriptsize
%	\bibliography{bib/bibliografia}

\bibitem[Dourado (2014)]{dourado_2014}{\textbf{Dourado (2014)}}
J.C Dourado.
Mapa de amea{\c c}a s{\'\i}smica do brasil.
Em \emph{Congresso Brasileiro de Geologia}.

\bibitem[Frankel (1995)]{frankel_1995}{\textbf{Frankel (1995)}}
Arthur Frankel.
Mapping seismic hazard in the central and eastern united states.
\emph{Seismological Research Letters}, 66\penalty0 (4):\penalty0
  8--21.

\bibitem[Giardini {\rm{\em et~al.}} (1999)Giardini, Gr{\"u}nthal, Shedlock, e
  Zhang]{giardini_1999}{\textbf{Giardini \emph{et~al}} (1999)}
D.~Giardini, G.~Gr{\"u}nthal, K.~M. Shedlock e P.~Zhang.
The {GSHAP} global seismic hazard map.
\emph{Annali di Geofisica}, 42\penalty0 (6):\penalty0 1225--1230.

\bibitem[Helmstetter e Werner (2012)]{helmstetter_2012}{\textbf{Helmstetter e
  Werner (2012)}}
Agn{\`e}s Helmstetter e Maximilian~J. Werner.
Adaptive spatiotemporal smoothing of seismicity for long-term
  earthquake forecasts in california.
\emph{Bulletin of the Seismological Society of America}, 102\penalty0
  (6):\penalty0 2518--2529.

\bibitem[Woo (1996)]{woo_1996}{\textbf{Woo (1996)}}
G.~Woo.
Kernel estimation methods for seismic hazard area source modeling.
\emph{Bulletin of the Seismological Society of America}, 86\penalty0
  (2):\penalty0 353--362.

\end{poster}
\end{document}